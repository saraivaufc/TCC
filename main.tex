\documentclass[
    article,
	% -- opções da classe memoir --
	12pt,				% tamanho da fonte
	openright,			% capítulos começam em pág ímpar (insere página vazia caso preciso)
	oneside,			% para impressão em verso e anverso. Oposto a oneside
	a4paper,			% tamanho do papel.
	%sumario=abnt-6027-2012,
	% -- opções da classe abntex2 --
	chapter=TITLE,		% títulos de capítulos convertidos em letras maiúsculas
	section=TITLE,		% títulos de seções convertidos em letras maiúsculas
	%subsection=TITLE,	% títulos de subseções convertidos em letras maiúsculas
	%subsubsection=TITLE,% títulos de subsubseções convertidos em letras maiúsculas
	% -- opções do pacote babel --
	english,			% idioma adicional para hifenização
	french,				% idioma adicional para hifenização
	spanish,			% idioma adicional para hifenização
	brazil				% o último idioma é o principal do documento
	]{abntex2}


% --- 
% CONFIGURAÇÕES DE PACOTES
% --- 
% ---
% Pacotes básicos 
% ---
\usepackage{lmodern}			% Usa a fonte Latin Modern			
\usepackage[T1]{fontenc}		% Selecao de codigos de fonte.
\usepackage[utf8]{inputenc}		% Codificacao do documento (conversão automática dos acentos)
\usepackage{lastpage}			% Usado pela Ficha catalográfica
\usepackage{indentfirst}		% Indenta o primeiro parágrafo de cada seção.
\usepackage{color}				% Controle das cores
\usepackage{graphicx}			% Inclusão de gráficos
\usepackage{microtype} 			% para melhorias de justificação
\usepackage{ufc-abntex2}
\usepackage{enumitem}
\usepackage{amsmath}
\usepackage{booktabs}
\usepackage{multirow}
\usepackage{titlesec}

%My Packages
\usepackage{threeparttable}
\usepackage{wasysym}


%Deixando o Caption centralizado mesmo com quebra de linha 
\usepackage[labelfont=bf, justification=raggedright, singlelinecheck=false]{caption}

%formatação do Sumário
\usepackage{etoolbox}                               
% Usado para alterar a fonte da Section no Sumário
\usepackage[nogroupskip,nonumberlist,acronym]{glossaries}                %
% ---
		
% ---
% Pacotes adicionais, usados apenas no âmbito do Modelo Canônico do abnteX2
% ---
\usepackage{lipsum}				% para geração de dummy text
% ---

\usepackage[alf, abnt-emphasize=bf, bibjustif, recuo=0cm, abnt-etal-cite=2, abnt-etal-list=0]{abntex2cite} 


\addto\captionsbrazil{
	\renewcommand{\bibname}{REFER\^ENCIAS}
	\renewcommand{\folhaderostoname}{FOLHA DE ROSTO}
	\renewcommand{\epigraphname}{EP\'iGRAFE}
	\renewcommand{\dedicatorianame}{DEDICAT\'ORIA}
	\renewcommand{\errataname}{ERRATA}
	\renewcommand{\agradecimentosname}{AGRADECIMENTOS}
	\renewcommand{\anexoname}{ANEXO}
	\renewcommand{\anexosname}{ANEXOS}
	\renewcommand{\apendicename}{AP\^ENDICE}
	\renewcommand{\apendicesname}{AP\^ENDICES}
	\renewcommand{\orientadorname}{ORIENTADOR:}
	\renewcommand{\coorientadorname}{CO-ORIENTADOR:}
	\renewcommand{\folhadeaprovacaoname}{FOLHA DE APROVA\c{C}\~AO}
	\renewcommand{\resumoname}{RESUMO}
	\renewcommand{\listadesiglasname}{LISTA DE ABREVIATURAS E SIGLAS}
	\renewcommand{\listadesimbolosname}{LISTA DE SÍMBOLOS}
	\renewcommand{\listfigurename}{LISTA DE ILUSTRAÇÕES}
	\renewcommand{\listtablename}{LISTA DE TABELAS}
	\renewcommand{\fontename}{FONTE}
	\renewcommand{\notaname}{NOTA}
	%% adjusts names used by \autoref
	\renewcommand{\pageautorefname}{P\'agina}
	\renewcommand{\sectionautorefname}{Se\c{c}\~ao}
	\renewcommand{\subsectionautorefname}{Subse\c{c}\~ao}
	\renewcommand{\subsubsectionautorefname}{Subse\c{c}\~ao}
	\renewcommand{\paragraphautorefname}{Paragrafo}
	\renewcommand{\indexname}{\'INDICE}
}
\newcommand\fnote[1]{\captionsetup{font=small}\caption*{#1}}



% Ambiente para alineas e e subalineas (incisos) com ponto
\newlist{alineascomponto}{itemize}{2}
\setlist[alineascomponto,1]{label={$\bullet$},topsep=0pt,itemsep=0pt,leftmargin=\parindent+\labelwidth-\labelsep}%
\setlist[alineascomponto,2]{label={--},topsep=0pt,itemsep=0pt,leftmargin=*}
\newlist{subalineascomponto}{enumerate}{1}
\setlist[subalineascomponto,1]{label={$\circ$},topsep=0pt,itemsep=0pt,leftmargin=*}%
% ---

% Ambiente para alineas e e subalineas (incisos) com numeros
\newlist{alineascomnumero}{enumerate}{2}
\setlist[alineascomnumero,1]{label={$\arabic*$.},topsep=0pt,itemsep=0pt,leftmargin=\parindent+\labelwidth-\labelsep}%
\setlist[alineascomnumero,2]{label={--},topsep=0pt,itemsep=0pt,leftmargin=*}
\newlist{subalineascomnumero}{enumerate}{1}
\setlist[subalineascomnumero,1]{label={$\arabic*$.},topsep=0pt,itemsep=0pt,leftmargin=*}%
% ---

% ---
% Configurações de aparência do PDF final

% alterando o aspecto da cor azul
\definecolor{blue}{RGB}{41,5,195}

% informações do PDF
\makeatletter
\usepackage{hyperref}% http://ctan.org/pkg/hyperref
\hypersetup{
     	%pagebackref=true,
		pdftitle={AMBIENTE VIRTUAL DE APRENDIZAGEM PARA AUXILIAR NO PROCESSO DE ENSINO E APRENDIZAGEM DE MATEM\'ATICA}, 
		pdfsubject={TECNOLOGIAS DA INFORMA\c{C}\~AO E COMUNICA\c{C}\~AO PARA AUXILIAR NA EDUCA\c{C}\~AO},
		pdfauthor={MARCIANO MACHADO SARAIVA},
    	pdfcreator={LaTeX with abnTeX2},
		pdfkeywords={Ambiente Virtual de Aprendizagem}{Matem\'atica}{Educa\c{c}\~ao}{Metodologia de Ensino}, 
		colorlinks=false,       		% false: boxed links; true: coloredlinks
    	linkcolor=blue,          	% color of internal links
    	citecolor=blue,        		% color of links to bibliography
    	filecolor=magenta,      		% color of file links
		urlcolor=blue,
		bookmarksdepth=4
}
\makeatother
% --- 


% --- 
% Espaçamentos entre linhas e parágrafos 
% --- 

% O tamanho do parágrafo é dado por:
\setlength{\parindent}{1.3cm}

% Controle do espaçamento entre um parágrafo e outro:
\setlength{\parskip}{0.2cm}  % tente também \onelineskip


% ---
% compila o indice
% ---
\makeindex
% ---



\usepackage{hyperref}% http://ctan.org/pkg/hyperref
\hypersetup{%
  colorlinks = false,
  linkcolor  = black
}


% Minhas Váriaveis
%\newcommand{\SystemType}{Ambiente Computacional Interativo}
%\newcommand{\SystemsTypes}{Ambientes Computacionais Interativos}
\newcommand{\SystemType}{Ambiente Virtual de Aprendizagem}
\newcommand{\SystemsTypes}{Ambientes Virtuais de Aprendizagem}

\newcommand{\SystemName}{AskMath}

\newcommand{\teachinglearning}{ensino/aprendizagem}
\newcommand{\TeachingLearning}{Ensino/Aprendizagem}






% Informações de dados para CAPA e FOLHA DE ROSTO
\titulo{\SystemType~para Auxiliar no Processo de Ensino e Aprendizagem de Matemática}
\autor{Marciano Machado Saraiva}
\local{Quixadá}
\data{Maio, 2016}
\orientador{Prof. Msc. Samy Soares Passos de Sá}
%\coorientador{Nome Coorientador}

% Escolher curso: Redes de Computadores (rc), Eng.Software (es), Ciências da Computação (cc) ou Sist.Informação (si)
%\include{fixos/instituicao/rc}
\instituicao{%
UNIVERSIDADE FEDERAL DO CEAR\'A \par
CAMPUS QUIXAD\'A \par
CURSO DE SISTEMAS DE INFORMA\c{C}\~AO
}
\tipotrabalho{Trabalho de Conclusão de Curso (Monografia)}
\preambulo{Monografia apresentada ao Curso de Sistemas de Informação do Campus Quixadá da Universidade Federal do Ceará, como requisito parcial para obtenção do Título de Bacharel em Sistemas de Informação.}

%\include{fixos/instituicao/es}
%\include{fixos/instituicao/cc}

%%criar um novo estilo de cabeçalhos e rodapés
\makepagestyle{header_style}
  %%cabeçalhos
  \makeoddhead{header_style} %%pagina ímpar ou com oneside
     {}
     {}
     {\thepage} 


\begin{document}
\frenchspacing 

%Formatação de título de seções
\titleformat{\section}
{\normalfont\normalsize\bfseries}{\thesection}{1em}{}
\titleformat{\subsection}
{\normalfont\normalsize\bfseries}{\thesubsection}{1em}{}
\titleformat{\subsubsection}
{\normalfont\normalsize\bfseries}{\thesubsubsection}{1em}{}

% ----------------------------------------------------------
% ELEMENTOS PRÉ-TEXTUAIS
% ----------------------------------------------------------
% \pretextual
% Capa
\imprimircapa

%----------- Apenas TCC 2
% Folha de rosto (* indica que haverá a ficha bibliográfica)
%\imprimirfolhaderosto

% Ficha Bibliográfica
%\include{fixos/fichabibliografica}

% Errata
%\include{editaveis/errata}

% Folha de Aprovação
% DEVE ser modificada para adicionar os membros da banca
%% ---
% Inserir folha de aprovação
% ---

% Isto é um exemplo de Folha de aprovação, elemento obrigatório da NBR
% 14724/2011 (seção 4.2.1.3). Você pode utilizar este modelo até a aprovação
% do trabalho. Após isso, substitua todo o conteúdo deste arquivo por uma
% imagem da página assinada pela banca com o comando abaixo:
%
% \includepdf{folhadeaprovacao_final.pdf}
%
\begin{folhadeaprovacao}
	\begin{center}
		{\ABNTEXchapterfont\large\imprimirautor}

		\vspace*{\fill}\vspace*{\fill}
		{\ABNTEXchapterfont\large\imprimirtitulo}
		\vspace*{\fill}

		\hspace{.45\textwidth}
		\begin{minipage}{.5\textwidth}
		\imprimirpreambulo
		\end{minipage}%
		\vspace*{\fill}
	\end{center}

	Aprovada em \_\_/\_\_/\_\_\_\_

	\begin{center}
		BANCA EXAMINADORA
	\end{center}

	\assinatura{\textbf{\imprimirorientador (Orientador)} \\ Universidade Federal do Ceará(UFC)}
	\assinatura{\textbf{\imprimirmembroum} \\ Universidade Federal do Ceará(UFC)}
	\assinatura{\textbf{\imprimirmembrodois} \\ Universidade Federal do Ceará(UFC)}


\end{folhadeaprovacao}
% ---

%\imprimirfolhadeaprovacao

% Dedicatória
%% ---
% Dedicatória
% ---
\begin{dedicatoria}
	\vspace*{\fill}
	\hspace{8cm}
	\begin{minipage}{.5\textwidth}
	\SingleSpacing
	Este trabalho é dedicado às crianças adultas que, quando pequenas, sonharam em se tornar cientistas.
	\end{minipage}%
\end{dedicatoria}
% ---

% Agradecimentos
%Ao meu orientador, Prof. Me. Samy Soares Passos de S\'{a}, pela paci\^{e}ncia na orienta\c{c}\~{a}o, pelo conv\'{\i}vio, pelo apoio e, principalmente, pela compreens\~{a}o.

Aos professores Dr. Wladmir Araujo Tavares e Me. Carlos Roberto Rodrigues Filho pelas considera\c{c}\~{o}es na proposta desse trabalho que permitiram qualifica-lo e pelas cr\'{\i}ticas construtivas feitas em torno deste trabalho.

Ao grupo PET-Tecnologia da Informa\c{c}\~{a}o da Universidade Federal do Cear\'{a}, pela proposta desse trabalho e por todo suporte que possibilitou o desenvolvimento dessa ferramenta.

Agrade\c{c}o a todos os professores por me proporcionarem o conhecimento n\~{a}o apenas racional, mas a manifesta\c{c}\~{a}o do car\'{a}ter e afetividade da educa\c{c}\~{a}o no processo de forma\c{c}\~{a}o profissional, por tanto que se dedicaram a mim, n\~{a}o somente por terem me ensinado, mas por terem me feito aprender. 




% Epígrafe
%``A matem\'atica \'e a única linguagem que temos em comum com a natureza.''

\autordaepigrafe{Stephen Hawkings}

% Deixe o espaço entre a epigrafe e o autor.

% RESUMOS
%\include{resumo/ptbr}
%% resumo em inglês
\begin{resumo}[ABSTRACT]
 \begin{otherlanguage*}{english}
   This is the english abstract.

   \vspace{\onelineskip}
 
   \noindent 
   \textbf{Keywords}: latex. abntex. text editoration.
 \end{otherlanguage*}
\end{resumo}
%\include{resumo/fr}
%\include{resumo/es}

% Lista de ilustrações
% \pdfbookmark[0]{\listfigurename}{lof}
% \listoffigures*
% \clearpage

% Lista de tabelas
%\pdfbookmark[0]{\listtablename}{lot}
%\listoftables*
%\cleardoublepage

% Abreviaturas e Siglas
%% Lista de abreviaturas e siglas
% ---
\begin{siglas}
	\item[UFC] Universidade Federal do Cear\'a
	\item[AVA] Ambiente Virtual de Aprendizagem
	\item[TIC] Tecnologia da Informa\c{c}\~ao e Comunica\c{c}\~ao
\end{siglas}
% ---

% Símbolos
%\include{editaveis/simbolos}
%----------- fim - Apenas TCC 2


% Sumário

\imprimirsumario

% ----------------------------------------------------------
% ELEMENTOS TEXTUAIS
% ----------------------------------------------------------
\textual

%aplicação de estilo de cabeçalho
\pagestyle{header_style}

%uso do input pois o include dá quebra de página no final
\section{INTRODUÇÃO}

A Matemática, como ciência, sempre teve uma relação muito especial com as novas tecnologias, desde as calculadoras, os computadores, aos sistemas multimédia e a Internet. No entanto, os professores costumam demorar a perceber como tirar partido destas tecnologias como ferramentas de trabalho. \cite{da1997ensino}. 

A medida que a quantidade de recursos tecnológicos na sala de aula foram aumentando, tornou-se necessário a criação de novas metodologias de ensino, especificamente na Educação Matemática. Tal busca procura fazer da Matemática uma disciplina atraente, desvinculada do ensino tradicional que já se mostrou ineficiente \cite{silva2009ambiente}.

Tendo em vista essa necessidade, o presente trabalho, busca apresentar o projeto e desenvolvimento de um Ambiente Virtual de Aprendizagem (AVA) \cite{valentini2010aprendizagem}, para auxiliar estudantes no ensino e aprendizagem de conteúdos matemáticos. Este ambiente se propõe a servir como ferramenta para estudantes que buscam estudar fora do ambiente escolar e de uma forma auto-ritmada, ou seja, no seu próprio ritmo. Entretanto, o ambiente também dará suporte ao ensino e aprendizagem dentro da sala de aula, auxiliando professores com informações relevantes sobre o andamento do aprendizado de cada um de seus aluno, além das dificuldades que os mesmos apresentam.

Grande parte dos alunos tem dificuldades em aprender matemática, e muitas vezes essas dificuldades ocorrem não pela falta de atenção ou por não gostar do conteúdo, mas por fatores mentais ou psicológicos que envolvem uma série de trabalhos
e conceitos que precisam ser desenvolvidos \cite{sa2015software}. Mas como auxiliar alunos com dificuldades na aprendizagem da matemática?

Em busca dessa resposta, diferentes sistemas de \textit{softwares} foram desenvolvidos buscando servir a educação. Em 2006, Salman Khan funda a Khan Academy, uma organização educacional que tem por objetivo oferecer exercícios, vídeos de instrução e um painel de aprendizado personalizado que habilita os estudantes a aprender no seu próprio ritmo dentro e fora da sala de aula \cite{khan2012one}. A plataforma criado por khan utiliza de vídeo-aulas e resolução de problemas para ensinar seus alunos, permitindo assim, que cada um tenha uma aprendizagem de forma independente e auto-ritmada.

Um outro trabalho também  importante nessa área, podemos citar o de  \citeonline{melis2001activemath}, onde os autores desenvolvem um AVA que permite os alunos desfrutarem da experiência de estudar num curso gerado dinamicamente a partir do estado em que o mesmo se encontra dentro do sistema. Estes e outros trabalhos são apresentados com mais detalhes em trabalhos relacionados.

\subsection{Objetivos do Trabalho}

Este trabalho visa contribuir com o processo de ensino e aprendizagem de matemática, buscando projetar e desenvolver um ambiente virtual que auxilie no processo de ensino e aprendizagem de matemática por alunos, dentro e fora da sala de aula. 

Como objetivos específicos para este trabalho, temos: 
\begin{alineascomponto}
    \item Elaborar um método para diagnosticar, nos alunos, dificuldades existentes em certos conteúdos de matemática.
	\item Desenvolver o AVA, aplicando técnicas de gamificação.
    \item Aplicar o AVA numa turma de graduação onde os alunos estejam cursando disciplinas de matemática.
    \item Analisar a influência e impactos provocados pelo AVA, por meio de dados gerados pela ferramenta ao longo de sua utilização e questionários aplicados aos alunos que utilizarem o AVA.

\end{alineascomponto}


\subsection{Divisão do Trabalho}

Fundamentando-se na problemática mencionada, e tendo em vista o objeto de estudo, esta dissertação foi dividida em cinco capítulos. No capítulo inicial é feita uma apresentação do tema a ser discutido durante toda esta dissertação e o objetivo do trabalho.

No Capítulo 2, destacam-se os aspectos teóricos sobre ensino e aprendizagem, assim como as tradicionais metodologias de ensino e as apoiadas por computador, além de conceitos sobre gamificação e os trabalhos que serviram de referencial para abordar os conceitos e ideias utilizadas no trabalho aqui desenvolvido.

No Capítulo 3 compreende-se a concepção, construção e modelagem do sistema, apresentando o que o mesmo deve possuir e por que.

No Capítulo 4 são apresentados os resultados preliminares.

\section{REVISÃO BIBLIOGRÁFICA}

Este capítulo aborda os seguintes temas: metodologias no ensino da matemática, conceito de ambientes virtuais de aprendizagem, o uso da gamificação aplicada em AVAs, e, por fim, o que está sendo 
desenvolvido por outros pesquisadores da área.

\subsection{Metodologias no Ensino da Matemática}

Ao longo da história, várias metodologias e abordagens matemáticas foram utilizadas visando a melhoria do ensino. Segundo \citeonline{hammes2003tendencias}, algumas delas foram aula expositiva, 
resolução de problemas, modelagem matemática, e o uso de computadores. Nas seções a seguir, descreveremos cada uma delas.

\subsubsection{Aula expositiva}

Em uma aula expositiva, o professor comumente faz uma revisão da aula anterior, apresenta o novo conteúdo e passa aos alunos uma série de exercícios de fixação. Esse novo conteúdo é apresentado de 
forma oral ou escrita, sem levar em consideração o conhecimento prévio dos estudantes nem tempo para perguntas. Essa é, sem dúvida, uma das mais utilizadas e antigas metodologias existentes. Durante 
o século passado, aulas expositivas foram o único processo empregado em sala de aula pelos professores. Dessa forma, a aula expositiva pode ser considerada cansativa e desinteressante, já que o aluno 
não participa do processo de ensino  \cite{hammes2003tendencias}.

Para \citeonline{de1996gerencia}, essa abordagem possui diversos problemas. \citeonline{de1996gerencia} sugere que a escola tradicional não somente está desatualizada para atender às necessidades 
crescentes da sociedade contemporânea, como também apresenta algumas características que inibem o desenvolvimento do potencial de criação dos alunos:
[Uniformizar esta lista de pontos, n\~ao sei o que significa :(]
\begin{alineascomponto}
\item Destaca-se a incompetência, a ignorância e a incapacidade do aluno, deixando de assinalar os talentos e habilidades de cada um; 
\item O ensino voltado para o passado, onde se enfatiza a reprodução e a memorização do conhecimento;
\item Desconsidera-se a imaginação e a fantasia como dimensões importantes da mente;
\item Exercício de resposta única, onde se cultua o medo do erro e do fracasso;
\item A obediência, dependência, passividade e conformismo são os traços mais cultivados;
\item Descaso em cultivar uma visão otimista do futuro;
\item As habilidades cognitivas são desenvolvidas de forma limitada;
\end{alineascomponto}

[REVER ISSO]
Autores como \citeonline{lopes1995aula}, defendem que a aula expositiva ``poderá ser transformada em uma atividade dinâmica, participativa e estimuladora do pensamento critico do aluno''. 
Mas os mesmos afirmam que isso é uma tarefa difícil, já que, em geral, é mal empregado pelos professores. Esse autor complementa ainda que essa nova abordagem ``valoriza a vivência dos alunos, seu 
conhecimento do concreto, e busca relacionar esses conhecimentos prévios com o assunto a ser estudado''. Ainda de acordo com \citeonline{lopes1995aula}, o professor

\begin{citacao}
``[...] jamais desconsidera uma pergunta em aula, mesmo que ela possa lhe parecer ingênua ou despropositada. Ao perceber uma pergunta mal formulada o papel do professor é ajudar o aluno a refazer a pergunta pois essa atitude educa o aluno a aprender a perguntar''.
\end{citacao}

\subsubsection{Resolução de problemas}

A resolução de problemas deve ser entendida como uma oportunidade para o aluno obter novos conhecimentos e não apenas conhecimentos prontos que fazem parte da nossa história. Ela ajuda, o aluno a 
desenvolver sua autonomia buscando as respostas para seus próprios questionamentos. \citeonline{fossa1998tendencias} ressaltam  que a resolução de problemas 
[Adicionar P\'agina]
\begin{citacao}
``[...] visa o desenvolvimento de habilidades metacognitivas, favorecendo a todo momento a reflexão e o questionamento. O aluno aprende a pensar por si mesmo, levantando hipóteses, testando-as, 
tirando conclusões e até discutindo-as com os colegas.'' \cite{fossa1998tendencias}. 
\end{citacao}

Nessa metodologia, é importante que os problemas apresentem uma incógnita que necessite ser descoberta. Para resolvê-los, o aluno terá que inventar estrat\'egias e gerar novas ideias. Segundo 
\citeonline{dante1991didatica}
[Adicionar P\'agina]
\begin{citacao}
``É importante que o problema possa gerar muitos processos de pensamento, levantar muitas hipóteses e propiciar várias estratégias de solução. O pensar e o fazer criativo devem ser componentes fundamentais no processo de resolução de problemas.'' \cite{dante1991didatica}.
\end{citacao}

\subsubsection{Modelagem Matemática}
[REVER TODA ESSE SE\c{C}\^AO]
Modelagem Matemática diz respeito ao processo de criação de um modelo matemático. \apudonline{granjer1997modelagem}{biembengut1999modelagem} define um modelo matemático como sendo...
\begin{citacao}
``[...] uma imagem que se forma na mente, no momento em que o espírito racional busca compreender e expressar de forma intuitiva uma sensação, procurando relacionar com algo já conhecido, efetuando deduções.''\apud[p.~78]{granjer1997modelagem}{biembengut1999modelagem}
\end{citacao}

A modelagem matemática é uma metodologia que busca proporcionar ao aluno uma visão prática do conhecimento teórico aprendido na sala de aula, através de problemas de ordem prática. Para \citeonline{fossa1998tendencias}
\begin{citacao}
``[...] a modelagem matemática começa com um grande problema de ordem prática ou de natureza empírica e depois busca a Matemática que deveria ser utilizada para ajudar a resolver a situação problemática.'' \cite[p.~15]{fossa1998tendencias}.
\end{citacao}

Diante do exposto, a principal diferença entre a resolução de problemas e a modelagem matemática, é que, na primeira, o problema é sempre posto e proposto pelo professor, na segunda, o problema deve ser proposto em conjunto, tanto pelo professor como pelo aluno, sendo que o passo inicial deve ser sempre do aluno.

\subsubsection{O Uso de Computadores}

A aprendizagem mediada por computadores surgiu em 1960 na Universidade de Illinois com o projeto PLATO (Programmed Logic for Automatic Teaching Operations)\cite{bitzer1961plato}, que deu origem ao 
primeiro sistema de ensino assistido por computador, o qual permitia a criação e apresentação de materiais sobre gram\'atica (passar um verbo para o passado, reescrever um substantivo no plural, 
etc.) com revisão automática. De acordo com \citeonline{woolley1994plato}, PLATO apoiou inicialmente apenas uma única sala de aula com 20 alunos, até que em 1972, o sistema migrou para uma nova 
geração de \textit{mainframes} que acabaria por apoiar milhares de terminais gráficos distribuídos em todo o mundo.

O principal fator motivador para a introdução do computador na educação, segundo \citeonline{silva2009ambiente}, foi o surgimento no final do século XX de um conhecimento baseado em simulação, 
característico da cultura informática, o que fez com que o computador fosse visto como  um recurso didático  indispensável.

Os benefícios da utilização do computador como um instrumento de ensino e aprendizagem, de acordo com \citeonline[p.12]{almeida2000proinfo}, referem-se a sua utilização como ``uma máquina que 
possibilita testar ideias ou hipóteses, que levam à criação de um mundo abstrato e simbólico, ao mesmo tempo em que permite introduzir diferentes formas de atuação e interação entre as pessoas''. Já a 
principal associação de professores de matemática dos Estados Unidos (NCTM)\footnote{nota de rodap\'e com o nome da associa\c{c}\~ao e o link do site}, diz que ``a tecnologia é essencial no ensino e 
na aprendizagem da Matemática'' e ``influencia a Matemática que é ensinada e melhora a aprendizagem dos alunos'' permitindo que estes se concentrem ``nas decisões a tomar, na reflexão, no raciocínio e 
na resolução de problemas''. \cite[p.26]{melo2007principios}.

Quando se fala no uso das Tecnologias da Informação e Comunicação (TIC) como um mediador para o processo de ensino e aprendizagem, muitas vezes, acaba-se esquecendo do papel do professor nesse 
processo. Num mundo em que há uma grande variedade de formas de utilização das TIC para apoio a educação, cabe ao professor decidir como e quando utilizar essas tecnologias e quais s\~ao formas 
mais eficazes para sua utilização levando em consideração os conteúdos que serão ofertados. Para isso, o professor necessita mudar sua metodologia de ensino, e isso pode resultar em problemas, já 
que, assim como afirmam \citeonline{bitner2002integrating}:

\begin{citacao}
``Adults do not change easily. Change of any kind brings about fear, anxiety, and concern. Using technology as a teaching and learning tool in the classroom does so to an even greater extent since it involves both changes in classroom procedures and the use of often-unfamiliar technologies. Those responsible for asking teachers to use technology in the curriculum should be aware that fears and concerns do exist.'' \cite{bitner2002integrating}.
\end{citacao}

[COLAGEM]
Os problemas com a inserção do computador na educação podem ser ainda maiores. Segundo \citeonline[p.38]{silva2009ambiente}, ``o uso do computador na educação pode ser problemático, tendo em vista que muito se cogita sobre seu uso no ensino ser a solução para muitos dos problemas da educação''; o referido ainda complementa que ``a maioria destes problemas não podem encontrar resposta nas tecnologias digitais o que pode resultar em uma visão muito simplista sobre o software e seu uso''.

Um dos fatores que pode influência negativamente no processo de aprendizagem mediada por computador é o domínio do computador pelo aluno, tendo em vista que sua rapidez de evolução assim como sua própria complexidade, torna essa tarefa muito difícil de ser alcançada.



\subsection{Ambientes Virtuais de Aprendizagem}

O Surgimento de Ambientes Virtuais de aprendizagem (AVAs) deu-se logo após o surgimento da internet nos anos 90. Nessa mesma época, novas ferramentas e produtos foram desenvolvidas para explorar os benefícios que a rede mundial de computadores trouxe \cite{oleary2002virtual}.

Para \citeonline{valentini2010aprendizagem}, um Ambiente Virtual de Aprendizagem (AVA) é um espaço social, constituído de interações cognitivo-sociais sobre, ou em torno de, um objeto de conhecimento, no qual as pessoas interagem mediadas pela linguagem da hipermídia visando o processo de ensino-aprendizagem.

Geralmente, os AVAs possuem algumas características que os distinguem de outros tipos de sistemas de softwares, segundo \citeonline{oleary2002virtual}, algumas dessas características são:
\begin{alineascomponto}
    \item a comunicação entre tutores e alunos - por exemplo, email, fórum de discussão e bata-papo virtual;
    \item a auto-avaliação e avaliação sumativa - por exemplo, avaliação de múltipla escolha com automatizada marcação e feedback imediato; 
    \item entrega de recursos de aprendizagem e materiais - por exemplo, através do fornecimento de notas de aula e materiais, imagens e clips de vídeo;
    \item áreas do grupo de trabalho compartilhados - possibilita os usuários compartilhares arquivos, bem como se comunicarem;
    \item suporte para estudantes - possibilitar a comunicação entre os tutores e seus estudantes, fornecimentos de materiais didáticos e alguma forma de tirar as dúvidas dos alunos;
    \item gestão e acompanhamento dos estudantes - sistema de autenticação para permitir que apenas estudantes tenham acesso aos cursos;
    \item ferramentas para o estudante - por exemplo agendas e calendários eletrônicos;
    \item aparência consistente e personalizável - uma interface padrão de fácil utilização, permitindo uma personalização, mas um modo de utilização essencial para permanecer constante.
\end{alineascomponto}

A utilização de AVAs traz diversas vantagens como cita \citeonline[p.153]{tajra2001ferramentas}: acessibilidade a fontes inesgotáveis de assuntos para pesquisas, páginas educacionais específicas para a pesquisa escolar, comunicação e interação com outras escolas, estímulo para pesquisar a partir de temas previamente definidos ou a partir da curiosidade dos próprios alunos, estímulo ao raciocínio lógico, troca de experiências entre professores/professores, aluno/aluno e professor/aluno, entre outras.

\citeonline{carvalho2013ambiente} afirma que AVAs integram funcionalidades de comunicação e partilha de informações e isso, permite aceder à aprendizagem de uma forma flexível; em qualquer espaço(\textit{anywhere}) e em qualquer hora (\textit{anytime}), o autor complementa que:

\begin{citacao}
``Um AVA deve, por um lado, enfatizar a aprendizagem através da integração de ferramentas interativas e comunicativas, da partilha de conteúdos multimédia, do alojamento de trabalhos e projetos, da integração de ferramentas de aprendizagem colaborativa, e por outro, deve proporcionar estratégias que potenciem a participação ativa e significativa dos alunos, abranger possibilidades didáticas de aprendizagem individual e em grupo, criar novos acessos a websites como forma de enriquecer o conhecimento, possuir ferramentas de controlo de acesso e registro de utilizadores e de gestão de grupos de trabalho'' \cite{carvalho2013ambiente}. 
\end{citacao}

Os AVAs, geralmente, utilizam diversas ferramentas para apoio ao ensino como já foi citado anteriormente, alguns exemplos são fóruns, chats, wikis, glossários, portfólios, enquetes, questionários, entre outros. Essas ferramentas, de acordo com \citeonline{masetto2012competencia}, são recursos em linguagem digital e podem colaborar significativamente para tornar a educação mais eficiente e eficaz.


\subsection{Gamificação}

Ao longo da história, o homem sempre buscou metodologias inovadoras para auxiliar a educação e  uma das que mais diferem do tradicional método de ensino é a utilização de jogos para o ensino e aprendizagem. 

Em 2002, por meio de Nick Pelling, programador de computadores e inventor britânico, surgiu o termo Gamificação. \citeonline{fardo2013gamificaccao} define a Gamificação como o emprego de conceitos e técnicas tais como sistema de feedback, sistema de recompensas, conflito, cooperação, competição, objetivos e regras claras, níveis, tentativa e erro, diversão, interação, interatividade, entre outros, criados e utilizados em jogos para auxiliar na educação. 

O objetivo da Gamificação não é transformar tudo em um jogo, mas sim, segundo \citeonline{halliwell2013gamification}, ``[...] encontrar a diversão, encontrar os aspectos `jogáveis' de um problema, quaisquer que sejam, e usá-los para criar um ambiente que mova as pessoas um pouco mais em direção a um objetivo que tenham criado''. Um exemplo bem interessante do uso de gamificação é o Duolingo \cite{von2013duolingo}, uma plataforma online de aprendizagem em línguas que trabalha com o conceito de conhecimento coletivo e voluntário. No Duolingo, usuários podem ganhar pontos com as respostas corretas e lições completadas, assim como perder pontos a cada resposta incorreta. O mesmo ainda atribui status de reconhecimento de acordo o conhecimento obtido durante o jogo.

\subsection{Trabalhos Relacionados}

Os \SystemsTypes~vêm sendo utilizados na educação, principalmente como uma ferramenta mais dinâmica comparada as metodologias de ensino tradicionais e são de grande potencial na área da educação. Existem diversas pesquisas voltadas a aplicação de \SystemsTypes~para apoiar o processo de \teachinglearning, em especial a Matemática se destaca entre elas.

\subsubsection{\textit{The one world schoolhouse: Education reimagined}}
Num trabalho iniciado em 2006, \citeonline{khan2012one} funda a chamada KhanAcademy\footnote{Plataforma de aprendizagem disponível em: \url{www.khanacademy.org}}, organização educacional que tem por objetivo oferece exercícios, vídeos de instrução e um painel de aprendizado personalizado que habilita os estudantes a aprender no seu próprio ritmo dentro e fora da sala de aula. Em sua plataforma, são abordados assuntos como matemática, ciência, programação de computadores, história, história da arte, economia e muito mais \cite{khan2012one}.

No ambiente, o desempenho do estudante é representado por medalhas. De acordo com o site da organização, medalhas e insignias estimulam o aprendizado de maneira lúdica. As estatísticas mostram o quanto de trabalho o estudante está fazendo a cada dia, o quanto o estudante está focado em áreas de habilidades e tópicos e as habilidades que o estudante concluiu. Com os relatórios gerados pela plataforma, o tutor pode acompanhar todos os passos do aluno.

Durante uma conferência TED\footnote{É uma série de conferências realizadas na Europa, Ásia e Américas pela fundação Sapling com o objetivo de disseminar ideias que podem mudar o mundo.} em 2011, denominada \textit{Let's use video to reinvent education} \cite{tedtalk2011reinvend}, Khan fala sobre o funcionamento da plataforma e também do provável motivo do seu sucesso. Segundo \citeonline{tedtalk2011reinvend}, o diferencial da plataforma está em dois pontos importantes, a aprendizagem auto-ritmada e nos dados fornecidos aos tutores sobre o aprendizado de seus alunos. 

Em relação a aprendizagem auto-ritmada, Khan afirma que:
\begin{citacao}
``When you talk about self-paced learning, it makes sense for everyone -- in education-speak, ``differentiated learning'' -- but it's kind of crazy, what happens when you see it in a classroom. Because every time we've done this, in every classroom we've done, over and over again, if you go five days into it, there's a group of kids who've raced ahead and a group who are a little bit slower. In a traditional model, in a snapshot assessment, you say, ``These are the gifted kids, these are the slow kids. Maybe they should be tracked differently. Maybe we should put them in different classes.'' But when you let students work at their own pace -- we see it over and over again -- you see students who took a little bit extra time on one concept or the other, but once they get through that concept, they just race ahead.And so the same kids that you thought were slow six weeks ago, you now would think are gifted. And we're seeing it over and over again. It makes you really wonder how much all of the labels maybe a lot of us have benefited from were really just due to a coincidence of time.'' \cite{tedtalk2011reinvend}.
\end{citacao}


\citeonline{tedtalk2011reinvend} também fala sobre a importância dos dados fornecidos pelos tutores:
\begin{citacao}
``[...]. So our paradigm is to arm teachers with as much data as possible -- data that, in any other field, is expected, in finance, marketing, manufacturing -- so the teachers can diagnose what's wrong with the students so they can make their interaction as productive as possible. Now teachers know exactly what the students have been up to, how long they've spent each day, what videos they've watched, when did they pause the videos, what did they stop watching, what exercises are they using, what have they focused on? [...]'' \cite{tedtalk2011reinvend}.
\end{citacao}

Esse trabalho, assim como o descrito aqui, apresenta o uso de um \SystemType~para auxiliar na educação. Dessa forma, esse trabalho servirá como referência para abordar o uso da aprendizagem auto-ritmada na educação matemática, assim como, para definir as informações que deverão ser apresentadas aos tutores sobre a evolução da aprendizagem de seus alunos na plataforma fruto deste trabalho. Contudo, os vídeos que fizeram da Khan Academy tão popular, não farão parte desse trabalho, tendo em vista que, muitos especialistas em educação como Célia Maria Carolina Pires\footnote{Professora da área de didática da Matemática na PUC-SP e pesquisadora de inovações curriculares na Educação Básica e na formação de Professores de Matemática.} acreditam que os vídeos de Khan estão indo contra do que é discutido hoje sobre educação matemática \cite{cartacapital2013} e Fredric Litto\footnote{Presidente da Associação Brasileira de Ensino a Distância (Abed)} que a iniciativa não é revolucionária como muitos dizem, mas sim ``ultrapassada''. Para ele ``O trabalho do Salman Khan tem alguns aspectos novos e outros mais do que tradicionais. O uso de vídeos para melhorar o conhecimento dos alunos, por exemplo, não é novo'' e que ``é a mesma coisa que um professor que dá sua aula em frente ao quadro-negro e o aluno apenas copia no caderno'' \cite{revistaeducacao2013}.

O aspecto mais importante a se considerar aqui está na forma como as duas plataformas lidam com Obstáculos Epistemológicos, segundo \citeonline{bachelard1996formaccao} durante o ato do conhecimento ocorrem ``lentidões e conflitos'', que levam o aluno a parar diante do problema. A esta ``inércia'' é que foi relacionado o conceito. Na metodologia criada por \citeonline{khan2012one}, quando a plataforma é aplicada dentro da sala de aula, o professor pode identificar através da ferramenta, os alunos que estão com esses obstáculos e o mesmo pode intervir para ajudar o aluno a superar essa barreira. No trabalho apresentado aqui, essa barreira será superada quando o sistema, ao identificar o obstáculo, oferecer ajuda ao aluno, caso o mesmo aceite a ajuda, o sistema enviara esse pedido de ajuda a todos os outros alunos que já concluíram a mesma lição com certo nível de proficiência, para que os mesmo possam ou não acatar esse pedido de ajuda e assumir o papel do professor na metodologia de \citeonline{khan2012one}. 


\subsubsection{\textit{ActiveMath: A generic and adaptive web-based learning environment}}

O projeto ActiveMath visa apoiar a aprendizagem verdadeiramente interativa, exploratória e assume que o estudante deve ser responsável por seu aprendizado, até certo ponto. Portanto, uma relativa liberdade para navegar através de um curso e para as escolhas de aprendizagem é dada e, por padrão, o modelo de usuário é inspecionável e modificável \cite{melis2001activemath}.

\citeonline{melis2001activemath} afirma que a maioria dos sistemas tutores inteligentes não contam com uma escolha de adaptação de conteúdos e isso, segundo ele, pode influenciar quando o público-alvo for alunos de faculdades e universidade, já que, diferentemente das escolas de ensino fundamental,  um mesmo assunto é
ensinado de forma diferente para diferentes grupos de utilizadores e em contextos diferentes \cite{melis2001activemath}.

Para conseguir toda essa dinâmica, ActiveMath utiliza regras pedagógicas que definem em quais momentos determinadas funcionalidades do sistema estarão disponíveis, em que ordem os conteúdos serão apresentados para os alunos e como os mesmos deverão ser apresentados.

O trabalho referido,  assim como o desenvolvido por \citeonline{khan2012one} descrito anteriormente, descreve a criação de um \SystemType. Entretanto, o mesmo utiliza técnicas que permitem a geração dinâmica de cursos para os alunos.


\subsubsection{\textit{Resolução de problemas em ambientes virtuais de aprendizagem num curso de licenciatura em matemática na modalidade a distância}}

Esse trabalho desenvolvido por \citeonline{dutra2011resoluccao}, trata da   utilização da metodologia de Resolução de Problemas em ambientes virtuais de  aprendizagem, com o objetivo de investigar que contribuições pode trazer para 
alunos da Licenciatura em Matemática da Universidade Federal de Ouro Preto (UFOP), na Educação a Distância (EaD). Para isso, o ambiente desenvolvido por \citeonline{dutra2011resoluccao} utiliza de fóruns semanais, para discussão e resolução dos problemas; além de chats, utilizados ao final de algumas atividades e um questionário final a ser respondido pelos alunos na última semana de aula.

Essa metodologia segundo \citeonline{dutra2011resoluccao}, funciona seguindo os seguintes passos:
\begin{enumerate}
	\item A atividade é postada na Plataforma Moodle\footnote{``A  palavra  Moodle  referia-se  originalmente  ao  acrônimo:  `Modular Object-Oriented  Dynamic  Learning  Environment'(...).  Em  inglês  a  palavra Moodle é também um verbo que descreve a ação que, com frequência, conduz a resultados criativos, de deambular com preguiça, enquanto se faz com gosto o  que  for  aparecendo  para  fazer''. O  Moodle  deu  o  nome  a  uma  plataforma  de  e-learning,  de  utilização livre  e  código  fonte  aberto,  pela  mão  de  Martin  Dougiamas \cite{oro29585}.} no início da semana, pela manhã (segunda-feira).  Assim,  as  atividades  são  distribuídas  aos  alunos  para  que possam ler, interpretar e entender o problema. 
	\item Os  alunos  passam a  semana  postando  suas  resoluções  dos  problemas e discutindo-as no fórum com os colegas, por meio da Plataforma Moodle. 
	\item A  pesquisadora  observa,  incentivava  e  participava  do  processo  de discussão. Ajuda nos problemas secundários, dando \textit{feedback} das resoluções postadas, respondendo e fazendo perguntas, tirando  dúvidas, acompanhando de perto as discussões entre os alunos no fórum.
	\item As impressões dos alunos sobre os problemas, no início da semana seguinte, e a formalização dos resultados são apresentadas nos chats semanais. Trata-se  de  uma  plenária  virtual  para  discutir  os  problemas,  finalizando-a  com  uma solução aceita por todos. 
	\item Uma resolução, após o chat, é postada na Plataforma Moodle, para todos os pesquisados, observando os conteúdos apresentados nos problemas. 
\end{enumerate}

O ambiente desenvolvido por \citeonline{dutra2011resoluccao} inspirou este trabalho na utilização da metodologia de Resolução de Problema aplicada num AVA através de fóruns e chats, entretanto, a forma como essas ferramentas darão suporte ao ensino em nossa metodologia será adaptada. O fórum que na metodologia de \citeonline{dutra2011resoluccao} era utilizado para a postagem da resolução dos problemas, será utilizado como ferramenta para os alunos tirarem dúvidas sobre os problemas apresentados, já o chat utilizado para discutir os problemas na metodologia de \citeonline{dutra2011resoluccao}, será utilizado para permitir que alunos que concluíram determinado conteúdo com certa proficiência, possa ajudar outros alunos que enfrentem obstaculos epistemológicos ao longo do aprendizado desse conteúdo.


\subsubsection{\textit{A Gamificação Aplicada em Ambientes de Aprendizagem}}

Nesse trabalho desenvolvido por \citeonline{fardo2013gamificaccao}, apresenta-se um conceito sobre gamificação, que vem ganhando visibilidade por sua capacidade de criar experiências significativas quando aplicada em  contextos  da  vida  cotidiana, além de linhas gerais sobre sua aplicação em ambientes de aprendizagem. Embora esse trabalho diferentemente dos anteriores descritos aqui, não apresente a criação de um \SystemType, o mesmo será utilizado como referência para explicar os aspectos da gamificação que serão aplicados no ambiente proposto, assim como a motivação que os levaram a sua utilização.

A aplicação das técnicas da gamificação apresentadas por \citeonline{fardo2013gamificaccao} nesse trabalho, servirá para motivar os alunos durante a utilização da plataforma assim como uma forma de isentivo para os alunos que ajudarem os outros com dificuldades, através de premiação e ranqueamento.















\subsection{Considerações finais}

O sistema que se propôs desenvolver faz uso da metodologia de resolução de problemas, possibilitando o aluno adquirir novos conhecimentos além daqueles estudados no momento. Por se tratar de um AVA, o mesmo possibilitará ao aluno uma maior interação com o professor e outros alunos, além da possibilidade de uma aprendizagem auto-ritmada. A aplicação da gamificação no sistema, servira para desenvolver  engajamento, participação e comprometimento entre os usuários do sistema. 

\section{PROCEDIMENTOS METODOLÓGICOS}

Os procedimentos metodológicos definem as principais atividades realizadas para o desenvolvimento deste trabalho, incluindo as pesquisas, o desenvolvimento e avaliação do \textit{software}. Nas seções a seguir, descrevemos essas etapas.

\subsection{Definição do Processo}
Processos de \textit{software} são utilizados pelos engenheiros de \textit{software} para controlar e coordenar projetos de desenvolvimento de \textit{softwares} reais \cite{talma2006desenvolvimento}. \citeonline{padua2003engenharia} descreve um processo como um conjunto de passos parcialmente  ordenados, constituídos por atividades, métodos, práticas e transformações, usado para atingir uma meta. Desta forma, um modelo de processo de \textit{software} é uma descrição simplificada de um processo, sendo também uma representação abstrata do mesmo para explicar as diferentes abordagens de desenvolvimento \cite{sommerville2003engenharia}.

Processos de software são complexos e dependem do julgamento humano como em qualquer processo intelectual, sendo assim, não existe um processo de software ideal, todos são desenvolvidos de maneiras diferentes por cada organização \cite{sommerville2003engenharia}.

O processo de \textit{software} utilizado para desenvolver o sistema foi baseado no modelo cascata, também chamado de ciclo de vida clássico, proposto por Royce em 1970. Neste modelo, as fases são sistematicamente seguidas de maneira sequencial \cite{pressman2006engenharia}. O modelo inicia com a fase de especificação de requisitos, passando pelo planejamento, modelagem, construção e implantação, finalizando na manutenção progressiva do software, como apresentamos na \autoref{figura_ciclo_cascata}.

As vantagens desse modelo se devem ao fato de que só se avança para a tarefa seguinte quando o cliente valida e aceita os produtos finais da tarefa atual, facilitando assim a compreensão adquirida ao longo do projeto, além de facilitar o processo de criação da documentação para o sistema \cite{pressman2006engenharia}. Já as principais desvantagens segundo \citeonline{pressman2006engenharia}, se devem ao fato de que os projetos reais raramente seguem o fluxo sequencial ao qual o modelo propõe; o mesmo afirma ainda que este modelo exige que todos os requisitos sejam estabelecidos na fase inicial, fato que geralmente é difícil tanto para o cliente quanto para o desenvolvedor, já que os requisitos mudam constantemente. Outro grande problema com esse modelo é que o cliente só recebe uma versão executável do sistema no final de todo o processo de desenvolvimento, o que não agrada a muitos clientes.

Levando em consideração as vantagens e desvantagens antes citadas, esse modelo foi escolhido como base para o processo por facilitar o desenvolvimento de uma documentação mais detalhada e principalmente pela equipe de desenvolvimento ser formada por uma única pessoa, o autor deste trabalho, impossibilitando assim, uma divisão de tarefas durante o desenvolvimento, característico de metodologias ágeis \footnote{Metodologias de desenvolvimento de software que tem enfoco nas pessoas e não em processos ou algoritmos, além de uma preocupação menor em documentação e maior em implementação \cite{michel2004metodologias}.}.

\begin{figure}[H]
\centering
\includegraphics[width=10cm]{figuras/figura_ciclo_cascata}
\caption{Modelo Cascata (fonte: \citeonline{pressman2006engenharia}  )}
\label{figura_ciclo_cascata}
\end{figure}

O processo desenvolvido pode ser encontrado no \autoref{apendice_processo}.

\subsection{Levantamento e Análise de Requisitos}


O Levantamento de Requisitos é a fase do desenvolvimento de um software onde o analista verifica junto ao usuário, quais as necessidades, condições e princípios que o \textit{software} deverá atender \cite{matuda2013mapas}. Essa fase possibilitou conhecer e estudar as necessidades do cliente, assim como as restrições que o software estará sujeito.

Para realizar a coleta dos requisitos, optamos por utilizar entrevistas com o cliente. Nessas entrevistas, que se caracterizarão como semi-estruturadas, já que foram guiadas por um roteiro previamente elaborado, composto por questões abertas \cite{belei2008uso}, foi possível obter os requisitos do sistema, assim como o público-alvo a quem o sistema atenderá. Essa técnica foi utilizada porque permitia uma organização flexível e ampliação dos questionamentos à medida que as informações foram sendo fornecidas pelo cliente \cite{fujisawa2000utilizaccao}.

Para uma melhor compreensão do público-alvo, foram criadas Personas 
\cite{pruitt2003personas}, personagens fictícios usados para caracterizar os papéis dos diferentes usuários do sistema \cite{guerra2010colaboraccao}, cada persona criada possuía um nome, hábitos, histórias pessoais, motivações, objetivos, entre outras (ver \autoref{apendice_personas}). A escolha dessa técnica deu-se pelo fato de que ela permitia ao desenvolvedor saber mais precisamente para quem ele deveria construir o sistema, além de permitir uma distinção maior do público-alvo e dessa forma, aprofundar-se nos interesses individuais de cada um.

Após o levantamento dos requisitos, foi realizado uma análise dos mesmos. Nessa análise, os requisitos foram agrupados em categorias. As categorias utilizadas são descritas por \citeonline{sommerville2003engenharia} como:
\begin{alineascomponto}
    \item Requisitos Funcionais: especificam ações que um sistema deve ser
capaz de executar, sem levar em consideração restrições físicas. Os requisitos
funcionais especificam, portanto, o comportamento de entrada e saída de um
sistema.
    \item Requisitos Não Funcionais: descrevem apenas atributos do sistema ou
atributos do ambiente do sistema, como segurança, desempenho, usabilidade e
confiabilidade.
    \item Requisitos de Domínio: são os requisitos do domínio da aplicação do sistema e que refletem características desse domínio.
\end{alineascomponto}

Após esse agrupamento, os requisitos funcionais foram representados em Casos de Uso \cite{jacobson92engenharia}, um caso de uso identifica os agentes envolvidos em uma interação e especifica o tipo de interação, utilizando anotações sugeridas pela Unified Modeling Language (UML) 
\citeonline{sommerville2003engenharia}. Em seguida, foi realizada a documentação dos requisitos (ver \autoref{apendice_requisitos}).

Na etapa final dessa fase, ocorreu a Validação dos Requisitos junto ao cliente.  A Validação dos Requisitos é definida como o processo que certifica que o modelo de requisitos gerado  esteja  consistente  com  as  necessidades  e  intenções  de  clientes  e usuários \cite{rilston2003metodologia}. Esta etapa permitiu que os requisitos coletados e documentados estejam de acordo com o que o cliente solicitou.

\subsection{Projeto do Sistema}

O Projeto de Software é à atividade de engenharia cujo foco é definir ``como'' os requisitos estabelecidos do projeto devem ser implementados no software \cite{pressman2006engenharia}. O objetivo da  atividade de projetar é gerar um modelo ou representação que apresente solidez, comodidade e deleite \cite{pressman2006engenharia}. 

Nesta fase, definimos como será aplicado o conhecimento obtido na pesquisa bibliográfica para se desenvolver o sistema. Para isto, definimos a arquitetura de software e as ferramentas que serão utilizadas durante o desenvolvimento do sistema. Nas seções a seguir, descrevemos um pouco sob cada um.

\subsubsection{Arquitetura}
Por arquitetura de software, entende-se a estrutura ou a organização de componentes de módulos, a maneira através da qual esses componentes interagem e a estrutura de dados que será usada pelos componentes \cite{pressman2006engenharia}.

A arquitetura utilizada baseia-se na arquitetura Cliente-Servidor\cite{david2013everything}, onde o processamento é dividido em processos distintos. Um processo é responsável pela manutenção da informação (servidor) e os outros são responsáveis pela captação de dados (clientes). Nessa arquitetura, os clientes enviam pedidos para o servidor, e este por sua vez processa estes dados e envia as respostas dos pedidos aos clientes.

Este modelo de arquitetura facilitará na manutenção do sistema, tendo em vista que toda atualização só necessitará ser realizada no servidor e automaticamente a mesma se propagará para todos os clientes. Com todos os recursos centralizados no servidor, podemos também ter um maior ganho com segurança, já que podemos centralizar os nossos esforços para manter a segurança das informações em apenas um único ponto, além de possibilitar que apenas cliente credenciados possam acessar e/ou alterar essas informações. Uma das outras grandes vantagens que temos  ao utilizar esse modelo, é que a medida que a quantidade de clientes aumente, será possível suprir esses clientes sem necessitar realizar nenhuma modificação essencial.

Para uma visualização mais detalhada da arquitetura de software definida, visitar o  \autoref{apendice_arquitetura}.

\subsubsection{Ferramentas}

 A análise do sistema foi feita com o auxílio da ferramenta de criação de diagramas Astah \cite{astah2016}, a implementação com a linguagem Python \cite{vanrossum2010python}, com o sistema de gerenciamento de banco de dados PostgresSQl \cite{momjian2001postgresql} e a camada de aplicação através da utilização do framework Django \cite{django2016}. Assim como a utilização do módulo Rosseta \cite{rosetta2016} para permitir a internacionalização do sistema.
 
A seguir a lista das ferramentas e das tecnologias utilizadas para o desenvolvimento do projeto:

\begin{alineas}
	\item Astah: Para a modelagem baseada em UML (Unified Modeling Language) do sistema.
	\item Python: Linguagem de programação para implementação do sistema.
    \item Django: Framework web responsável pela camada de aplicação.
    \item Rosetta: Aplicação desenvolvida em Django que facilitará a tradução do projeto para diversas línguas.  
    \item PostgreSQL: Como banco de dados para armazenamento e consulta de informações.
    \item Metro UI Css: Framework que faz uso de HTML, Cascading Stype Sheet (CSS) e Javascript para criação do front-end do sistema.
    \item MathJax: É uma engine\footnote{Uma biblioteca ou pacote de funcionalidades que são utilizadas para facilitar o desenvolvimento de alguma tecnologia.} de código aberto desenvolvido em javascript na forma de um plugin para incluir equações matemáticas em todos os navegadores, esse plugin aceita expressões em  MathML e Latex.

\end{alineas}

Essas ferramentas foram selecionadas por se tratarem, algumas, de ferramentas Open Source, ou seja, que seu código-fonte fonte pode ser alterado para diferentes fins, possibilitando assim que qualquer um consulte, examine ou as modifique,e outras por serem ferramentas que possibilitam um rápido desenvolvimento.

No final dessa fase, foi gerado o Plano de Projeto (ver \autoref{apendice_projeto}), esse documento guiar\'a os desenvolvedores durante 
todo o processo de desenvolvimento.

\subsection{Implementação do Sistema}

A implementação envolve as atividades de codificação, compilação e integração. A codificação visa traduzir o design num programa, utilizando linguagens e  ferramentas adequadas. A codificação deve refletir a estrutura e o comportamento descrito no projeto. Os componentes arquiteturais devem ser codificados de forma independente e depois integrados \cite{aguiar2012requisitos}.

\subsection{Verificação e Validação}
Essa fase destina-se a mostrar que o sistema está de acordo com a especificação 
e que ele atende às expectativas de clientes e usuários. Al\'em de assegurar 
que o  programa está fazendo aquilo que foi definido na sua especificação e não 
possui  erros  de  execução \cite{aguiar2012requisitos}. 

Durante essa fase, ser\~ao realizados Testes de Unidade e Integra\c{c}\~ao em 
cada modulo do sistema, assim como Testes de Sistema no sistema como um todo.

\citeonline{aniche2014teste} define essas categorias de Testes de Software como:

\begin{alineascomponto}
	\item Teste de Unidade é aquele que testa uma única unidade do sistema. 
Ele a testa de maneira isolada, geralmente simulando as prováveis dependências 
que aquela unidade tem. Em sistemas orientados a objetos, é comum que a unidade 
seja uma classe. Ou seja, quando queremos escrever testes de unidade para a 
classe Pedido, essa bateria de testes testará o funcionamento da classe Pedido, 
isolada, sem interações com outras classes.

	\item  Teste de Integração é aquele que testa a integração entre duas 
partes do seu sistema. Os testes que você escreve para a sua classe PedidoDao, 
por exemplo, onde seu teste vai até o banco de dados, é um teste de integração. 
Afinal, você está testando a integração do seu sistema com o sistema externo, 
que é o banco de dados. Testes que garantem que suas classes comunicam-se bem 
com serviços web, escrevem arquivos texto, ou mesmo mandam mensagens via socket 
são considerados testes de integração.

	\item Teste de Sistema garante que o sistema funciona como um todo. Este 
nível de teste está interessado se o sistema funciona como um todo, com todas as 
unidades trabalhando juntas. Ele é comumente chamado de teste de caixa preta, já 
que o sistema é testado “com tudo ligado”: banco de dados, serviços web, batch 
jobs, e etc. 
\end{alineascomponto}

\subsection{Definição dos Conteúdo para o Sistema}

Após o sistema está verificado e validado, ele terá que possui conteúdos para ser utilizado pelo usuário final durante a fase de aplicação. 

Sendo que a aplicação da primeira versão do sistema está planejada para ocorrer com uma turma de matemática da Universidade Federal do Ceará - Campus Quixadá, decidimos optar por deixar os monitores\footnote{É o aluno de graduação concursado para exercer, juntamente com o professor, atividades técnico-didáticas condizentes com o seu grau de conhecimento junto à determinada disciplina, já por ele cursada.} da disciplina desenvolver o conteúdo que será utilizado no sistema durante essa fase. 

Os monitores passaram por um treinamento, onde aprenderam a utilizar o sistema para assim, adicionar os conteúdos desenvolvidos.

\subsection{Aplicação da Solução na Universidade Federal do Ceará - Campus Quixadá}

Essa fase tem por objetivo, permitir que sejam coletados dados para verificar o impacto da ferramenta com usuários reais. Para isso, aplicaremos o sistema em metade da turma de uma disciplina de Matemática Básica da UFC - Campus Quixadá, sendo que a outra metade não utilizará o sistema.

Nosso objetivo nessa fase, é verificar a evolução no aprendizado dos alunos dessa turma, sejam os que utilizaram o sistema como também os que não utilizaram. Para tentar verificar qualquer tipo de mudança que a ferramenta possa ter gerado nos alunos que a utilizaram.  

\subsection{Cronograma de Execução}
\begin{table}[H]
\centering
\caption{Cronograma de Execução}
\label{cronograma}

\resizebox{\textwidth}{!}{
\begin{tabular}{|l|c|c|c|c|c|c|c|c|c|c|c|c|c|c|c|}
\hline
\multicolumn{1}{|c|}{\multirow{2}{*}{ATIVIDADES}} & \multicolumn{6}{c|}{2015} & \multicolumn{9}{c|}{2016} \\ \cline{2-16} 
\multicolumn{1}{|c|}{} & Jan & Fev & Mar & Abr & Mai & Jun - Dez & \multicolumn{1}{l|}{Jan - Mar} & \multicolumn{1}{l|}{Abr} & \multicolumn{1}{l|}{Mai} & \multicolumn{1}{l|}{Jun} & \multicolumn{1}{l|}{Jul} & \multicolumn{1}{l|}{Ago} & \multicolumn{1}{l|}{Set} & \multicolumn{1}{l|}{Out} & \multicolumn{1}{l|}{Nov} \\ \hline
Definição do Processo & x &  &  &  &  &  &  &  &  &  &  &  &  &  &  \\ \hline
Levantamento e Análise dos Requisitos & x & x & x &  &  &  &  &  &  &  &  &  &  &  &  \\ \hline
Projeto do Sistema &  &  &  & x & x &  &  &  &  &  &  &  &  &  &  \\ \hline
Implementação do Sistema &  &  &  &  &  & x & x & x & x & x &  &  &  &  &  \\ \hline
Verificação e Validação &  &  &  &  &  &  &  &  &  &  & x &  &  &  &  \\ \hline
Desenvolver o conteúdo para o sistema &  &  &  &  &  &  &  & x & x & x & x & x &  &  &  \\ \hline
Aplicação na UFC - Campus Quixadá &  &  &  &  &  &  &  &  &  &  &  & x & x & x &  \\ \hline
Definição do Projeto de Pesquisa &  &  &  &  &  &  &  & x &  &  &  &  &  &  &  \\ \hline
Defesa do Projeto de Pesquisa &  &  &  &  &  &  &  &  &  & x &  &  &  &  &  \\ \hline
Ajustes Solicitados &  &  &  &  &  &  &  &  &  &  & x &  &  &  &  \\ \hline
Análise dos Resultados Obtidos na Aplicação. &  &  &  &  &  &  &  &  &  &  &  &  &  &  & x \\ \hline
Defesa do Trabalho Final &  &  &  &  &  &  &  &  &  &  &  &  &  &  & x \\ \hline
\end{tabular}}
\fnote{Fonte: Elaborado pelo autor}
\end{table}


\section{RESULTADOS PRELIMINARES}

Nesta seção apresentaremos o andamento deste trabalho até o presente momento.

Até o momento, de concluído, temos o processo que está sendo utilizado durante o desenvolvimento do sistema, os requisitos do sistema que já foram coletados, analisados e validados, além do próprio projeto do sistema. 

Na fase que está em andamento, que é a fase de implementação, já temos os seguintes módulos concluídos: 

\begin{alineascomponto}
	\item Gerenciador de Usuários: módulo responsável por gerenciar os usuários do sistema como professores, assistentes e alunos.
	\item Gerenciador de Turmas: módulo responsável por gerenciar as turmas de alunos do sistema.
    \item Gerenciador de Disciplinas: módulo responsável por gerenciar as disciplinas que serão cadastradas no sistema.
    \item Gerenciador de Lições: módulo responsável por gerenciar as lições que os professores irão cadastrar no sistema.
    \item Gerenciador de Questões: módulo responsável por gerenciar os problemas que os assistentes e professores poderão cadastrar para cada lição.
    \item Gerenciador de Pontuação: módulo responsável por gerenciar a pontuação ganha pelos alunos, assim como seu nível de experiência ao longo da utilização do sistema. 
    \item Fórum: módulo responsável por permitir que alunos postem dúvidas dos mais variados  assuntos relacionadas ao sistema, seja dúvidas em relação ao conteúdo apresentado em sala de aula, assim como informações sobre o sistema e sugestões. 
\end{alineascomponto}

Os módulos que ainda restam para serem desenvolvidos nesta fase são:

\begin{alineascomponto}
	\item Gerenciador do Progresso: módulo responsável por acompanhar o andamento de cada aluno durante seu aprendizado, para identificar obstáculos epistemológicos enfrentados pelos alunos para poder assim, alertar o professor, caso o aluno faça parte de alguma turma, ou sugerir que esse aluno peça ajuda ao sistema, para que o sistema envie esse pedido de ajuda a outros alunos que já concluíram aquele conteúdo com certo nível de proficiência.
	\item Chat: quando um aluno enfrentar um obstáculo epistemológico e o sistema sugerir que ele peça ajuda a outros alunos, esse módulo será responsável por mediar a comunicação entre esse aluno com dificuldade e o que se disponibilizar a prestar essa ajuda.
    \item Gerador de Estatísticas: módulo responsável por gerar as estatísticas que o professor utilizará para acompanhar o andamento de suas turmas e alunos, assim como para o uso pelo aluno, que utilizará para acompanhar seu próprio progresso durante sua aprendizagem no sistema.
    \item Ranking: módulo responsável por manter um ranking\footnote{É uma classificação ordenada de acordo com critérios determinados.} com os posicionamentos dos alunos de acordo com seu desempenho no sistema durante a semana.
    
\end{alineascomponto}

A seguir, apresentaremos algumas telas do sistema:

\begin{figure}[H]
  \centering
  \begin{minipage}[b]{0.49\textwidth}
	\caption{Tela Inicial}
    \includegraphics[width=\textwidth]{figuras/askmath/1}
    \\
    \\
  \end{minipage}
  \hfill
  \begin{minipage}[b]{0.49\textwidth}
	\caption{Tela Inicial do Aluno}
    \includegraphics[width=\textwidth]{figuras/askmath/2}
  	\\
    \\
  \end{minipage}
 
  \begin{minipage}[b]{0.49\textwidth}
    \caption{Tela de Administra\c{c}\~ao}
    \includegraphics[width=\textwidth]{figuras/askmath/3}
  \end{minipage}
  \hfill
  \begin{minipage}[b]{0.49\textwidth}
	\caption{Tela de Problemas do Aluno}
    \includegraphics[width=\textwidth]{figuras/askmath/4}
  \end{minipage}
\end{figure}

Os conteúdos que serão utilizados para popular o sistema, durante sua aplicação na UFC-Campus Quixadá, já estão sendo desenvolvidos pelos monitores citados anteriormente. Após a validação e verificação do sistema, esses conteúdos serão adicionados por esses monitores, que passaram por um treinamento para aprenderem a utilizar o sistema.









% ----------------------------------------------------------
% ELEMENTOS PÓS-TEXTUAIS
% ----------------------------------------------------------
%\postextual

% Referências bibliográficas

\renewcommand{\refname}{REFERÊNCIAS}
\bibliography{bibtex/referencias}
\addcontentsline{toc}{section}{REFERÊNCIAS}

% Glossário (Consulte o manual da classe abntex2 para orientações sobre o glossário)
%\glossary

% Apêndices
%\addtocontents{toc}{\protect\setcounter{tocdepth}{0}}
\addtocontents{toc}{\protect\setcounter{tocdepth}{-1}}
\begin{apendicesenv}
\addtocontents{toc}{\protect\setcounter{tocdepth}{0}}
\partapendices
\setcounter{secnumdepth}{1} % or 2 for numbered sections, or whatever
\setcounter{chapter}{0}

\addtocontents{toc}{\protect\setcounter{tocdepth}{1}}
\chapter*{APÊNDICE A - DOCUMENTO DE PROCESSO}\label{apendice_processo}
\addcontentsline{toc}{chapter}{APÊNDICE A - DOCUMENTO DE PROCESSO}
\addtocontents{toc}{\protect\setcounter{tocdepth}{-1}}

\begin{figure}[H]
\centering
\caption[caption]{Processo Desenvolvido}
\includegraphics[width=10cm]{figuras/figura_processo}
\label{figura_processo}
\fnote{Disponível em: \url{www.askmath.quixada.ufc.br/static/process/}}
\end{figure}
\input{secoes/apendices/apendice_b}
\newpage
\chapter{DOCUMENTO DE PERSONAS}\label{apendice_personas}

\section{Charlie - Técnico em Informática}

\begin{figure}[H]
\centering
\includegraphics[width=5cm]{figuras/personas/figura_persona_1}
\caption{Imagem de Charlie}
\label{figura_persona_1}
\end{figure}


Empresa: Ele trabalha na TechSolutions, uma empresa pequena, mas está ganhando
clientes e ampliando seus negócios.

Idade: 24 anos.

Genêro: Masculino.

Educação: Ensino técnico.

Mídias: Lê a revista Exame e usa ativamente o email para trocar informações com 
outros funcionários da empresa.

Objetivos: Como a empresa que Charlie trabalha está ampliando os negócios, ela
necessita que alguns de seus funcionários possuam um melhor conhecimento na 
área, por isso, Charlie decidiu investir num ensino superior, seu objetivo é 
concluir seu curso de Engenharia de Software e voltar a trabalhar normalmente 
para sua empresa.

Desafios: Charlie está com muita dificuldade na primeira disciplina de 
Matemática de seu curso, como ele já concluiu o ensino médio há bastante tempo, 
o mesmo não se lembra mais de como resolver questões simples de matemática como 
funções polinomiais, racionais e trigonométricas. Ele começou a estudar, mas 
não consegue encontrar uma forma intuitiva para verificar se está indo bem nos 
seus estudos, ficando preso aos exercícios que ele encontra nos livros.

Como minha empresa pode ajudá-la: O AskMath possibilitará que Charlie estude 
todos os conteúdos que ele tinha visto no ensino médio e já os tinha esquecido, 
são lições agrupadas por disciplina, cada lição possui um conjunto de questões 
e vídeo aulas criadas especialmente para pessoas com dificuldade nessas 
disciplinas. Para as questões, nosso sistema possibilitará que Charlie veja 
instantaneamente se sua resposta é correta ou não, ele também poderá pedir ajuda 
e saltar uma questão o mesmo não se sinta a vontade para respondê-la naquele 
momento. Com isso é possível que ele acompanhe através de estatísticas como 
está seu desempenho no sistema.

\section{Ruby - Bolsista de Graduação}

\begin{figure}[H]
\centering
\includegraphics[width=5cm]{figuras/personas/figura_persona_2}
\caption{Imagem de Ruby}
\label{figura_persona_2}
\end{figure}


Empresa: Ela trabalha como bolsista de monitoria das disciplinas envolvendo 
Matemática na Universidade Federal do Ceará.

Idade: 21 anos.

Genêro: Feminino.

Educação: Ensino superior.

Mídias: Usa ativamente o facebook, twitter e wattsapp.

Objetivos: O principal objetivo de Ruby é terminar sua graduação e tentar uma 
bolsa de mestrado numa universidade renomada.

Desafios: Ruby como monitora das disciplinas de Matemática, não consegue se
aproximar dos alunos para tirar suas dúvidas, segundo ela, "eles tem medo de 
dizer para nós monitores que está com dificuldade", então Ruby está em busca de 
novas formas de ajudar os estudantes que estão com dificuldades.

Como minha empresa pode ajudá-la: Com o AskMath, Ruby poderá ajudar esses 
alunos, ela poderá adicionar questões para os mesmos exercitarem 
seus conhecimentos.

\section{Samuel - Professor}

\begin{figure}[H]
\centering
\includegraphics[width=5cm]{figuras/personas/figura_persona_3}
\caption{Imagem de Samuel}
\label{figura_persona_3}
\end{figure}

Empresa: Ele trabalho como professor da disciplina de Matemática Básica na
Universidade Federal do Ceará.

Idade: 59 anos.

Genêro: Masculino.

Educação: Doutorado.

Mídias: Lê o jornal The New York Times e usa email para tirar dúvidas dos alunos
Objetivos: Samuel é um professor exemplar, se preocupa muito em ensinar seus 
alunos, seu principal objetivo é ensinar da melhor forma possível seus alunos, 
de forma que todos possam seguir excelentes carreiras quando se formar e se 
tornam ótimos profissionais.

Desafios: Samuel gosta de acompanhar o andamento dos estudos de seus alunos,
entretanto, quando ele os questiona sobre alguma dificuldade que eles estejam
enfrentando, os mesmo dizem que está tudo bem, o que não reflete em suas notas.
Sabendo disso, Samuel fica triste, já que não consegue saber como anda o nível 
de aprendizado de sua turma e gostaria de ajudá-los mais.

Como minha empresa pode ajudá-la: Com o AskMath, Samuel pode acompanhar o
andamento de sua turma, e saber em quais conteúdos eles estão com mais 
dificuldade, para ele poder focar mais neles. Samuel também poderá adicionar 
ele mesmo os conteúdo que ele achar mais interessantes para seus alunos, além 
disso, ele verá as dúvidas que eles postam no fórum e poderá ele mesmo 
responder a essas dúvidas.

\section{Stefane - Estudante do Ensino Médio}

\begin{figure}[H]
\centering
\includegraphics[width=5cm]{figuras/personas/figura_persona_4}
\caption{Imagem de Stefane}
\label{figura_persona_4}
\end{figure}

Empresa: Ela trabalha no salão de beleza de sua mãe.

Idade: 18 anos.

Genêro: Feminino.

Educação: Ensino médio.

Mídias: Usa ativamente o facebook e WatssApp.

Objetivos: Stefane busca concluir o ensino médio e tirar uma boa nota no ENEM 
para tentar conseguir uma vaga para cursar Sistemas de Informação na UFC, que é 
o curso de seus sonhos.

Desafios: Stefane, como estudante, não gosta de estudar matemática por livros, 
ela acha incomodo ter que levar seus livros enormes de matemática em sua bolsa 
para o salão de beleza de sua mãe, mas ela quer ficar estudando enquanto não 
aparece clientes no salão. Um outro problema de Stefane, é que às vezes, ela 
fica com dúvida na resolução de alguma questão e precisa ligar para seus amigos 
em busca de explicações que nem sempre encontra. Stefane também sente uma certa 
dificuldade em aprender através de papéis, ela acha que nada é melhor do que a 
explicação visual de uma pessoal.

Como minha empresa pode ajudá-la: Com o AskMath, Stefane não precisará mais
levar livros de Matemática em sua bolsa, para aprender matemática ela precisa 
apenas de uma celular, isso possibilita que ela possa continuar estudando no 
salão quando não tiver muitos clientes. Para solucionar as dúvidas de Stefane, 
temos pessoas qualificadas esperando para ajudá-la, ela poderá publicar suas 
dúvidas num fórum onde essas pessoas podem responder, assim como outros 
pessoas, assim como Stefane, mas que já tenha tido a mesma dúvida. Como Stefane 
não gosta de papéis, o AskMath possui um conjunto de vídeo aulas criadas por 
pessoas capacitadas, que sabem a melhor forma de ensinar através de recursos 
visuais.
\addtocontents{toc}{\protect\setcounter{tocdepth}{1}}
\chapter*{APÊNDICE D - DOCUMENTO DE ARQUITETURA}\label{apendice_arquitetura}
\addcontentsline{toc}{chapter}{APÊNDICE D - DOCUMENTO DE ARQUITETURA}
\addtocontents{toc}{\protect\setcounter{tocdepth}{-1}}


\section{Introdução}
Neste documento será detalhado a arquitetura do sistema proposto para o projeto 
AskMath. Como o projeto é voltado para a web, o sistema é formado por diversos 
padrões de projetos de mercado, principalmente, padrões orientados a objetos. 
Vamos destacar cada parte da arquitetura escolhida para a fácil compreensão de 
outras equipes que venham a trabalhar no projeto, seguindo a risca seu modelo 
para permitir que o sistema seja facilmente modificado ou incrementado sem que a 
estrutura do software seja perdida.

\section{Objetivos}
\begin{alineascomponto}
	\item Prover uma visão geral da arquitetura do projeto, detalhando cada 
parte da estrutura do sistema para permitir a compreensão do mesmo.
    \item Permitir que este documento seja utilizado por outras equipes de 
desenvolvimento que estão dando continuidade ao projeto, quanto a novos 
integrantes inseridos na equipe.
    \item Permitir que este documento sirva como meio de comunicação entre o 
Arquiteto de Sistema e a equipe desenvolvedores.
    \item Apresentar aos \textit{stakeholders} uma visão de alto nível de como o 
sistema é implementado e como esta estruturado.
    \item Aumentar o desempenho e a robustez, bem como a capacidade de 
distribuição e manutenibilidade do sistema.
\end{alineascomponto}

\section{Considerações Gerais}
As definições de arquitetura do processo de desenvolvimento do software esta 
preocupado como o sistema deve ser organizado e com a estrutura geral do 
sistema, estas definições tem que atender as especificações do projeto, desde as 
especificações de segurança, regras de negócio, até a parte de persistência de 
banco de dados. As decisões de projeto de arquitetura têm efeitos profundos 
sobre a possibilidade de o sistema atender ou não aos requisitos críticos, como 
desempenho, confiabilidade e manutenibilidade.

As definições da arquitetura atendem as especificações de projeto documentadas 
até o presente momento, apresentando um modelo completo do sistema que mostre 
seus diferentes componentes, suas interfaces e conexões.

\section{Responsabilidades}
Toda a equipe de desenvolvimento é responsável por elaborar este documento  e 
por manter a integridade do mesmo durante o processo de desenvolvimento. Cada 
membro deve:
\begin{alineascomponto}
	\item Analisar todas as mudanças arquiteturais significativas e 
documentá-las;
    \item Ao verificar uma possível alteração na arquitetura, convocar uma 
reunião com toda e equipe para discutir a possível alteração. 
\end{alineascomponto}

\section{Arquitetura}
A arquitetura foi desenvolvida para ser de baixo acoplamento e que ao mesmo 
tempo fosse independente de tecnologias existentes do mercado, sendo assim, 
poderíamos por exemplo futuramente trocar o framework Django pelo o Spring, e o 
mesmo seria transparente para os programadores e para o sistema .

\subsection{Elementos que compõe a Arquitetura}
A arquitetura é composta por elementos, em que em conjunto produzem o produto 
final. Esses elementos são: Database, Models, Views e Templates.

Nos tópicos a seguir descrevemos cada um dos componentes e o seu papel  dentro 
da arquitetura como um todo, além de discutirmos a tecnologia e os padrões 
adotados para a implementação dos mesmos.

\subsubsection{Database}
No desenvolvimento de sistemas, necessitamos muitas vezes salvar as informações 
geradas para eventuais utilizações das mesmas, para isso, temos disponíveis os 
banco de dados. Em geral, as linguagens de programação nos proprocionam formas 
de acessar esses dados, porém de forma muito  complexa que acaba ocasionando um 
alto acoplamento para termos de alta coesão. O framework que utilizaremos, nós 
propociona uma camada apenas de banco de dados ou camada de persistência.

Nessa camada estão as entidades do sistema, e as mesmas implementam 
funcionalidades de conexão e outros controles que são iguais a todos os tipos de 
acessos a bancos de dados. 

Com esse tipo de acesso aos dados, conseguimos trocar facilmente de banco de 
dados sem afetar a implementação do sistema. Por exemplo, caso estejamos 
utilizando um sgbd relacional como o postgresql para armazenar os dados e 
necessitarmos migrar para um sgbd não “relacional”, os famosos banco de dados 
NoSQL, como o MongoDB, não será necessário alterar nenhum aspecto da 
implementação,  pois as regras de negócio vão está totalmente separadas das 
funcionalidades de acesso ao banco de dados.

\subsubsection{Models}
São as entidades do sistema. Entidades são abstrações do mundo real modelados em 
forma de tabelas que guardarão informações no banco de dados. Nessa arquitetura, 
os modelos contemplarão também as associações entre as entidades, sendo que cada 
entidade também terá alem de seus atributos, um conjunto de métodos para se 
relacionar com o sistema.

Os modelos implementarão um ORM(Object-Relational Mapping) para as views 
acessarem o banco de dados, tornando o sistema independente do acesso ao banco. 

\subsubsection{Views}
Todo sistema é formado por um conjunto de regras de negócio. Uma regra de 
negocio é fluxo lógico que deve ser processado para que seja gerado um resultado 
válido.

Uma View é responsável pela execução de um ou mais fluxos de execução que são 
modelados em um caso de uso, ou seja, uma visão é uma implementação da regra de 
negocio, elas são basicamente funções que aceitam como primeiro parâmetro uma 
\textit{request} que representa uma requisição Web (além de outros parâmetros) 
vinda do usuário pelo \textit{browser}, ela irá tratar essa requisição e 
retornar uma resposta ao usuário.

Dessa forma, cada caso de uso acaba tornando se uma view, sendo que views podem 
dispor de outras views para realizarem suas funções. 

\subsection{Desenho geral da arquitetura}
\begin{figure}[H]
\centering
\includegraphics[width=15cm]{figuras/figura_arquitetura.png}
\caption{Desenho Geral da Arquitetura}
\label{figura_arquitetura}
\end{figure}

Nessa arquitetura, o browser do cliente utiliza de uma URL para realizar uma 
requisição numa view, essa view pode utilizar os models para extrair dados do 
banco de dados e ela própria criar um template ela mesma retornar como uma forma 
de resposta ao usuário, ou então ela pode utilizar um template pronto apenas 
moldar esses dados dentro de um template pronto e retornar isso ao browser do 
cliente. 

\section{Padrões de projeto}
Um padrão de projeto representa o trabalho de uma pessoa que encontrou o mesmo 
problema, tentou muitas soluções possíveis, selecionou e descreveu uma das 
melhores e você deve se aproveitar disso.

O conhecimento de padrões permite decidir o que deve ser feito e o que deve ser 
evitado, sistemas baseados em padrões têm mais qualidade. Para o AskMath, forma 
analisados alguns padrões  de projeto e selecionados aqueles que poderiam ser 
satisfatoriamente aplicados. 

\subsection{Proxy}
Fornecer um substituto ou marcador da localização de outro objeto para controlar 
o acesso a esse objeto, podemos controlar o acesso aos métodos de uma entidade 
da seguinte forma: cria se uma classe EntidadeProxy que extende de Entidade, e 
reimplementa cada um dos métodos, nessa nova implementação, você solicitar que o 
usuário informe uma autenticação que possua acesso, ao informar, esse método ira 
apenas chamar o método da Entidade.   

\subsection{Chain of Responsibility}
Evitar o acoplamento do remetente de uma solicitação ao seu receptor, ao dar a 
mais de um objeto a oportunidade de tratar a solicitação. Encadear os objetos 
receptores, passando a solicitação ao longo da cadeia até que um objeto a trate.

\section{Objetivos e Restrições Arquiteturas}

Apresentaremos aqui os requisitos e objetivos do software que têm algum impacto 
na arquitetura, tais como: segurança, proteção de dados, privacidade, 
portabilidade, distribuição, reuso. Também são descritos nesta seção restrições 
arquiteturais que se aplicam ao projeto, tais  como:  estratégias  de  modelagem 
 e  implementação,  ferramentas  de  desenvolvimento, sistemas legados. 

\subsection{Requisitos básicos}
\begin{alineascomponto}
	\item Ubuntu Server como sistema operacional de produção.
	\item Utilização apenas de componentes opensource.
	\item PostgresSQL como sistema de gerenciamento de banco de dados.
	\item Django como framework de desenvolvimento.	
	\item O sistema devera ser Web.
\end{alineascomponto}

\subsection{Estratégias de implementação}
\begin{alineascomponto}
	\item Persistência de tipagem (ex: formato de CPF, Data) devem ser feitos 
tanto no cliente através de Javascript, como no servidor através de 
regex(Regular Expression).
	\item Persistência de obrigatoriedade, verificar se todos os campos 
obrigatórios foram  preenchidos deve ser feito tanto no cliente via javascript 
como no servidor e nas duas opções deve-se apresentar alertas caso algum campo 
seja enviado vazio para a requisição.
    
\end{alineascomponto}

\newpage
\chapter{DOCUMENTO DE PROJETO}\label{apendice_projeto}

\section{Introdução}
Este documento é o plano de projeto para o AskMath, ele será usado para 
gerenciar a execução projeto. Nele, está contido os planos de gerenciamento de 
pessoal, riscos e o cronograma geral que será usado para executar cada atividade 
necessária para desenvolver o sistema.

\section{Proposito}
Esse documento foi desenvolvido para facilitar o acompanhamento do projeto 
pelos interessados, ele será utilizado durante o projeto, as revisões, as 
verificações 
até a entrega e aceitação final do produto pelo cliente, conforme acontecer o 
avanço no ciclo de vida do projeto, esse documento poderá ser atualizado para 
comportar as mudanças que venha ocorrer.

\section{Público Alvo}
Esse projeto é voltado para o Cliente que solicitou o produto como também para 
toda a equipe de desenvolvimento, para assim, facilitar o entendimento do 
processo em que o produto será desenvolvido.

\section{Evolução do Plano de Projeto}
Esse plano de projeto estará em constante mudança, a medida que o projeto for 
sendo utilizado pela equipe, novas e melhores formas de se gerenciar poderão vir 
a aparecer, acarretando assim numa atualização do projeto atual.

\section{Visão geral do Projeto}
\subsection{Necessidades do Cliente}

\begin{table}[H]
\centering
\caption{Nescessidades do Cliente}
\label{table_nescessidades_cliente}
\resizebox{\textwidth}{!}{
\begin{tabular}{|l|l|}
\hline
\multicolumn{2}{|c|}{Nescessidades do Cliente} \\ \hline
NEC01 & O Cliente necessita fornecer aos seus alunos uma ferramenta onde eles 
possam utilizar para quantificar seu nível de conhecimento. \\ \hline
NEC02 & O Cliente necessita que a ferramenta gere relatórios que mostrem o 
andamento dos alunos na ferramenta. \\ \hline
NEC03 & Na ferramenta, deve existir um fórum anônimo onde os alunos possam expor 
suas duvidas para que outros alunos posam lhes ajudar. \\ \hline
\end{tabular}}
\end{table}

\subsection{Escopo do Projeto}
O Sistema AskMath proposto pelo PETTI, tem por objetivo auxiliar no aprendizado 
dos alunos da disciplina de Matemática Básica, dispondo de um ambiente de estudo 
em que o aluno possa aprender de forma simples e intuitiva, além de compartilhar 
suas duvidas com outros alunos e professores para obter ajuda o mais rápido 
possível. Do sistema, o professor também deve obter relatórios do andamentos dos 
alunos para poder assim, adaptar suas metodologias de ensino no decorrer da 
disciplinas.

\section{Gerenciamento de Riscos}
Nessa seção, apresentaremos as formas como iremos abordar, planejar e executar 
as atividades para esse projeto.

\subsection{Metodologia}

A metodologia adotada será um processo iterativo que continua ao longo do 
desenvolvimento do projeto. Um plano de gerenciamento de ricos inicial será 
elaborado, onde este vai ser monitorado para detectar a possibilidade de riscos 
emergentes. A obtenção de mais informações de riscos é realizada durante o 
processo de desenvolvimento do plano de projeto e do projeto, caso as 
informações estejam disponíveis, os riscos serão reavaliados e será verificado 
se sua prioridade de risco mudou.

O Diagrama do processo de desenvolvimento de risco pode ser encontrado na 
\autoref{figura_processo_riscos}

\begin{figure}[H]
\centering
\includegraphics[width=10cm]{figuras/figura_processo_riscos}
\caption{Processo utilizado para gerenciar os riscos}
\label{figura_processo_riscos}
\end{figure}

\subsubsection{Identificação de Riscos}
A Identificação de Riscos diz respeito à identificação de riscos que podem 
representar uma ameaça para o processo de desenvolvimento de software, o 
software que esta sendo desenvolvido ou a organização de desenvolvimento. Um 
checklist de tipos diferentes de ricos será utilizado para uma identificação 
inicial dos ricos. Este estágio do processo gera uma lista de potenciais ricos, 
estes ricos podem ser encontrados na \autoref{table_riscos_encontrados}.

\begin{table}[H]
\centering
\caption{Riscos Encontrados}
\label{table_riscos_encontrados}

\resizebox{\textwidth}{!}{
\begin{tabular}{|l|l|}
\hline
\multicolumn{1}{|c|}{\textbf{Tipo de riscos}} & 
\multicolumn{1}{c|}{\textbf{Possíveis Riscos}} \\ \hline
Tecnologia & \begin{tabular}[c]{@{}l@{}}Atrasos de especificação (1)\\ Mudança 
de tecnologia (2)\\ Componentes reusáveis não atendem as especificações (3)\\ 
Problema para Integrar os Modulos (4)\end{tabular} \\ \hline
Pessoas & Pessoas estão doentes (5) \\ \hline
Organizacional & O produto fere principios da intituição (6) \\ \hline
Ferramentas & \begin{tabular}[c]{@{}l@{}}Ferramentas são difíceis de serem 
utilizadas (7)\\ Dificuldade na instalação das ferramentas (8)\end{tabular} \\ 
\hline
Requisitos & \begin{tabular}[c]{@{}l@{}}Mudança de requisitos que requerem 
retrabalho do projeto(9)\\ O Cliente não possui tempo livre para se reunir 
pessoalmente com a equipe (10)\end{tabular} \\ \hline
Estimativa & \begin{tabular}[c]{@{}l@{}}Tamanho do software subestimado (11)\\ 
Atrasos no Cronograma (12)\\ A taxa de reparo de defeito é substimado 
(13)\end{tabular} \\ \hline
\end{tabular}}
\end{table}


\subsubsection{Análise de Riscos}
Nesta fase do processo cada risco identificado na etapa 
anterior é analisado e classificado sobre a probabilidade de riscos e seu 
impacto no projeto caso ocorra (As definições de probabilidade e impacto de 
risco se encontra em dos tópicos seguintes). Uma vez analisados e classificados 
os riscos mais significativos serão identificados e estes serão riscos que vão 
ser monitorados durando o projeto.

Este estágio do processo gera uma lista de prioridade de riscos, esta lista se 
encontra na \autoref{table_lista_prioridade_riscos}.

\begin{table}[H]
\centering
\caption{Lista de Prioridade dos Riscos}
\label{table_lista_prioridade_riscos}
\resizebox{\textwidth}{!}{
\begin{tabular}{|l|l|l|l|}
\hline
\multicolumn{1}{|c|}{\textbf{Risco}} & 
\multicolumn{1}{c|}{\textbf{Probabilidade}} & 
\multicolumn{1}{c|}{\textbf{Efeito}} & \multicolumn{1}{c|}{\textbf{Afeta}} \\ 
\hline
Atrasos de especificação (1) & Quase Certo & Médio & Projeto e Produto \\ \hline
Mudança de tecnologia (2) & Pouco Provável & Muito Alto & Projeto e Produto \\ 
\hline
Componentes reusáveis não atendem as especificações (3) & Pouco Provável & Muito 
Alto & Projeto \\ \hline
Problema para Integrar os Módulos (4) & Pouco Provável & Alto & Produto \\ 
\hline
Pessoas estão doentes (5) & Raro & Baixo & Organização \\ \hline
O produto fere princípios da instituição (6) & Improvável & Alto & Produto \\ 
\hline
Ferramentas são difíceis de serem utilizadas (7) & Pouco Provável & Médio & 
Projeto \\ \hline
Dificuldade na instalação das ferramentas (8) & Pouco Provável & Médio & Projeto 
\\ \hline
Mudança de requisitos que requerem retrabalho do projeto(9) & Pouco Provável & 
Muito Alto & Projeto e Produto \\ \hline
O Cliente não possui tempo livre para se reunir pessoalmente com a equipe (10) & 
Muito Provável & Médio & Projeto \\ \hline
Tamanho do software subestimado (11) & Quase Certo & Médio & Projeto \\ \hline
Atrasos no Cronograma (12) & Quase Certo & Baixo & Projeto \\ \hline
A taxa de reparo de defeito é subestimado (13) & Quase Certo & Baixo & Projeto e 
Produto \\ \hline
\end{tabular}}
\end{table}

\subsubsection{Planejamento dos Riscos}

O Planejamento dos riscos, considera cada um dos riscos que foram identificados 
e desenvolve estratégias para gerenciar estes ricos. Este estágio do processo 
gera uma lista de prevenção de riscos e planos de contingência, esta lista pode 
ser encontrada na \autoref{table_lista_prevencao_contigencia}.

\begin{table}[H]
\centering
\caption{Lista Prevenção de Riscos e Planos de Contingência}
\label{table_lista_prevencao_contigencia}
\resizebox{\textwidth}{!}{
\begin{tabular}{|l|l|}
\hline
\multicolumn{1}{|c|}{\textbf{Risco}} & \multicolumn{1}{c|}{\textbf{Estratégia}} 
\\ \hline
Atrasos de especificação (1) & Componentes e requisitos que são pré-requisitos 
de outros possuem seus desenvolvimentos adiantados. \\ \hline

Mudança de tecnologia (2) & As tecnologias a serem utilizadas são analizadas e 
avaliadas antes de serem usadas para evitar possíveis mudanças. Uma tecnologia 
reserva deve ser escolhida em caso de eventual mudança não prevista. \\ \hline

Problema para Integrar os Módulos (4) & Os módulos serão desenvolvidos pensando 
em como integra-los futuramente, usando algum modelo de desenvolvimento, como 
MVC ou Camadas. Em caso de módulos externos bibliotecas que fazem a comunicação 
terão de ser criadas (aumento de custo e possível atraso no cronograma). \\ 
\hline

O produto fere princípios da instituição (6) & Estudar as normas da instituição 
e desenvolver o software dentro delas. \\ \hline

Ferramentas são difíceis de serem utilizadas (7) & Estar sempre se comunicando 
com o usuário em cada fase de desenvolvimento para o recebimento de feedbacks, 
evitando mudanças drásticas em etapas finais do projeto. \\ \hline

Mudança de requisitos que requerem retrabalho do projeto(9) & Estar sempre se 
comunicando com o usuário em cada fase de desenvolvimento para o recebimento de 
feedbacks, evitando mudanças drásticas em etapas finais do projeto. \\ \hline

A taxa de reparo de defeito é subestimado (13) & Avaliar o defeito antes de 
estimular seu reparo. \\ \hline
\end{tabular}}
\end{table}

\subsubsection{Monitoração de riscos}

O Monitoração de riscos verifica se as suposições definidas sobre os riscos 
não foram alteradas durante o processo de desenvolvimento do projeto. Para isso 
são avaliados os outros fatores que dão a probabilidade de riscos e seus 
efeitos. Este estágio de processo gera a tabela de avaliação de riscos.
\end{apendicesenv}

% Anexos
% \input{editaveis/anexos}

%---------------------------------------------------------------------
% INDICE REMISSIVO
%---------------------------------------------------------------------
%\phantompart
%\printindex
%---------------------------------------------------------------------

\end{document}