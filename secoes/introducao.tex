\section{INTRODUÇÃO}

A Matemática, como ciência, sempre teve uma relação muito especial com as novas tecnologias, desde as calculadoras, os computadores, aos sistemas multimédia e a Internet. No entanto, os professores costumam demorar a perceber como tirar partido destas tecnologias como ferramentas de trabalho. \cite{da1997ensino}. 

A medida que a quantidade de recursos tecnológicos na sala de aula foram aumentando, tornou-se necessário a criação de novas metodologias de ensino, especificamente na Educação Matemática. Tal busca procura fazer da Matemática uma disciplina atraente, desvinculada do ensino tradicional que já se mostrou ineficiente \cite{silva2009ambiente}.

Tendo em vista essa necessidade, o presente trabalho, busca apresentar o projeto e desenvolvimento de um Ambiente Virtual de Aprendizagem (AVA) \cite{valentini2010aprendizagem}, para auxiliar estudantes no ensino e aprendizagem de conteúdos matemáticos. Este ambiente se propõe a servir como ferramenta para estudantes que buscam estudar fora do ambiente escolar e de uma forma auto-ritmada, ou seja, no seu próprio ritmo. Entretanto, o ambiente também dará suporte ao ensino e aprendizagem dentro da sala de aula, auxiliando professores com informações relevantes sobre o andamento do aprendizado de cada um de seus aluno, além das dificuldades que os mesmos apresentam.

Grande parte dos alunos tem dificuldades em aprender matemática, e muitas vezes essas dificuldades ocorrem não pela falta de atenção ou por não gostar do conteúdo, mas por fatores mentais ou psicológicos que envolvem uma série de trabalhos
e conceitos que precisam ser desenvolvidos \cite{sa2015software}. Mas como auxiliar alunos com dificuldades na aprendizagem da matemática?

Em busca dessa resposta, diferentes sistemas de \textit{softwares} foram desenvolvidos buscando servir a educação. Em 2006, Salman Khan funda a Khan Academy, uma organização educacional que tem por objetivo oferecer exercícios, vídeos de instrução e um painel de aprendizado personalizado que habilita os estudantes a aprender no seu próprio ritmo dentro e fora da sala de aula \cite{khan2012one}. A plataforma criado por khan utiliza de vídeo-aulas e resolução de problemas para ensinar seus alunos, permitindo assim, que cada um tenha uma aprendizagem de forma independente e auto-ritmada.

Um outro trabalho também  importante nessa área, podemos citar o de  \citeonline{melis2001activemath}, onde os autores desenvolvem um AVA que permite os alunos desfrutarem da experiência de estudar num curso gerado dinamicamente a partir do estado em que o mesmo se encontra dentro do sistema. Estes e outros trabalhos são apresentados com mais detalhes em trabalhos relacionados.

\subsection{Objetivos do Trabalho}

Este trabalho visa contribuir com o processo de ensino e aprendizagem de matemática, buscando projetar e desenvolver um ambiente virtual que auxilie no processo de ensino e aprendizagem de matemática por alunos, dentro e fora da sala de aula. 

Como objetivos específicos para este trabalho, temos: 
\begin{alineascomponto}
    \item Elaborar um método para diagnosticar, nos alunos, dificuldades existentes em certos conteúdos de matemática.
	\item Desenvolver o AVA, aplicando técnicas de gamificação.
    \item Aplicar o AVA numa turma de graduação onde os alunos estejam cursando disciplinas de matemática.
    \item Analisar a influência e impactos provocados pelo AVA, por meio de dados gerados pela ferramenta ao longo de sua utilização e questionários aplicados aos alunos que utilizarem o AVA.

\end{alineascomponto}


\subsection{Divisão do Trabalho}

Fundamentando-se na problemática mencionada, e tendo em vista o objeto de estudo, esta dissertação foi dividida em cinco capítulos. No capítulo inicial é feita uma apresentação do tema a ser discutido durante toda esta dissertação e o objetivo do trabalho.

No Capítulo 2, destacam-se os aspectos teóricos sobre ensino e aprendizagem, assim como as tradicionais metodologias de ensino e as apoiadas por computador, além de conceitos sobre gamificação e os trabalhos que serviram de referencial para abordar os conceitos e ideias utilizadas no trabalho aqui desenvolvido.

No Capítulo 3 compreende-se a concepção, construção e modelagem do sistema, apresentando o que o mesmo deve possuir e por que.

No Capítulo 4 são apresentados os resultados preliminares.