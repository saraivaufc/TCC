\section{REVISÃO BIBLIOGRÁFICA}

Este capítulo aborda os seguintes temas: metodologias no ensino da matemática, conceito de ambientes virtuais de aprendizagem, o uso da gamificação aplicada em AVAs, e, por fim, o que está sendo 
desenvolvido por outros pesquisadores da área.

\subsection{Metodologias no Ensino da Matemática}

Ao longo da história, várias metodologias e abordagens matemáticas foram utilizadas visando a melhoria do ensino. Segundo \citeonline{hammes2003tendencias}, algumas delas foram aula expositiva, 
resolução de problemas, modelagem matemática, e o uso de computadores. Nas seções a seguir, descreveremos cada uma delas.

\subsubsection{Aula expositiva}

Em uma aula expositiva, o professor comumente faz uma revisão da aula anterior, apresenta o novo conteúdo e passa aos alunos uma série de exercícios de fixação. Esse novo conteúdo é apresentado de 
forma oral ou escrita, sem levar em consideração o conhecimento prévio dos estudantes nem tempo para perguntas. Essa é, sem dúvida, uma das mais utilizadas e antigas metodologias existentes. Durante 
o século passado, aulas expositivas foram o único processo empregado em sala de aula pelos professores. Dessa forma, a aula expositiva pode ser considerada cansativa e desinteressante, já que o aluno 
não participa do processo de ensino  \cite{hammes2003tendencias}.

Para \citeonline{de1996gerencia}, essa abordagem possui diversos problemas. \citeonline{de1996gerencia} sugere que a escola tradicional não somente está desatualizada para atender às necessidades 
crescentes da sociedade contemporânea, como também apresenta algumas características que inibem o desenvolvimento do potencial de criação dos alunos:
[Uniformizar esta lista de pontos, n\~ao sei o que significa :(]
\begin{alineascomponto}
\item Destaca-se a incompetência, a ignorância e a incapacidade do aluno, deixando de assinalar os talentos e habilidades de cada um; 
\item O ensino voltado para o passado, onde se enfatiza a reprodução e a memorização do conhecimento;
\item Desconsidera-se a imaginação e a fantasia como dimensões importantes da mente;
\item Exercício de resposta única, onde se cultua o medo do erro e do fracasso;
\item A obediência, dependência, passividade e conformismo são os traços mais cultivados;
\item Descaso em cultivar uma visão otimista do futuro;
\item As habilidades cognitivas são desenvolvidas de forma limitada;
\end{alineascomponto}

[REVER ISSO]
Autores como \citeonline{lopes1995aula}, defendem que a aula expositiva ``poderá ser transformada em uma atividade dinâmica, participativa e estimuladora do pensamento critico do aluno''. 
Mas os mesmos afirmam que isso é uma tarefa difícil, já que, em geral, é mal empregado pelos professores. Esse autor complementa ainda que essa nova abordagem ``valoriza a vivência dos alunos, seu 
conhecimento do concreto, e busca relacionar esses conhecimentos prévios com o assunto a ser estudado''. Ainda de acordo com \citeonline{lopes1995aula}, o professor

\begin{citacao}
``[...] jamais desconsidera uma pergunta em aula, mesmo que ela possa lhe parecer ingênua ou despropositada. Ao perceber uma pergunta mal formulada o papel do professor é ajudar o aluno a refazer a pergunta pois essa atitude educa o aluno a aprender a perguntar''.
\end{citacao}

\subsubsection{Resolução de problemas}

A resolução de problemas deve ser entendida como uma oportunidade para o aluno obter novos conhecimentos e não apenas conhecimentos prontos que fazem parte da nossa história. Ela ajuda, o aluno a 
desenvolver sua autonomia buscando as respostas para seus próprios questionamentos. \citeonline{fossa1998tendencias} ressaltam  que a resolução de problemas 
[Adicionar P\'agina]
\begin{citacao}
``[...] visa o desenvolvimento de habilidades metacognitivas, favorecendo a todo momento a reflexão e o questionamento. O aluno aprende a pensar por si mesmo, levantando hipóteses, testando-as, 
tirando conclusões e até discutindo-as com os colegas.'' \cite{fossa1998tendencias}. 
\end{citacao}

Nessa metodologia, é importante que os problemas apresentem uma incógnita que necessite ser descoberta. Para resolvê-los, o aluno terá que inventar estrat\'egias e gerar novas ideias. Segundo 
\citeonline{dante1991didatica}
[Adicionar P\'agina]
\begin{citacao}
``É importante que o problema possa gerar muitos processos de pensamento, levantar muitas hipóteses e propiciar várias estratégias de solução. O pensar e o fazer criativo devem ser componentes fundamentais no processo de resolução de problemas.'' \cite{dante1991didatica}.
\end{citacao}

\subsubsection{Modelagem Matemática}
[REVER TODA ESSE SE\c{C}\^AO]
Modelagem Matemática diz respeito ao processo de criação de um modelo matemático. \apudonline{granjer1997modelagem}{biembengut1999modelagem} define um modelo matemático como sendo...
\begin{citacao}
``[...] uma imagem que se forma na mente, no momento em que o espírito racional busca compreender e expressar de forma intuitiva uma sensação, procurando relacionar com algo já conhecido, efetuando deduções.''\apud[p.~78]{granjer1997modelagem}{biembengut1999modelagem}
\end{citacao}

A modelagem matemática é uma metodologia que busca proporcionar ao aluno uma visão prática do conhecimento teórico aprendido na sala de aula, através de problemas de ordem prática. Para \citeonline{fossa1998tendencias}
\begin{citacao}
``[...] a modelagem matemática começa com um grande problema de ordem prática ou de natureza empírica e depois busca a Matemática que deveria ser utilizada para ajudar a resolver a situação problemática.'' \cite[p.~15]{fossa1998tendencias}.
\end{citacao}

Diante do exposto, a principal diferença entre a resolução de problemas e a modelagem matemática, é que, na primeira, o problema é sempre posto e proposto pelo professor, na segunda, o problema deve ser proposto em conjunto, tanto pelo professor como pelo aluno, sendo que o passo inicial deve ser sempre do aluno.

\subsubsection{O Uso de Computadores}

A aprendizagem mediada por computadores surgiu em 1960 na Universidade de Illinois com o projeto PLATO (Programmed Logic for Automatic Teaching Operations)\cite{bitzer1961plato}, que deu origem ao 
primeiro sistema de ensino assistido por computador, o qual permitia a criação e apresentação de materiais sobre gram\'atica (passar um verbo para o passado, reescrever um substantivo no plural, 
etc.) com revisão automática. De acordo com \citeonline{woolley1994plato}, PLATO apoiou inicialmente apenas uma única sala de aula com 20 alunos, até que em 1972, o sistema migrou para uma nova 
geração de \textit{mainframes} que acabaria por apoiar milhares de terminais gráficos distribuídos em todo o mundo.

O principal fator motivador para a introdução do computador na educação, segundo \citeonline{silva2009ambiente}, foi o surgimento no final do século XX de um conhecimento baseado em simulação, 
característico da cultura informática, o que fez com que o computador fosse visto como  um recurso didático  indispensável.

Os benefícios da utilização do computador como um instrumento de ensino e aprendizagem, de acordo com \citeonline[p.12]{almeida2000proinfo}, referem-se a sua utilização como ``uma máquina que 
possibilita testar ideias ou hipóteses, que levam à criação de um mundo abstrato e simbólico, ao mesmo tempo em que permite introduzir diferentes formas de atuação e interação entre as pessoas''. Já a 
principal associação de professores de matemática dos Estados Unidos (NCTM)\footnote{nota de rodap\'e com o nome da associa\c{c}\~ao e o link do site}, diz que ``a tecnologia é essencial no ensino e 
na aprendizagem da Matemática'' e ``influencia a Matemática que é ensinada e melhora a aprendizagem dos alunos'' permitindo que estes se concentrem ``nas decisões a tomar, na reflexão, no raciocínio e 
na resolução de problemas''. \cite[p.26]{melo2007principios}.

Quando se fala no uso das Tecnologias da Informação e Comunicação (TIC) como um mediador para o processo de ensino e aprendizagem, muitas vezes, acaba-se esquecendo do papel do professor nesse 
processo. Num mundo em que há uma grande variedade de formas de utilização das TIC para apoio a educação, cabe ao professor decidir como e quando utilizar essas tecnologias e quais s\~ao formas 
mais eficazes para sua utilização levando em consideração os conteúdos que serão ofertados. Para isso, o professor necessita mudar sua metodologia de ensino, e isso pode resultar em problemas, já 
que, assim como afirmam \citeonline{bitner2002integrating}:

\begin{citacao}
``Adults do not change easily. Change of any kind brings about fear, anxiety, and concern. Using technology as a teaching and learning tool in the classroom does so to an even greater extent since it involves both changes in classroom procedures and the use of often-unfamiliar technologies. Those responsible for asking teachers to use technology in the curriculum should be aware that fears and concerns do exist.'' \cite{bitner2002integrating}.
\end{citacao}

[COLAGEM]
Os problemas com a inserção do computador na educação podem ser ainda maiores. Segundo \citeonline[p.38]{silva2009ambiente}, ``o uso do computador na educação pode ser problemático, tendo em vista que muito se cogita sobre seu uso no ensino ser a solução para muitos dos problemas da educação''; o referido ainda complementa que ``a maioria destes problemas não podem encontrar resposta nas tecnologias digitais o que pode resultar em uma visão muito simplista sobre o software e seu uso''.

Um dos fatores que pode influência negativamente no processo de aprendizagem mediada por computador é o domínio do computador pelo aluno, tendo em vista que sua rapidez de evolução assim como sua própria complexidade, torna essa tarefa muito difícil de ser alcançada.



\subsection{Ambientes Virtuais de Aprendizagem}

O Surgimento de Ambientes Virtuais de aprendizagem (AVAs) deu-se logo após o surgimento da internet nos anos 90. Nessa mesma época, novas ferramentas e produtos foram desenvolvidas para explorar os benefícios que a rede mundial de computadores trouxe \cite{oleary2002virtual}.

Para \citeonline{valentini2010aprendizagem}, um Ambiente Virtual de Aprendizagem (AVA) é um espaço social, constituído de interações cognitivo-sociais sobre, ou em torno de, um objeto de conhecimento, no qual as pessoas interagem mediadas pela linguagem da hipermídia visando o processo de ensino-aprendizagem.

Geralmente, os AVAs possuem algumas características que os distinguem de outros tipos de sistemas de softwares, segundo \citeonline{oleary2002virtual}, algumas dessas características são:
\begin{alineascomponto}
    \item a comunicação entre tutores e alunos - por exemplo, email, fórum de discussão e bata-papo virtual;
    \item a auto-avaliação e avaliação sumativa - por exemplo, avaliação de múltipla escolha com automatizada marcação e feedback imediato; 
    \item entrega de recursos de aprendizagem e materiais - por exemplo, através do fornecimento de notas de aula e materiais, imagens e clips de vídeo;
    \item áreas do grupo de trabalho compartilhados - possibilita os usuários compartilhares arquivos, bem como se comunicarem;
    \item suporte para estudantes - possibilitar a comunicação entre os tutores e seus estudantes, fornecimentos de materiais didáticos e alguma forma de tirar as dúvidas dos alunos;
    \item gestão e acompanhamento dos estudantes - sistema de autenticação para permitir que apenas estudantes tenham acesso aos cursos;
    \item ferramentas para o estudante - por exemplo agendas e calendários eletrônicos;
    \item aparência consistente e personalizável - uma interface padrão de fácil utilização, permitindo uma personalização, mas um modo de utilização essencial para permanecer constante.
\end{alineascomponto}

A utilização de AVAs traz diversas vantagens como cita \citeonline[p.153]{tajra2001ferramentas}: acessibilidade a fontes inesgotáveis de assuntos para pesquisas, páginas educacionais específicas para a pesquisa escolar, comunicação e interação com outras escolas, estímulo para pesquisar a partir de temas previamente definidos ou a partir da curiosidade dos próprios alunos, estímulo ao raciocínio lógico, troca de experiências entre professores/professores, aluno/aluno e professor/aluno, entre outras.

\citeonline{carvalho2013ambiente} afirma que AVAs integram funcionalidades de comunicação e partilha de informações e isso, permite aceder à aprendizagem de uma forma flexível; em qualquer espaço(\textit{anywhere}) e em qualquer hora (\textit{anytime}), o autor complementa que:

\begin{citacao}
``Um AVA deve, por um lado, enfatizar a aprendizagem através da integração de ferramentas interativas e comunicativas, da partilha de conteúdos multimédia, do alojamento de trabalhos e projetos, da integração de ferramentas de aprendizagem colaborativa, e por outro, deve proporcionar estratégias que potenciem a participação ativa e significativa dos alunos, abranger possibilidades didáticas de aprendizagem individual e em grupo, criar novos acessos a websites como forma de enriquecer o conhecimento, possuir ferramentas de controlo de acesso e registro de utilizadores e de gestão de grupos de trabalho'' \cite{carvalho2013ambiente}. 
\end{citacao}

Os AVAs, geralmente, utilizam diversas ferramentas para apoio ao ensino como já foi citado anteriormente, alguns exemplos são fóruns, chats, wikis, glossários, portfólios, enquetes, questionários, entre outros. Essas ferramentas, de acordo com \citeonline{masetto2012competencia}, são recursos em linguagem digital e podem colaborar significativamente para tornar a educação mais eficiente e eficaz.


\subsection{Gamificação}

Ao longo da história, o homem sempre buscou metodologias inovadoras para auxiliar a educação e  uma das que mais diferem do tradicional método de ensino é a utilização de jogos para o ensino e aprendizagem. 

Em 2002, por meio de Nick Pelling, programador de computadores e inventor britânico, surgiu o termo Gamificação. \citeonline{fardo2013gamificaccao} define a Gamificação como o emprego de conceitos e técnicas tais como sistema de feedback, sistema de recompensas, conflito, cooperação, competição, objetivos e regras claras, níveis, tentativa e erro, diversão, interação, interatividade, entre outros, criados e utilizados em jogos para auxiliar na educação. 

O objetivo da Gamificação não é transformar tudo em um jogo, mas sim, segundo \citeonline{halliwell2013gamification}, ``[...] encontrar a diversão, encontrar os aspectos `jogáveis' de um problema, quaisquer que sejam, e usá-los para criar um ambiente que mova as pessoas um pouco mais em direção a um objetivo que tenham criado''. Um exemplo bem interessante do uso de gamificação é o Duolingo \cite{von2013duolingo}, uma plataforma online de aprendizagem em línguas que trabalha com o conceito de conhecimento coletivo e voluntário. No Duolingo, usuários podem ganhar pontos com as respostas corretas e lições completadas, assim como perder pontos a cada resposta incorreta. O mesmo ainda atribui status de reconhecimento de acordo o conhecimento obtido durante o jogo.

\subsection{Trabalhos Relacionados}

Os \SystemsTypes~vêm sendo utilizados na educação, principalmente como uma ferramenta mais dinâmica comparada as metodologias de ensino tradicionais e são de grande potencial na área da educação. Existem diversas pesquisas voltadas a aplicação de \SystemsTypes~para apoiar o processo de \teachinglearning, em especial a Matemática se destaca entre elas.

\subsubsection{\textit{The one world schoolhouse: Education reimagined}}
Num trabalho iniciado em 2006, \citeonline{khan2012one} funda a chamada KhanAcademy\footnote{Plataforma de aprendizagem disponível em: \url{www.khanacademy.org}}, organização educacional que tem por objetivo oferece exercícios, vídeos de instrução e um painel de aprendizado personalizado que habilita os estudantes a aprender no seu próprio ritmo dentro e fora da sala de aula. Em sua plataforma, são abordados assuntos como matemática, ciência, programação de computadores, história, história da arte, economia e muito mais \cite{khan2012one}.

No ambiente, o desempenho do estudante é representado por medalhas. De acordo com o site da organização, medalhas e insignias estimulam o aprendizado de maneira lúdica. As estatísticas mostram o quanto de trabalho o estudante está fazendo a cada dia, o quanto o estudante está focado em áreas de habilidades e tópicos e as habilidades que o estudante concluiu. Com os relatórios gerados pela plataforma, o tutor pode acompanhar todos os passos do aluno.

Durante uma conferência TED\footnote{É uma série de conferências realizadas na Europa, Ásia e Américas pela fundação Sapling com o objetivo de disseminar ideias que podem mudar o mundo.} em 2011, denominada \textit{Let's use video to reinvent education} \cite{tedtalk2011reinvend}, Khan fala sobre o funcionamento da plataforma e também do provável motivo do seu sucesso. Segundo \citeonline{tedtalk2011reinvend}, o diferencial da plataforma está em dois pontos importantes, a aprendizagem auto-ritmada e nos dados fornecidos aos tutores sobre o aprendizado de seus alunos. 

Em relação a aprendizagem auto-ritmada, Khan afirma que:
\begin{citacao}
``When you talk about self-paced learning, it makes sense for everyone -- in education-speak, ``differentiated learning'' -- but it's kind of crazy, what happens when you see it in a classroom. Because every time we've done this, in every classroom we've done, over and over again, if you go five days into it, there's a group of kids who've raced ahead and a group who are a little bit slower. In a traditional model, in a snapshot assessment, you say, ``These are the gifted kids, these are the slow kids. Maybe they should be tracked differently. Maybe we should put them in different classes.'' But when you let students work at their own pace -- we see it over and over again -- you see students who took a little bit extra time on one concept or the other, but once they get through that concept, they just race ahead.And so the same kids that you thought were slow six weeks ago, you now would think are gifted. And we're seeing it over and over again. It makes you really wonder how much all of the labels maybe a lot of us have benefited from were really just due to a coincidence of time.'' \cite{tedtalk2011reinvend}.
\end{citacao}


\citeonline{tedtalk2011reinvend} também fala sobre a importância dos dados fornecidos pelos tutores:
\begin{citacao}
``[...]. So our paradigm is to arm teachers with as much data as possible -- data that, in any other field, is expected, in finance, marketing, manufacturing -- so the teachers can diagnose what's wrong with the students so they can make their interaction as productive as possible. Now teachers know exactly what the students have been up to, how long they've spent each day, what videos they've watched, when did they pause the videos, what did they stop watching, what exercises are they using, what have they focused on? [...]'' \cite{tedtalk2011reinvend}.
\end{citacao}

Esse trabalho, assim como o descrito aqui, apresenta o uso de um \SystemType~para auxiliar na educação. Dessa forma, esse trabalho servirá como referência para abordar o uso da aprendizagem auto-ritmada na educação matemática, assim como, para definir as informações que deverão ser apresentadas aos tutores sobre a evolução da aprendizagem de seus alunos na plataforma fruto deste trabalho. Contudo, os vídeos que fizeram da Khan Academy tão popular, não farão parte desse trabalho, tendo em vista que, muitos especialistas em educação como Célia Maria Carolina Pires\footnote{Professora da área de didática da Matemática na PUC-SP e pesquisadora de inovações curriculares na Educação Básica e na formação de Professores de Matemática.} acreditam que os vídeos de Khan estão indo contra do que é discutido hoje sobre educação matemática \cite{cartacapital2013} e Fredric Litto\footnote{Presidente da Associação Brasileira de Ensino a Distância (Abed)} que a iniciativa não é revolucionária como muitos dizem, mas sim ``ultrapassada''. Para ele ``O trabalho do Salman Khan tem alguns aspectos novos e outros mais do que tradicionais. O uso de vídeos para melhorar o conhecimento dos alunos, por exemplo, não é novo'' e que ``é a mesma coisa que um professor que dá sua aula em frente ao quadro-negro e o aluno apenas copia no caderno'' \cite{revistaeducacao2013}.

O aspecto mais importante a se considerar aqui está na forma como as duas plataformas lidam com Obstáculos Epistemológicos, segundo \citeonline{bachelard1996formaccao} durante o ato do conhecimento ocorrem ``lentidões e conflitos'', que levam o aluno a parar diante do problema. A esta ``inércia'' é que foi relacionado o conceito. Na metodologia criada por \citeonline{khan2012one}, quando a plataforma é aplicada dentro da sala de aula, o professor pode identificar através da ferramenta, os alunos que estão com esses obstáculos e o mesmo pode intervir para ajudar o aluno a superar essa barreira. No trabalho apresentado aqui, essa barreira será superada quando o sistema, ao identificar o obstáculo, oferecer ajuda ao aluno, caso o mesmo aceite a ajuda, o sistema enviara esse pedido de ajuda a todos os outros alunos que já concluíram a mesma lição com certo nível de proficiência, para que os mesmo possam ou não acatar esse pedido de ajuda e assumir o papel do professor na metodologia de \citeonline{khan2012one}. 


\subsubsection{\textit{ActiveMath: A generic and adaptive web-based learning environment}}

O projeto ActiveMath visa apoiar a aprendizagem verdadeiramente interativa, exploratória e assume que o estudante deve ser responsável por seu aprendizado, até certo ponto. Portanto, uma relativa liberdade para navegar através de um curso e para as escolhas de aprendizagem é dada e, por padrão, o modelo de usuário é inspecionável e modificável \cite{melis2001activemath}.

\citeonline{melis2001activemath} afirma que a maioria dos sistemas tutores inteligentes não contam com uma escolha de adaptação de conteúdos e isso, segundo ele, pode influenciar quando o público-alvo for alunos de faculdades e universidade, já que, diferentemente das escolas de ensino fundamental,  um mesmo assunto é
ensinado de forma diferente para diferentes grupos de utilizadores e em contextos diferentes \cite{melis2001activemath}.

Para conseguir toda essa dinâmica, ActiveMath utiliza regras pedagógicas que definem em quais momentos determinadas funcionalidades do sistema estarão disponíveis, em que ordem os conteúdos serão apresentados para os alunos e como os mesmos deverão ser apresentados.

O trabalho referido,  assim como o desenvolvido por \citeonline{khan2012one} descrito anteriormente, descreve a criação de um \SystemType. Entretanto, o mesmo utiliza técnicas que permitem a geração dinâmica de cursos para os alunos.


\subsubsection{\textit{Resolução de problemas em ambientes virtuais de aprendizagem num curso de licenciatura em matemática na modalidade a distância}}

Esse trabalho desenvolvido por \citeonline{dutra2011resoluccao}, trata da   utilização da metodologia de Resolução de Problemas em ambientes virtuais de  aprendizagem, com o objetivo de investigar que contribuições pode trazer para 
alunos da Licenciatura em Matemática da Universidade Federal de Ouro Preto (UFOP), na Educação a Distância (EaD). Para isso, o ambiente desenvolvido por \citeonline{dutra2011resoluccao} utiliza de fóruns semanais, para discussão e resolução dos problemas; além de chats, utilizados ao final de algumas atividades e um questionário final a ser respondido pelos alunos na última semana de aula.

Essa metodologia segundo \citeonline{dutra2011resoluccao}, funciona seguindo os seguintes passos:
\begin{enumerate}
	\item A atividade é postada na Plataforma Moodle\footnote{``A  palavra  Moodle  referia-se  originalmente  ao  acrônimo:  `Modular Object-Oriented  Dynamic  Learning  Environment'(...).  Em  inglês  a  palavra Moodle é também um verbo que descreve a ação que, com frequência, conduz a resultados criativos, de deambular com preguiça, enquanto se faz com gosto o  que  for  aparecendo  para  fazer''. O  Moodle  deu  o  nome  a  uma  plataforma  de  e-learning,  de  utilização livre  e  código  fonte  aberto,  pela  mão  de  Martin  Dougiamas \cite{oro29585}.} no início da semana, pela manhã (segunda-feira).  Assim,  as  atividades  são  distribuídas  aos  alunos  para  que possam ler, interpretar e entender o problema. 
	\item Os  alunos  passam a  semana  postando  suas  resoluções  dos  problemas e discutindo-as no fórum com os colegas, por meio da Plataforma Moodle. 
	\item A  pesquisadora  observa,  incentivava  e  participava  do  processo  de discussão. Ajuda nos problemas secundários, dando \textit{feedback} das resoluções postadas, respondendo e fazendo perguntas, tirando  dúvidas, acompanhando de perto as discussões entre os alunos no fórum.
	\item As impressões dos alunos sobre os problemas, no início da semana seguinte, e a formalização dos resultados são apresentadas nos chats semanais. Trata-se  de  uma  plenária  virtual  para  discutir  os  problemas,  finalizando-a  com  uma solução aceita por todos. 
	\item Uma resolução, após o chat, é postada na Plataforma Moodle, para todos os pesquisados, observando os conteúdos apresentados nos problemas. 
\end{enumerate}

O ambiente desenvolvido por \citeonline{dutra2011resoluccao} inspirou este trabalho na utilização da metodologia de Resolução de Problema aplicada num AVA através de fóruns e chats, entretanto, a forma como essas ferramentas darão suporte ao ensino em nossa metodologia será adaptada. O fórum que na metodologia de \citeonline{dutra2011resoluccao} era utilizado para a postagem da resolução dos problemas, será utilizado como ferramenta para os alunos tirarem dúvidas sobre os problemas apresentados, já o chat utilizado para discutir os problemas na metodologia de \citeonline{dutra2011resoluccao}, será utilizado para permitir que alunos que concluíram determinado conteúdo com certa proficiência, possa ajudar outros alunos que enfrentem obstaculos epistemológicos ao longo do aprendizado desse conteúdo.


\subsubsection{\textit{A Gamificação Aplicada em Ambientes de Aprendizagem}}

Nesse trabalho desenvolvido por \citeonline{fardo2013gamificaccao}, apresenta-se um conceito sobre gamificação, que vem ganhando visibilidade por sua capacidade de criar experiências significativas quando aplicada em  contextos  da  vida  cotidiana, além de linhas gerais sobre sua aplicação em ambientes de aprendizagem. Embora esse trabalho diferentemente dos anteriores descritos aqui, não apresente a criação de um \SystemType, o mesmo será utilizado como referência para explicar os aspectos da gamificação que serão aplicados no ambiente proposto, assim como a motivação que os levaram a sua utilização.

A aplicação das técnicas da gamificação apresentadas por \citeonline{fardo2013gamificaccao} nesse trabalho, servirá para motivar os alunos durante a utilização da plataforma assim como uma forma de isentivo para os alunos que ajudarem os outros com dificuldades, através de premiação e ranqueamento.















\subsection{Considerações finais}

O sistema que se propôs desenvolver faz uso da metodologia de resolução de problemas, possibilitando o aluno adquirir novos conhecimentos além daqueles estudados no momento. Por se tratar de um AVA, o mesmo possibilitará ao aluno uma maior interação com o professor e outros alunos, além da possibilidade de uma aprendizagem auto-ritmada. A aplicação da gamificação no sistema, servira para desenvolver  engajamento, participação e comprometimento entre os usuários do sistema. 