\section{REVISÃO BIBLIOGRÁFICA}

Este capítulo aborda os seguintes temas: metodologias no ensino da matemática explanando algumas das diversas metodologias usadas no ensino da matemática ao longo da história, conceito de ambientes virtuais de aprendizagem, o uso da gamificação aplicada em AVAs, e por fim o que está sendo desenvolvido por outros pesquisadores nessa área em trabalhos relacionados.

\subsection{Metodologias no Ensino da Matemática}

Ao longo da história, várias metodologias e abordagens matemáticas foram utilizadas visando a melhoria do ensino. Segundo \citeonline{hammes2003tendencias}, algumas delas foram aula expositiva, resolução de problemas, modelagem matemática e o uso de computadores. Nas seções a seguir descreveremos cada uma delas.

\subsubsection{Aula expositiva}

Numa aula expositiva, geralmente o professor faz uma revisão da aula anterior, apresenta o novo conteúdo e no final passa aos alunos uma série de exercícios de fixação. Esse novo conteúdo é apresentado de forma oral ou escrita, sem levar em consideração o conhecimento prévio dos estudantes, nem espaço para perguntas. Essa é, sem dúvida, uma das mais utilizadas e antigas metodologias existentes. Durante o século passado, aulas expositivas foram o único processo empregado em sala de aula pelos professores. Dessa forma, a aula expositiva pode ser considerada cansativa e desinteressante, já que o aluno não participa do processo de ensino  \cite{hammes2003tendencias}.

Para \citeonline{de1996gerencia}, essa abordagem possui diversos problemas. O mesmo ainda sugere que a escola tradicional não somente está desatualizada para atender às necessidades crescentes da sociedade contemporânea, como também apresenta algumas características que inibem o desenvolvimento do potencial de criação dos alunos: 

\begin{alineascomponto}

\item Destaca-se a incompetência, a ignorância e a incapacidade do aluno, deixando de assinalar os talentos e habilidades de cada um; 
\item O ensino voltado para o passado, onde se enfatiza a reprodução e a memorização do conhecimento;
\item Desconsidera-se a imaginação e a fantasia como dimensões importantes da mente;
\item Exercício de resposta única, onde se cultua o medo do erro e do fracasso;
\item A obediência, dependência, passividade e conformismo são os traços mais cultivados;
\item Descaso em cultivar uma visão otimista do futuro;
\item As habilidades cognitivas são desenvolvidas de forma limitada;

\end{alineascomponto}

O papel do professor nessa metodologia acaba tendo uma vital importância no processo. De acordo com \citeonline{brasil1999parametros}, o professor         

\begin{citacao}
``[...] é quem seleciona conteúdos instrucionais compatíveis com os objetivos definidos no projeto pedagógico; problematiza tais conteúdos, promove e media o diálogo educativo; favorece o surgimento de condições para que os alunos assumam o centro da atividade educativa, tornando-se agentes do aprendizado; articula abstrato e concreto, assim como teoria e prática; cuida da continua adequação da linguagem, com a crescente capacidade do aluno, evitando a fala e os símbolos incompreensíveis, assim como as repetições desnecessárias e desmotivantes'' \cite{brasil1999parametros}.
\end{citacao}

Alguns autores como \citeonline{lopes1995aula} por exemplo, defendem que a aula expositiva ``poderá ser transformada em uma atividade dinâmica, participativa e estimuladora do pensamento critico do aluno''. Mas os mesmos afirmam que isso é uma tarefa difícil, já que, em geral, é mal empregado pelos professores. Esse autor complementa ainda que essa nova abordagem ``valoriza a vivência dos alunos, seu conhecimento do concreto, e busca relacionar esses conhecimentos prévios com o assunto a ser estudado''. Ainda de acordo com \citeonline{lopes1995aula}, o professor

\begin{citacao}
``[...] jamais desconsidera uma pergunta em aula, mesmo que ela possa lhe parecer ingênua ou despropositada. Ao perceber uma pergunta mal formulada o papel do professor é ajudar o aluno a refazer a pergunta pois essa atitude educa o aluno a aprender a perguntar''.
\end{citacao}

\subsubsection{Resolução de problemas}

Normalmente, as pessoas costumam criar um conceito equivocado da metodologia de ensino através de resolução de problemas. Muitos acreditam erroneamente que essa metodologia não é nada além de uma lista de exercícios aplicada após a apresentação do conteúdo.

No entanto, a resolução de problemas pode ser entendida como uma oportunidade para o aluno obter novos conhecimentos e não apenas conhecimentos prontos que fazem parte da nossa história. Isso ajuda o aluno a desenvolver sua autonomia, buscando as respostas para seus próprios questionamentos. \citeonline{fossa1998tendencias} ressaltam  que a resolução de problemas 

\begin{citacao}
``[...] visa o desenvolvimento de habilidades metacognitivos, favorecendo a todo momento a reflexão e o questionamento. O aluno aprende a pensar por si mesmo, levantando hipóteses, testando-as, tirando conclusões e até discutindo-as com os colegas.'' \cite{fossa1998tendencias}. 
\end{citacao}

Nessa metodologia, é importante que os problemas apresentem uma incógnita que necessite ser descoberta. Para resolvê-los, o aluno terá que inventar estrategias e gerar novas ideias. Segundo \citeonline{dante1991didatica}

\begin{citacao}
``É importante que o problema possa gerar muitos processos de pensamento, levantar muitas hipóteses e propiciar várias estratégias de solução. O pensar e o fazer criativo devem ser componentes fundamentais no processo de resolução de problemas.'' \cite{dante1991didatica}.
\end{citacao}

\subsubsection{Modelagem Matemática}
Modelagem Matemática diz respeito ao processo de criação de um modelo matemático. \apudonline{granjer1997modelagem}{biembengut1999modelagem} define um modelo matemático como sendo...
\begin{citacao}
``[...] uma imagem que se forma na mente, no momento em que o espírito racional busca compreender e expressar de forma intuitiva uma sensação, procurando relacionar com algo já conhecido, efetuando deduções.''\apud[p.~78]{granjer1997modelagem}{biembengut1999modelagem}
\end{citacao}

A modelagem matemática é uma metodologia que busca proporcionar ao aluno uma visão prática do conhecimento teórico aprendido na sala de aula, através de problemas de ordem prática. Para \citeonline{fossa1998tendencias}
\begin{citacao}
``[...] a modelagem matemática começa com um grande problema de ordem prática ou de natureza empírica e depois busca a Matemática que deveria ser utilizada para ajudar a resolver a situação problemática.'' \cite[p.~15]{fossa1998tendencias}.
\end{citacao}

Diante do exposto, a principal diferença entre a resolução de problemas e a modelagem matemática, é que, na primeira, o problema é sempre posto e proposto pelo professor, na segunda, o problema deve ser proposto em conjunto, tanto pelo professor como pelo aluno, sendo que o passo inicial deve ser sempre do aluno.

\subsubsection{O Uso de Computadores}

A aprendizagem mediada por computador surgiu em 1960 na Universidade de Illinois com o projeto PLATO (Programmed Logic for Automatic Teaching Operations)\cite{bitzer1961plato}, que deu origem ao primeiro sistema de ensino assistido por computador, o qual permitia a criação e apresentação de materiais sobre gramatica (passar um verbo para o passado, reescrever um substantivo no plural, etc.) com revisão automática. De acordo com \citeonline{woolley1994plato}, PLATO apoiou inicialmente apenas uma única sala de aula com 20 alunos, até que em 1972, o sistema migrou para uma nova geração de \textit{mainframes} que acabaria por apoiar milhares de terminais gráficos distribuídos em todo o mundo.

O principal fator motivador para a introdução do computador na educação segundo \citeonline{silva2009ambiente}, foi o surgimento no final do século XX de um conhecimento baseado em simulação, característico da cultura informática, e isso fez com que o computador fosse visto como  um recurso didático cada dia mais indispensável.

Os benefícios da utilização do computador como um objeto de ensino e aprendizagem de acordo com \citeonline[p.12]{almeida2000proinfo}, refere-se a sua utilização como ``uma máquina que possibilita testar ideias ou hipóteses, que levam à criação de um mundo abstrato e simbólico, ao mesmo tempo em que permite introduzir diferentes formas de atuação e interação entre as pessoas''. Já a principal associação de professores de matemática dos Estados Unidos (NCTM), diz que ``a tecnologia é essencial no ensino e na aprendizagem da Matemática'' e ``influencia a Matemática que é ensinada e melhora a aprendizagem dos alunos'' permitindo que estes se concentrem ``nas decisões a tomar, na reflexão, no raciocínio e na resolução de problemas''. \cite[p.26]{melo2007principios}.

Quando se fala no uso das Tecnologias da Informação e Comunicação (TIC) como um mediador para o processo de ensino e aprendizagem, muitas vezes, acabam se esquecendo do papel do professor nesse processo, e num mundo em que há uma grande variedade de formas de utilização das TIC para apoio a educação, cabe ao professor decidir formas mais eficazes para sua utilização, levando em consideração os conteúdos que serão utilizados e como e quando utilizar essas tecnologias. Mas para isso, o professor necessita mudar sua metodologia de ensino, e isso pode resultar em problemas, já que assim como afirma \citeonline{bitner2002integrating}:

\begin{citacao}
``Adults do not change easily. Change of any kind brings about fear, anxiety, and concern. Using technology as a teaching and learning tool in the classroom does so to an even greater extent since it involves both changes in classroom procedures and the use of often-unfamiliar technologies. Those responsible for asking teachers to use technology in the curriculum should be aware that fears and concerns do exist.'' \cite{bitner2002integrating}.
\end{citacao}

Os problemas com a inserção do computador na educação podem ser ainda maiores. Segundo \citeonline[p.38]{silva2009ambiente}, ``o uso do computador na educação pode ser problemático, tendo em vista que muito se cogita sobre seu uso no ensino ser a solução para muitos dos problemas da educação''; o referido ainda complementa que ``a maioria destes problemas não podem encontrar resposta nas tecnologias digitais o que pode resultar em uma visão muito simplista sobre o software e seu uso''.

Um dos fatores que pode influência negativamente no processo de aprendizagem mediada por computador é o domínio do computador pelo aluno, tendo em vista que sua rapidez de evolução assim como sua própria complexidade, torna essa tarefa muito difícil de ser alcançada.



\subsection{Ambientes Virtuais de Aprendizagem}

O Surgimento de Ambientes Virtuais de aprendizagem (AVAs) deu-se logo após o surgimento da internet nos anos 90. Nessa mesma época, novas ferramentas e produtos foram desenvolvidas para explorar os benefícios que a rede mundial de computadores trouxe \cite{oleary2002virtual}.

Para \citeonline{valentini2010aprendizagem}, um Ambiente Virtual de Aprendizagem (AVA) é um espaço social, constituído de interações cognitivo-sociais sobre, ou em torno de, um objeto de conhecimento, no qual as pessoas interagem mediadas pela linguagem da hipermídia visando o processo de ensino-aprendizagem.

Geralmente, os AVAs possuem algumas características que os distinguem de outros tipos de sistemas de softwares, segundo \citeonline{oleary2002virtual}, algumas dessas características são:
\begin{alineascomponto}
    \item a comunicação entre tutores e alunos - por exemplo, email, fórum de discussão e bata-papo virtual;
    \item a auto-avaliação e avaliação sumativa - por exemplo, avaliação de múltipla escolha com automatizada marcação e feedback imediato; 
    \item entrega de recursos de aprendizagem e materiais - por exemplo, através do fornecimento de notas de aula e materiais, imagens e clips de vídeo;
    \item áreas do grupo de trabalho compartilhados - possibilita os usuários compartilhares arquivos, bem como se comunicarem;
    \item suporte para estudantes - possibilitar a comunicação entre os tutores e seus estudantes, fornecimentos de materiais didáticos e alguma forma de tirar as dúvidas dos alunos;
    \item gestão e acompanhamento dos estudantes - sistema de autenticação para permitir que apenas estudantes tenham acesso aos cursos;
    \item ferramentas para o estudante - por exemplo agendas e calendários eletrônicos;
    \item aparência consistente e personalizável - uma interface padrão de fácil utilização, permitindo uma personalização, mas um modo de utilização essencial para permanecer constante.
\end{alineascomponto}

A utilização de AVAs traz diversas vantagens como cita \citeonline[p.153]{tajra2001ferramentas}: acessibilidade a fontes inesgotáveis de assuntos para pesquisas, páginas educacionais específicas para a pesquisa escolar, comunicação e interação com outras escolas, estímulo para pesquisar a partir de temas previamente definidos ou a partir da curiosidade dos próprios alunos, estímulo ao raciocínio lógico, troca de experiências entre professores/professores, aluno/aluno e professor/aluno, entre outras.

\citeonline{carvalho2013ambiente} afirma que AVAs integram funcionalidades de comunicação e partilha de informações e isso, permite aceder à aprendizagem de uma forma flexível; em qualquer espaço(\textit{anywhere}) e em qualquer hora (\textit{anytime}), o autor complementa que:

\begin{citacao}
``Um AVA deve, por um lado, enfatizar a aprendizagem através da integração de ferramentas interativas e comunicativas, da partilha de conteúdos multimédia, do alojamento de trabalhos e projetos, da integração de ferramentas de aprendizagem colaborativa, e por outro, deve proporcionar estratégias que potenciem a participação ativa e significativa dos alunos, abranger possibilidades didáticas de aprendizagem individual e em grupo, criar novos acessos a websites como forma de enriquecer o conhecimento, possuir ferramentas de controlo de acesso e registro de utilizadores e de gestão de grupos de trabalho'' \cite{carvalho2013ambiente}. 
\end{citacao}

Os AVAs, geralmente, utilizam diversas ferramentas para apoio ao ensino como já foi citado anteriormente, alguns exemplos são fóruns, chats, wikis, glossários, portfólios, enquetes, questionários, entre outros. Essas ferramentas, de acordo com \citeonline{masetto2012competencia}, são recursos em linguagem digital e podem colaborar significativamente para tornar a educação mais eficiente e eficaz.


\subsection{Gamificação}

Ao longo da história, o homem sempre buscou metodologias inovadoras para auxiliar a educação e  uma das que mais diferem do tradicional método de ensino é a utilização de jogos para o ensino e aprendizagem. 

Em 2002, por meio de Nick Pelling, programador de computadores e inventor britânico, surgiu o termo Gamificação. \citeonline{fardo2013gamificaccao} define a Gamificação como o emprego de conceitos e técnicas tais como sistema de feedback, sistema de recompensas, conflito, cooperação, competição, objetivos e regras claras, níveis, tentativa e erro, diversão, interação, interatividade, entre outros, criados e utilizados em jogos para auxiliar na educação. 

O objetivo da Gamificação não é transformar tudo em um jogo, mas sim, segundo \citeonline{halliwell2013gamification}, ``[...] encontrar a diversão, encontrar os aspectos `jogáveis' de um problema, quaisquer que sejam, e usá-los para criar um ambiente que mova as pessoas um pouco mais em direção a um objetivo que tenham criado''. Um exemplo bem interessante do uso de gamificação é o Duolingo \cite{von2013duolingo}, uma plataforma online de aprendizagem em línguas que trabalha com o conceito de conhecimento coletivo e voluntário. No Duolingo, usuários podem ganhar pontos com as respostas corretas e lições completadas, assim como perder pontos a cada resposta incorreta. O mesmo ainda atribui status de reconhecimento de acordo o conhecimento obtido durante o jogo.

\subsection{Trabalhos Relacionados}

Os Ambientes Virtuais de Aprendizagem vêm sendo utilizados na educação, principalmente como uma ferramenta mais dinâmica comparada as metodologias de ensino tradicionais e são de grande potencial na 
área da educação. Existem diversas pesquisas voltadas a aplicação de AVAs para apoiar o processo de ensino-aprendizagem e, em especial, a Matemática se destaca entre elas.

\subsubsection{\textit{The one world schoolhouse: Education reimagined}}
Em um trabalho iniciado em 2006, \citeonline{khan2012one} fundou a chamada KhanAcademy\footnote{Plataforma de aprendizagem disponível em: \url{www.khanacademy.org}}, organização educacional que tem 
por objetivo oferecer exercícios, vídeos de instrução e um painel de aprendizado personalizado que habilita os estudantes a aprender no seu próprio ritmo dentro e fora da sala de aula. Em sua 
plataforma, são abordados assuntos como matemática, ciência, programação de computadores, história, história da arte, economia e muito mais \cite{khan2012one}.

No ambiente, o desempenho do estudante é representado por medalhas. De acordo com o site da organização, medalhas e insignias estimulam o aprendizado de maneira lúdica. As estatísticas mostram o quanto de trabalho o estudante está fazendo a cada dia, o quanto o estudante está focado em áreas de habilidades e tópicos e as habilidades que o estudante concluiu. Com os relatórios gerados pela plataforma, o tutor pode acompanhar todos os passos do aluno.

Durante uma conferência TED\footnote{É uma série de conferências realizadas na Europa, Ásia e Américas pela fundação Sapling com o objetivo de disseminar ideias que podem mudar o mundo.} em 2011, denominada \textit{Let's use video to reinvent education} \cite{tedtalk2011reinvend}, Khan fala sobre o funcionamento da plataforma e também do provável motivo do seu sucesso. Segundo \citeonline{tedtalk2011reinvend}, o diferencial da plataforma está em dois pontos importantes, a aprendizagem auto-ritmada e nos dados fornecidos aos tutores sobre o aprendizado de seus alunos. 

Em relação a aprendizagem auto-ritmada, Khan afirma que:
\begin{citacao}
``Então quando fala-se em aprendizagem auto-ritmada, isso faz sentido para todo mundo – na chamada aprendizagem diferenciada – mas é meio maluco quando você vê isso na sala de aula. Porque toda vez 
que fizemos isso, em cada sala de aula que fizemos, repetidas vezes, depois de cinco dias nisso, há um grupo de garotos que está adiantado e há outro grupo de garotos um pouco atrás. E num modelo 
tradicional, se você fizesse uma avaliação pontual, diria: ``Esses são garotos inteligentes, aqueles são garotos lerdos. Talvez eles devessem ser acompanhados de forma diferente. Talvez devêssemos 
coloca-los em salas diferentes.'' Mas quando você deixa cada aluno trabalhar em seu próprio ritmo – e vemos isso repetidas vezes – você vê alunos que tomam um tempo extra em um conceito ou outro, mas 
uma vez que adquirem esse conceito, eles apenas vão adiante. E assim os mesmos garotos que você pensava que eram lerdos semanas atrás, agora você pensa que são inteligentes. E vemos isso repetidas 
vezes. E isso faz você se perguntar quantos estereótipos que talvez vários de nós recebemos eram apenas devido a uma coincidência de tempo'' \cite[13:29, Tradu\c{c}\~ao 
Livre]{tedtalk2011reinvend}.
\end{citacao}.


\citeonline{tedtalk2011reinvend} também fala sobre a importância dos dados fornecidos pelos tutores:
\begin{citacao}
``[...]. Nosso paradigma é armar os professores com a maior quantidade de dados possível – dados que, em quase qualquer outro campo, são esperados, se você trabalha com finanças ou propaganda ou 
fabricação. E assim o professores podem realmente diagnosticar o que está errado com os alunos de maneira que podem fazer suas interações mais produtivas possível. Agora os professores sabem 
exatamente o que os alunos têm feito, quanto tempo eles gastam todo dia, a quais vídeos assistiram, quando pausaram os vídeos, o que fez que parassem de ver, quais exercícios estavam fazendo, no que 
eles estavam se focando? [...]'' \cite[12:26, Tradu\c{c}\~ao Livre]{tedtalk2011reinvend}.
\end{citacao}

O trabalho de \citeonline{khan2012one}, assim como o apresentado nesta monografia, apresenta o uso de um AVA para auxiliar na educação. Dessa forma, esse trabalho servirá como referência para abordar 
o uso da aprendizagem auto-ritmada na educação matemática, assim como, para definir as informações que deverão ser apresentadas aos tutores sobre a evolução da aprendizagem de seus alunos na 
plataforma fruto deste trabalho. Contudo, os vídeos que fizeram da Khan Academy tão popular, não farão parte desse trabalho.

O aspecto mais importante a se considerar aqui está na forma como as duas plataformas lidam com Obstáculos Epistemológicos. Segundo \citeonline{bachelard1996formaccao}, durante o ato do conhecimento, 
ocorrem ``lentidões e conflitos'' que levam o aluno a parar diante do problema. A esta ``inércia'' é que foi relacionado o conceito. Na metodologia criada por \citeonline{khan2012one}, quando a 
plataforma é aplicada dentro da sala de aula, o professor pode identificar através da ferramenta, os alunos que estão com esses obstáculos e o mesmo pode intervir para ajudar o aluno a superar essa 
barreira. 

No trabalho apresentado aqui, essa barreira será superada quando o sistema, ao identificar o obstáculo enfrentado, indicar ao aluno a exist\^encia desse obst\'aculo e o(s) conte\'udo(s) que ele 
possui defici\^encia (causador(es) da barreira) para ele assim poder pausar o conte\'udo que est\'a estudando e voltar ou n\~ao a estudar o(s) conte\'udo(s) que o sistema indicar.


\subsubsection{\textit{Duolingo: Learn a Language for Free while Helping to Translate the Web}}\label{trabalho_relacionados_duolingo}

Nesse trabalho desenvolvido por \citeonline{von2013duolingo}, \'e apresentado o Duolingo, uma plataforma de ensino de idiomas e tradu\c{c}\~ao autom\'atica de documentos. O ambiente funciona de 
maneira que os usuários progridam nas lições ao mesmo tempo que traduzem conteúdo real da internet. 

O método utilizado pela plataforma se caracteriza pela li\c{c}\~oes fragmentadas, pelas quais os 
alunos, atrav\'es do m\'etodo de repeti\c{c}\~ao, fixam o conte\'udo da língua estudada. A medida que o usu\'ario avan\c{c}a, ele progride em uma \'arvore de habilidades que leva o estudante 
gradativamente ao fim do curso, enquanto oferece a op\c{c}\~ao de voltar atr\'as para refazer li\c{c}\~oes antigas que j\'a poderiam ter sidas esquecidas. 


Um estudo realizado na Universidade da Cidade de Nova York \cite{vesselinov2012duolingo} disse que 34 horas gastas no Duolingo igualou-se a um semestre de um curso de l\'inguas.

Uma das características mais marcantes dessa plataforma \'e quantidade de técnicas de Gamifica\c{c}\~ao empregadas. Possui sistema de pontua\c{c}\~ao, n\'iveis, rankings, miss\~oes, medalhas, 
personaliza\c{c}\~ao, entre outras. 

Assim como a plataforma desenvolvida por \citeonline{von2013duolingo}, o ambiente que desenvolveremos tamb\'em contar\'a com t\'ecnicas de Gamifica\c{c}\~ao que ser\~ao utilizadas para melhorar a 
experi\^encia do usu\'ario, assim como um para motivar os usu\'arios durante sua utiliza\c{c}\~ao.   


\subsubsection{\textit{ActiveMath: A generic and adaptive web-based learning environment}}

O projeto ActiveMath visa apoiar a aprendizagem verdadeiramente interativa, exploratória, e assume que o estudante deve ser responsável por seu aprendizado, até certo ponto. Portanto, uma relativa 
liberdade para navegar através de um curso e para as escolhas de aprendizagem lhes é dada e, por padrão, o modelo de usuário é inspecionável e modificável \cite{melis2001activemath}.

\citeonline{melis2001activemath} afirmam que a maioria dos sistemas tutores inteligentes não contam com uma escolha de adaptação de conteúdos e isso, segundo ele, pode influenciar quando o 
público-alvo for alunos de faculdades e universidade, já que, diferentemente das escolas de ensino fundamental,  um mesmo assunto é
ensinado de forma diferente para diferentes grupos de utilizadores e em contextos diferentes \cite{melis2001activemath}.

Para conseguir um ambiente dinâmico de aprendizagem, ActiveMath utiliza regras pedagógicas que definem em quais momentos determinadas funcionalidades do sistema estarão disponíveis, em que ordem os 
conteúdos serão apresentados para os alunos e como os mesmos deverão ser apresentados. O trabalho referido,  assim como o desenvolvido por \citeonline{khan2012one} descrito anteriormente, descreve a 
criação de um AVA. Mas tamb\'em, o mesmo utiliza técnicas que permitem a geração dinâmica de cursos para os alunos.

O trabalho desenvolvido por \citeonline{melis2001activemath} assim como o descrito aqui, apresenta um subsistema de exerc\'icios que suporta diagn\'osticos de erros e equ\'ivocos dos estudantes, que 
gera estrat\'egias tutoriais configur\'aveis para o \textit{feedback}.

\subsubsection{\textit{Resolução de problemas em ambientes virtuais de aprendizagem num curso de licenciatura em matemática na modalidade a distância}}

O trabalho desenvolvido por \citeonline{dutra2011resoluccao}, trata da utilização da metodologia de Resolução de Problemas (apresentada na \autoref{resolucao_problemas}) em um AVA com o objetivo de 
investigar as contribuições que essa jun\c{c}\~ao pode trazer para um curso de Licenciatura em Matem\'atica da Universidade Federal de Ouro Preto (UFOP), na Educa\c{c}\~ao \`a Dist\^ancia. Para isso, 
o ambiente desenvolvido por \citeonline{dutra2011resoluccao} utiliza de fóruns semanais para discussão e resolução dos problemas, além de chats, utilizados ao final de algumas atividades, e um 
questionário final a ser respondido pelos alunos na última semana de aula.

Essa metodologia, segundo \citeonline{dutra2011resoluccao}, funciona atrav\'es dos seguintes passos:
\begin{enumerate}
	\item A atividade é postada na Plataforma Moodle\footnote{``A  palavra  Moodle  referia-se  originalmente  ao  acrônimo:  `Modular Object-Oriented  Dynamic  Learning  Environment'(...).  Em  inglês  a  palavra Moodle é também um verbo que descreve a ação que, com frequência, conduz a resultados criativos, de deambular com preguiça, enquanto se faz com gosto o  que  for  aparecendo  para  fazer''. O  Moodle  deu  o  nome  a  uma  plataforma  de  e-learning,  de  utilização livre  e  código  fonte  aberto,  pela  mão  de  Martin  Dougiamas \cite{oro29585}.} no início da semana, pela manhã (segunda-feira).  Assim,  as  atividades  são  distribuídas  aos  alunos  para  que possam ler, interpretar e entender o problema. 
	\item Os  alunos  passam a  semana  postando  suas  resoluções  dos  problemas e discutindo-as no fórum com os colegas, por meio da Plataforma Moodle. 
	\item A  pesquisadora  observa,  incentivava  e  participava  do  processo  de discussão. Ajuda nos problemas secundários, dando \textit{feedback} das resoluções postadas, respondendo e fazendo perguntas, tirando  dúvidas, acompanhando de perto as discussões entre os alunos no fórum.
	\item As impressões dos alunos sobre os problemas, no início da semana seguinte, e a formalização dos resultados são apresentadas nos chats semanais. Trata-se  de  uma  plenária  virtual  para  discutir  os  problemas,  finalizando-a  com  uma solução aceita por todos. 
	\item Uma resolução, após o chat, é postada na Plataforma Moodle, para todos os pesquisados, observando os conteúdos apresentados nos problemas. 
\end{enumerate}

O ambiente desenvolvido por \citeonline{dutra2011resoluccao} inspirou este trabalho na utilização da metodologia de Resolução de Problema aplicada num AVA através de fóruns e chats, entretanto, a forma como essas ferramentas darão suporte ao ensino em nossa metodologia será adaptada. O fórum que na metodologia de \citeonline{dutra2011resoluccao} era utilizado para a postagem da resolução dos problemas, será utilizado como ferramenta para os alunos tirarem dúvidas sobre os problemas apresentados, já o chat utilizado para discutir os problemas na metodologia de \citeonline{dutra2011resoluccao}, será utilizado para permitir que alunos que concluíram determinado conteúdo com certa proficiência, possa ajudar outros alunos que enfrentem obstaculos epistemológicos ao longo do aprendizado desse conteúdo.











\subsection{Considerações finais}

O sistema que se propôs desenvolver faz uso da metodologia de resolução de problemas, possibilitando o aluno adquirir novos conhecimentos além daqueles estudados no momento. Por se tratar de um AVA, o mesmo possibilitará ao aluno uma maior interação com o professor e outros alunos, além da possibilidade de uma aprendizagem auto-ritmada. A aplicação da gamificação no sistema, servira para desenvolver  engajamento, participação e comprometimento entre os usuários do sistema. 