\subsection{Trabalhos Relacionados}

Os \SystemsTypes~vêm sendo utilizados na educação, principalmente como uma ferramenta mais dinâmica comparada as metodologias de ensino tradicionais e são de grande potencial na área da educação. Existem diversas pesquisas voltadas a aplicação de \SystemsTypes~para apoiar o processo de \teachinglearning, em especial a Matemática se destaca entre elas.

\subsubsection{\textit{The one world schoolhouse: Education reimagined}}
Num trabalho iniciado em 2006, \citeonline{khan2012one} funda a chamada KhanAcademy\footnote{Plataforma de aprendizagem disponível em: \url{www.khanacademy.org}}, organização educacional que tem por objetivo oferece exercícios, vídeos de instrução e um painel de aprendizado personalizado que habilita os estudantes a aprender no seu próprio ritmo dentro e fora da sala de aula. Em sua plataforma, são abordados assuntos como matemática, ciência, programação de computadores, história, história da arte, economia e muito mais \cite{khan2012one}.

No ambiente, o desempenho do estudante é representado por medalhas. De acordo com o site da organização, medalhas e insignias estimulam o aprendizado de maneira lúdica. As estatísticas mostram o quanto de trabalho o estudante está fazendo a cada dia, o quanto o estudante está focado em áreas de habilidades e tópicos e as habilidades que o estudante concluiu. Com os relatórios gerados pela plataforma, o tutor pode acompanhar todos os passos do aluno.

Durante uma conferência TED\footnote{É uma série de conferências realizadas na Europa, Ásia e Américas pela fundação Sapling com o objetivo de disseminar ideias que podem mudar o mundo.} em 2011, denominada \textit{Let's use video to reinvent education} \cite{tedtalk2011reinvend}, Khan fala sobre o funcionamento da plataforma e também do provável motivo do seu sucesso. Segundo \citeonline{tedtalk2011reinvend}, o diferencial da plataforma está em dois pontos importantes, a aprendizagem auto-ritmada e nos dados fornecidos aos tutores sobre o aprendizado de seus alunos. 

Em relação a aprendizagem auto-ritmada, Khan afirma que:
\begin{citacao}
``When you talk about self-paced learning, it makes sense for everyone -- in education-speak, ``differentiated learning'' -- but it's kind of crazy, what happens when you see it in a classroom. Because every time we've done this, in every classroom we've done, over and over again, if you go five days into it, there's a group of kids who've raced ahead and a group who are a little bit slower. In a traditional model, in a snapshot assessment, you say, ``These are the gifted kids, these are the slow kids. Maybe they should be tracked differently. Maybe we should put them in different classes.'' But when you let students work at their own pace -- we see it over and over again -- you see students who took a little bit extra time on one concept or the other, but once they get through that concept, they just race ahead.And so the same kids that you thought were slow six weeks ago, you now would think are gifted. And we're seeing it over and over again. It makes you really wonder how much all of the labels maybe a lot of us have benefited from were really just due to a coincidence of time.'' \cite{tedtalk2011reinvend}.
\end{citacao}


\citeonline{tedtalk2011reinvend} também fala sobre a importância dos dados fornecidos pelos tutores:
\begin{citacao}
``[...]. So our paradigm is to arm teachers with as much data as possible -- data that, in any other field, is expected, in finance, marketing, manufacturing -- so the teachers can diagnose what's wrong with the students so they can make their interaction as productive as possible. Now teachers know exactly what the students have been up to, how long they've spent each day, what videos they've watched, when did they pause the videos, what did they stop watching, what exercises are they using, what have they focused on? [...]'' \cite{tedtalk2011reinvend}.
\end{citacao}

Esse trabalho, assim como o descrito aqui, apresenta o uso de um \SystemType~para auxiliar na educação. Dessa forma, esse trabalho servirá como referência para abordar o uso da aprendizagem auto-ritmada na educação matemática, assim como, para definir as informações que deverão ser apresentadas aos tutores sobre a evolução da aprendizagem de seus alunos na plataforma fruto deste trabalho. Contudo, os vídeos que fizeram da Khan Academy tão popular, não farão parte desse trabalho, tendo em vista que, muitos especialistas em educação como Célia Maria Carolina Pires\footnote{Professora da área de didática da Matemática na PUC-SP e pesquisadora de inovações curriculares na Educação Básica e na formação de Professores de Matemática.} acreditam que os vídeos de Khan estão indo contra do que é discutido hoje sobre educação matemática \cite{cartacapital2013} e Fredric Litto\footnote{Presidente da Associação Brasileira de Ensino a Distância (Abed)} que a iniciativa não é revolucionária como muitos dizem, mas sim ``ultrapassada''. Para ele ``O trabalho do Salman Khan tem alguns aspectos novos e outros mais do que tradicionais. O uso de vídeos para melhorar o conhecimento dos alunos, por exemplo, não é novo'' e que ``é a mesma coisa que um professor que dá sua aula em frente ao quadro-negro e o aluno apenas copia no caderno'' \cite{revistaeducacao2013}.

O aspecto mais importante a se considerar aqui está na forma como as duas plataformas lidam com Obstáculos Epistemológicos, segundo \citeonline{bachelard1996formaccao} durante o ato do conhecimento ocorrem ``lentidões e conflitos'', que levam o aluno a parar diante do problema. A esta ``inércia'' é que foi relacionado o conceito. Na metodologia criada por \citeonline{khan2012one}, quando a plataforma é aplicada dentro da sala de aula, o professor pode identificar através da ferramenta, os alunos que estão com esses obstáculos e o mesmo pode intervir para ajudar o aluno a superar essa barreira. No trabalho apresentado aqui, essa barreira será superada quando o sistema, ao identificar o obstáculo, oferecer ajuda ao aluno, caso o mesmo aceite a ajuda, o sistema enviara esse pedido de ajuda a todos os outros alunos que já concluíram a mesma lição com certo nível de proficiência, para que os mesmo possam ou não acatar esse pedido de ajuda e assumir o papel do professor na metodologia de \citeonline{khan2012one}. 


\subsubsection{\textit{ActiveMath: A generic and adaptive web-based learning environment}}

O projeto ActiveMath visa apoiar a aprendizagem verdadeiramente interativa, exploratória e assume que o estudante deve ser responsável por seu aprendizado, até certo ponto. Portanto, uma relativa liberdade para navegar através de um curso e para as escolhas de aprendizagem é dada e, por padrão, o modelo de usuário é inspecionável e modificável \cite{melis2001activemath}.

\citeonline{melis2001activemath} afirma que a maioria dos sistemas tutores inteligentes não contam com uma escolha de adaptação de conteúdos e isso, segundo ele, pode influenciar quando o público-alvo for alunos de faculdades e universidade, já que, diferentemente das escolas de ensino fundamental,  um mesmo assunto é
ensinado de forma diferente para diferentes grupos de utilizadores e em contextos diferentes \cite{melis2001activemath}.

Para conseguir toda essa dinâmica, ActiveMath utiliza regras pedagógicas que definem em quais momentos determinadas funcionalidades do sistema estarão disponíveis, em que ordem os conteúdos serão apresentados para os alunos e como os mesmos deverão ser apresentados.

O trabalho referido,  assim como o desenvolvido por \citeonline{khan2012one} descrito anteriormente, descreve a criação de um \SystemType. Entretanto, o mesmo utiliza técnicas que permitem a geração dinâmica de cursos para os alunos.


\subsubsection{\textit{Resolução de problemas em ambientes virtuais de aprendizagem num curso de licenciatura em matemática na modalidade a distância}}

Esse trabalho desenvolvido por \citeonline{dutra2011resoluccao}, trata da   utilização da metodologia de Resolução de Problemas em ambientes virtuais de  aprendizagem, com o objetivo de investigar que contribuições pode trazer para 
alunos da Licenciatura em Matemática da Universidade Federal de Ouro Preto (UFOP), na Educação a Distância (EaD). Para isso, o ambiente desenvolvido por \citeonline{dutra2011resoluccao} utiliza de fóruns semanais, para discussão e resolução dos problemas; além de chats, utilizados ao final de algumas atividades e um questionário final a ser respondido pelos alunos na última semana de aula.

Essa metodologia segundo \citeonline{dutra2011resoluccao}, funciona seguindo os seguintes passos:
\begin{enumerate}
	\item A atividade é postada na Plataforma Moodle\footnote{``A  palavra  Moodle  referia-se  originalmente  ao  acrônimo:  `Modular Object-Oriented  Dynamic  Learning  Environment'(...).  Em  inglês  a  palavra Moodle é também um verbo que descreve a ação que, com frequência, conduz a resultados criativos, de deambular com preguiça, enquanto se faz com gosto o  que  for  aparecendo  para  fazer''. O  Moodle  deu  o  nome  a  uma  plataforma  de  e-learning,  de  utilização livre  e  código  fonte  aberto,  pela  mão  de  Martin  Dougiamas \cite{oro29585}.} no início da semana, pela manhã (segunda-feira).  Assim,  as  atividades  são  distribuídas  aos  alunos  para  que possam ler, interpretar e entender o problema. 
	\item Os  alunos  passam a  semana  postando  suas  resoluções  dos  problemas e discutindo-as no fórum com os colegas, por meio da Plataforma Moodle. 
	\item A  pesquisadora  observa,  incentivava  e  participava  do  processo  de discussão. Ajuda nos problemas secundários, dando \textit{feedback} das resoluções postadas, respondendo e fazendo perguntas, tirando  dúvidas, acompanhando de perto as discussões entre os alunos no fórum.
	\item As impressões dos alunos sobre os problemas, no início da semana seguinte, e a formalização dos resultados são apresentadas nos chats semanais. Trata-se  de  uma  plenária  virtual  para  discutir  os  problemas,  finalizando-a  com  uma solução aceita por todos. 
	\item Uma resolução, após o chat, é postada na Plataforma Moodle, para todos os pesquisados, observando os conteúdos apresentados nos problemas. 
\end{enumerate}

O ambiente desenvolvido por \citeonline{dutra2011resoluccao} inspirou este trabalho na utilização da metodologia de Resolução de Problema aplicada num AVA através de fóruns e chats, entretanto, a forma como essas ferramentas darão suporte ao ensino em nossa metodologia será adaptada. O fórum que na metodologia de \citeonline{dutra2011resoluccao} era utilizado para a postagem da resolução dos problemas, será utilizado como ferramenta para os alunos tirarem dúvidas sobre os problemas apresentados, já o chat utilizado para discutir os problemas na metodologia de \citeonline{dutra2011resoluccao}, será utilizado para permitir que alunos que concluíram determinado conteúdo com certa proficiência, possa ajudar outros alunos que enfrentem obstaculos epistemológicos ao longo do aprendizado desse conteúdo.


\subsubsection{\textit{A Gamificação Aplicada em Ambientes de Aprendizagem}}

Nesse trabalho desenvolvido por \citeonline{fardo2013gamificaccao}, apresenta-se um conceito sobre gamificação, que vem ganhando visibilidade por sua capacidade de criar experiências significativas quando aplicada em  contextos  da  vida  cotidiana, além de linhas gerais sobre sua aplicação em ambientes de aprendizagem. Embora esse trabalho diferentemente dos anteriores descritos aqui, não apresente a criação de um \SystemType, o mesmo será utilizado como referência para explicar os aspectos da gamificação que serão aplicados no ambiente proposto, assim como a motivação que os levaram a sua utilização.

A aplicação das técnicas da gamificação apresentadas por \citeonline{fardo2013gamificaccao} nesse trabalho, servirá para motivar os alunos durante a utilização da plataforma assim como uma forma de isentivo para os alunos que ajudarem os outros com dificuldades, através de premiação e ranqueamento.













