\section{RESULTADOS PRELIMINARES}

Nesta seção apresentaremos o andamento deste trabalho.

Até o momento, de concluído, temos o processo que está sendo utilizado durante o desenvolvimento do sistema, os requisitos do sistema que já foram coletados, analisados e validados, além do próprio projeto do sistema. 

Na fase que está em andamento, que é a fase de implementação, temos os seguintes módulos concluídos: 

\begin{alineascomponto}
	\item Gerenciador de Usuários: módulo responsável por gerenciar os usuários do sistema como professores, assistentes e alunos.
	\item Gerenciador de Turmas: módulo responsável por gerenciar as turmas de alunos do sistema.
    \item Gerenciador de Disciplinas: módulo responsável por gerenciar as disciplinas que serão cadastradas no sistema.
    \item Gerenciador de Lições: módulo responsável por gerenciar as lições que os professores irão cadastrar no sistema.
    \item Gerenciador de Questões: módulo responsável por gerenciar os problemas que os assistentes e professores poderão cadastrar para cada lição.
    \item Gerenciador de Pontuação: módulo responsável por gerenciar a pontuação ganha pelos alunos, assim como seu nível de experiência ao longo da utilização do sistema. 
    \item Fórum: módulo responsável por permitir que alunos postem dúvidas dos mais variados  assuntos relacionadas ao sistema, seja dúvidas em relação ao conteúdo apresentado em sala de aula, assim como informações sobre o sistema e sugestões. 
\end{alineascomponto}

Os módulos que ainda restam para serem desenvolvidos nesta fase são:

\begin{alineascomponto}
	\item Gerenciador de Progresso: módulo responsável por acompanhar o andamento de cada estudante durante seu aprendizado e identificar os obstáculo epistemológicos enfrentados, indicando ao aluno 
a exist\^encia desses obst\'aculos e o(s) conte\'udo(s) que ele possui defici\^encia (causador(es) do obstáculo) para ele assim poder pausar o conte\'udo que est\'a estudando e voltar a 
estudar o(s) conte\'udo(s) que o sistema indicar.

	\item Gerador de Estatísticas: módulo responsável por gerar as estatísticas que o professor utilizará para acompanhar o andamento de suas turmas e alunos, assim como para o uso pelo estudante, 
que utilizará para acompanhar seu próprio progresso durante sua aprendizagem no sistema.
    
    \item Ranking: módulo responsável por manter um ranking\footnote{É uma classificação ordenada de acordo com critérios determinados.} com os posicionamentos dos alunos de acordo com seu desempenho 
no sistema durante a semana.
    
\end{alineascomponto}

A seguir, apresentaremos algumas telas do sistema:

\begin{figure}[H]
  \centering
  \begin{minipage}[b]{0.49\textwidth}
	\caption{Tela Inicial}
    \includegraphics[width=\textwidth]{figuras/askmath/1}
    \fnote{Fonte: \url{www.askmath.quixada.ufc.br}.}
  \end{minipage}
  \hfill
  \begin{minipage}[b]{0.49\textwidth}
	\caption{Tela Inicial do Estudante}
    \includegraphics[width=\textwidth]{figuras/askmath/2}
  	\fnote{Fonte: \url{www.askmath.quixada.ufc.br}.}
  \end{minipage}
 
  \begin{minipage}[b]{0.49\textwidth}
    \caption{Tela de Administra\c{c}\~ao}
    \includegraphics[width=\textwidth]{figuras/askmath/3}
    \fnote{Fonte: \url{www.askmath.quixada.ufc.br}.}
  \end{minipage}
  \hfill
  \begin{minipage}[b]{0.49\textwidth}
	\caption{Tela de Problemas do Estudante}
    \includegraphics[width=\textwidth]{figuras/askmath/4}
    \fnote{Fonte: \url{www.askmath.quixada.ufc.br}.}
  \end{minipage}
\end{figure}

Os conteúdos que serão utilizados para popular o sistema, durante sua aplicação na UFC-Campus Quixadá, já estão sendo desenvolvidos pelos monitores citados anteriormente. Após a validação e verificação do sistema, esses conteúdos serão adicionados por esses monitores, que passaram por um treinamento para aprenderem a utilizar o sistema.







