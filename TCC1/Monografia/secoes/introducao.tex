\section{INTRODUÇÃO}

A Matemática, como ciência, tem uma relação muito especial com as novas tecnologias desde as calculadoras aos computadores, sistemas multimédia, e a Internet. No entanto, alguns professores costumam demorar a perceber como tirar proveito destas tecnologias como ferramentas de trabalho \cite{da1997ensino}. \`A medida que a quantidade de recursos tecnológicos na sala de aula foram aumentando, tornou-se conveniente a criação de novas metodologias de ensino, especificamente na Educação Matemática. A 
busca por novas metodologias de ensino que fazem uso destes recursos procura fazer da matemática uma disciplina atraente e desvinculada do ensino tradicional que já se mostrou ineficiente 
\cite{silva2009ambiente}.

Grande parte dos estudantes tem dificuldades em aprender matemática, e muitas vezes essas dificuldades ocorrem não pela falta de atenção ou por não gostar do conteúdo, mas por fatores mentais ou psicológicos que envolvem uma série de trabalhos e conceitos que precisam ser desenvolvidos \cite{sa2015software}. Mas como auxiliar estudantes com dificuldades na aprendizagem da matemática? Em busca dessa resposta, diferentes sistemas de \textit{software} foram desenvolvidos buscando servir \`a  educação. Em 2006, Salman Khan fundou a Khan Academy, uma organização educacional que tem 
por objetivo oferecer exercícios, vídeos de instrução e um painel de aprendizado personalizado que habilita os estudantes a aprender no seu próprio ritmo dentro e fora da sala de aula 
\cite{khan2012one}. A plataforma criada por Khan utiliza vídeo-aulas e resolução de problemas para ensinar seus alunos, permitindo que cada um aprenda de forma independente e no seu pr\'oprio ritmo. Um outro trabalho também  importante nessa área, \'e o de  \citeonline{melis2001activemath}. Os autores desenvolvem uma plataforma que permite aos estudantes desfrutarem da experiência de estudar num curso gerado dinamicamente. Os conte\'udos s\~ao representados no formato XML \cite{bray1998extensible} e armazenados numa base de conhecimento, onde s\~ao recuperados para gerar cursos individuais de acordo com regras pedagógicas\footnote{S\~ao regras que determinam quando e quais ferramentas do sistema devem apresentadas e em que ordem.} \cite{melis2004activemath}. Estes e outros trabalhos serão 
apresentados com mais detalhes na \autoref{trabalhos_relacionados}.

O presente trabalho apresenta o projeto de um Ambiente Virtual de Aprendizagem (AVA) \cite{valentini2010aprendizagem}, para auxiliar estudantes no ensino e aprendizagem de conteúdos matemáticos. Este 
ambiente se propõe a servir como ferramenta para estudantes que buscam estudar fora do ambiente escolar e em seu próprio ritmo. O ambiente também dará suporte ao ensino e aprendizagem em sala de 
aula, auxiliando professores com informações relevantes sobre o andamento do aprendizado de cada um de seus alunos, além das dificuldades que os mesmos apresentam.

A ideia deste AVA surgiu quando professores das disciplinas de matemática da Universidade Federal do Cear\'a observaram um grande número de reprovações e desistências em suas turmas. Segundo os professores, um dos fatores que pode ser o causador são as deficiências de formação em matemática desde o ensino médio, e, por isso, faltaria aos alunos a base para compreender os novos assuntos. Essa suspeita é reforçada por um estudo realizado pela Organização Para a Cooperação e Desenvolvimento Econômico (OCDE)\cite{pisainfocus2016}. O estudo revelou que 67.1\% dos alunos brasileiros ainda estão abaixo do nível 2 (os níveis são de 1 a 6) deixando o país em 58º lugar na escala do PISA\footnote{O Programa Internacional de Avaliação de Estudantes (Pisa) é uma iniciativa de avaliação comparada, aplicada a estudantes na faixa dos 15 anos, idade em que se pressupõe o término da escolaridade básica obrigatória na maioria dos países e que visa melhorar as políticas e resultados educacionais.}, sendo que somente 0,8\% dos estudantes brasileiros alcançaram os últimos patamares.

\subsection{Objetivos do Trabalho}

Este trabalho objetiva apresentar o projeto de um AVA para auxiliar no processo de ensino e aprendizagem de matemática por alunos dentro e fora da sala de aula, buscando contribuir com o 
processo de ensino e aprendizagem de matem\'atica. 

Como objetivos específicos para este trabalho, temos: 
\begin{alineas}
	\item Levantar os requisitos do sistema.
	\item Desenvolver os módulos com funções de administração do sistema.
    \item Desenvolver os módulos com funções que serão utilizadas pelos estudantes aplicando técnicas de Gamificação.
    \item Verificar a influência e impactos provocados pelo AVA no aprendizado de alunos por meio de exames aplicados a cada estudante antes e após o uso do sistema na Universidade Federal do Ceará.
\end{alineas}


\subsection{Divisão do Trabalho}

Fundamentando-se na problemática mencionada e tendo em vista o objeto de estudo, dividimos esta monografia em quatro capítulos. No capítulo inicial fizemos uma apresentação do tema e os objetivos do 
trabalho. No Capítulo 2, destacaremos os aspectos teóricos sobre ensino e aprendizagem, assim como as tradicionais metodologias de ensino e as apoiadas por computador. Abordaremos tamb\'em conceitos 
de gamificação e os trabalhos que servem de refer\^encia para os conceitos e id\'eias utilizadas no trabalho aqui desenvolvido. No Capítulo 3, discutiremos a concepção, construção e modelagem do 
sistema, apresentando o que o mesmo deve possuir e por que, e no Capítulo 4, apresentaremos os resultados preliminares.