%%%%%%%%%%%%%%%%%%%%%%%%%%%%%%%%%%%%%%%%%%%%%%%%%%%%%%%
%%      Para começar a usar este template, primeiro, %%
%% você dever criar uma conta no ShareLates. Depois, %%
%% vá nasopções no canto esquerdo superior da tela e %%
%% clique em "Copiar Projeto". Dê um novo nome para  %%
%% o projeto. This work has the LPPL maintenance     %%
%% status `maintained' The Currentt Maintainer of    %%
%% this work are:                                    %%
%%                                                   %%
%%        Ednardo Moreira Rodrigues (UFC/DEE)        %%
%%                      and                          %%
%%        Alan Batista de Oliveira (UFC/DEE)         %%
%%                                                   %%
%% Review:                                           %%
%%                                                   %%
%% - Eliene Maria Vieira de Moura;                   %%
%% - Francisco Edvander Pires Santos;                %%  
%% - Izabel Lima dos Santos;                         %%
%% - Juliana Soares Lima;                            %%
%% - Kalline Yasmin Soares Feitosa.                  %%
%%                                                   %%
%% This work may be distributed and/or modified under%%
%% theconditions of the LaTeX Project Public License,%%
%% either version 1.3 of this license or (at your    %%
%% option) any                                       %%
%% later version. The latest version of this license %%
%% is in http://www.latex-project.org/lppl.txt and   %%
%% version 1.3 or later is part of all distributions %%
%% of LaTeX version 2005/12/01 or later.             %%
%% The First Maintainer of this work was:            %%
%% Thiago Nascimento  (UECE)                         %%
%% Project available on:                             %%
%% https://github.com/thiagodnf/uecetex2             %%
%% Further information about abnTeX2                 %%
%% are available on http://abntex2.googlecode.com/   %%
%%%%%%%%%%%%%%%%%%%%%%%%%%%%%%%%%%%%%%%%%%%%%%%%%%%%%%%

\documentclass[        
    a4paper,          % Tamanho da folha A4
    12pt,             % Tamanho da fonte 12pt
    chapter=TITLE,    % Todos os capitulos devem ter caixa alta
    section=Title,    % Todas as secoes devem ter caixa alta somente na primeira letra
    subsection=Title, % Todas as subsecoes devem ter caixa alta somente na primeira letra
    oneside,          % Usada para impressao em apenas uma face do papel
    english,          % Hifenizacoes em ingles
    spanish,          % Hifenizacoes em espanhol
    brazil,           % Ultimo idioma eh o idioma padrao do documento
    fleqn             % Coloca as equações alinhadas a esquerda
]{abntex2}

\input{lib/preambulo}

%%%%%%%%%%%%%%%%%%%%%%%%%%%%%%%%%%%%%%%%%%%%%%%%%%%%%
%%          Configuracoes do ufctex                %%
%%%%%%%%%%%%%%%%%%%%%%%%%%%%%%%%%%%%%%%%%%%%%%%%%%%%%

% Opcoes disponiveis

\trabalhoacademico{tccgraduacao}
%\trabalhoacademico{tccespecializacao}
%\trabalhoacademico{dissertacao}
%\trabalhoacademico{tese}

% Define se o trabalho eh uma qualificacao
% Coloque 'nao' para versao final do trabalho

\ehqualificacao{nao}

% Remove as bordas vermelhas e verdes do PDF gerado
% Coloque 'sim' pare remover

\removerbordasdohyperlink{sim} 

% Adiciona a cor Azul a todos os hyperlinks

\cordohyperlink{nao}

%%%%%%%%%%%%%%%%%%%%%%%%%%%%%%%%%%%%%%%%%%%%%%%%%%%%%
%%          Informação sobre a IES                 %%
%%%%%%%%%%%%%%%%%%%%%%%%%%%%%%%%%%%%%%%%%%%%%%%%%%%%%

\ies{Universidade Federal do Ceará}
\iessigla{UFC}
\centro{Centro de Xxxxxxxx}

%%%%%%%%%%%%%%%%%%%%%%%%%%%%%%%%%%%%%%%%%%%%%%%%%%%%%
%%        Informação para TCC de Graduacao %%
%%%%%%%%%%%%%%%%%%%%%%%%%%%%%%%%%%%%%%%%%%%%%%%%%%%%%

\graduacaoem{Engenharia Xxxxxxx}
\habilitacao{bacharel} % Pode colocar tambem 'licenciada'

%%%%%%%%%%%%%%%%%%%%%%%%%%%%%%%%%%%%%%%%%%%%%%%%%%%%%
%%     Informação para TCC de Especializacao       %%
%%%%%%%%%%%%%%%%%%%%%%%%%%%%%%%%%%%%%%%%%%%%%%%%%%%%%

\especializacaoem{Descargas Atmosféricas}

%%%%%%%%%%%%%%%%%%%%%%%%%%%%%%%%%%%%%%%%%%%%%%%%%%%%%
%%         Informação para Dissertacao             %%
%%%%%%%%%%%%%%%%%%%%%%%%%%%%%%%%%%%%%%%%%%%%%%%%%%%%%

\programamestrado{Programa de Pós-Graduação em Xxxxxxx}
\nomedomestrado{Mestrado Acadêmico em Xxxxxxx}
\mestreem{Engenharia Xxxxxx}
\areadeconcentracaomestrado{Engenharia Xxxxxx}

%%%%%%%%%%%%%%%%%%%%%%%%%%%%%%%%%%%%%%%%%%%%%%%%%%%%%
%%               Informação para Tese              %%
%%%%%%%%%%%%%%%%%%%%%%%%%%%%%%%%%%%%%%%%%%%%%%%%%%%%%

\programadoutorado{Programa de Pós-Graduação em Xxxxxx}
\nomedodoutorado{Doutorado em Xxxxxxx}
\doutorem{Engenharia Xxxxxx}
\areadeconcentracaodoutorado{Engenharia Xxxxxxx}

%%%%%%%%%%%%%%%%%%%%%%%%%%%%%%%%%%%%%%%%%%%%%%
%%  Informacoes relacionadas ao trabalho     %%
%%%%%%%%%%%%%%%%%%%%%%%%%%%%%%%%%%%%%%%%%%%%%%

\autor{Nome Sobrenome}
\titulo{Título do Trabalho}
\data{2016}
\local{Fortaleza}

% Exemplo: \dataaprovacao{01 de Janeiro de 2012}
\dataaprovacao{}

%%%%%%%%%%%%%%%%%%%%%%%%%%%%%%%%%%%%%%%%%%%%%
%%     Informação sobre o Orientador       %%
%%%%%%%%%%%%%%%%%%%%%%%%%%%%%%%%%%%%%%%%%%%%%

\orientador{Prof. Dr. Xxxxxxx Xxxxxx Xxxxxxx}
\orientadories{Universidade Federal do Ceará (UFC)}
\orientadorcentro{Centro de Ciências e Tecnologia (CCT)}
\orientadorfeminino{nao} % Coloque 'sim' se for do sexo feminino

%%%%%%%%%%%%%%%%%%%%%%%%%%%%%%%%%%%%%%%%%%%%%
%%      Informação sobre o Co-orientador   %%
%%%%%%%%%%%%%%%%%%%%%%%%%%%%%%%%%%%%%%%%%%%%%

% Deixe o nome do coorientador em branco para remover do documento

\coorientador{}
\coorientadories{Universidade Co-orientador (SIGLA)}
\coorientadorcentro{Centro do Co-orientador (SIGLA)}
\coorientadorfeminino{nao} % Coloque 'sim' se for do sexo feminino

%%%%%%%%%%%%%%%%%%%%%%%%%%%%%%%%%%%%%%%%%%%%%
%%      Informação sobre a banca           %%
%%%%%%%%%%%%%%%%%%%%%%%%%%%%%%%%%%%%%%%%%%%%%

% Atenção! Deixe o nome do membro da banca para remover da folha de aprovacao

% Exemplo de uso:
% \membrodabancadois{Prof. Dr. Fulano de Tal}
% \membrodabancadoisies{Universidade Federal do Ceará - UFC}

\membrodabancadois{Prof. Dr. Xxxxxxx Xxxxxx Xxxxxxx}
\membrodabancadoiscentro{Faculdade de Filosofia Dom Aureliano Matos (FAFIDAM)}
\membrodabancadoisies{Universidade do Membro da Banca Dois (SIGLA)}
\membrodabancatres{Prof. Dr. Xxxxxxx Xxxxxx Xxxxxxx}
\membrodabancatrescentro{Centro de Ciências e Tecnologia (CCT)}
\membrodabancatresies{Universidade do Membro da Banca Três (SIGLA)}
\membrodabancaquatro{Prof. Dr. Xxxxxxx Xxxxxx Xxxxxxx}
\membrodabancaquatrocentro{Centro de Ciências e Tecnologia (CCT)}
\membrodabancaquatroies{Universidade do Membro da Banca Quatro (SIGLA)}
\membrodabancacinco{Prof. Dr. Xxxxxxx Xxxxxx Xxxxxxx}
\membrodabancacincocentro{Teste}
\membrodabancacincoies{Universidade do Membro da Banca Cinco (SIGLA)}
\membrodabancaseis{Prof. Dr. Xxxxxxx Xxxxxx Xxxxxxx}
\membrodabancaseiscentro{}
\membrodabancaseisies{Universidade do Membro da Banca Seis (SIGLA)}

\begin{document}	

	% Elementos pré-textuais
	\imprimircapa
	\imprimirfolhaderosto{}
	\imprimirfichacatalografica{elementos-pre-textuais/ficha-catalografica}
	%\imprimirerrata{elementos-pre-textuais/errata}
	\imprimirfolhadeaprovacao
	\imprimirdedicatoria{elementos-pre-textuais/dedicatoria}
	\imprimiragradecimentos{elementos-pre-textuais/agradecimentos}
	\imprimirepigrafe{elementos-pre-textuais/epigrafe}
	\imprimirresumo{elementos-pre-textuais/resumo}
	\imprimirabstract{elementos-pre-textuais/abstract}
	\imprimirlistadeilustracoes
	\imprimirlistadetabelas
	%\imprimirlistadequadros
	%\imprimirlistadealgoritmos
	%\imprimirlistadecodigosfonte
	\imprimirlistadeabreviaturasesiglas
	\imprimirlistadesimbolos{elementos-pre-textuais/lista-de-simbolos}   
	\imprimirsumario
	
	%Elementos textuais
	\textual
	\section{INTRODUÇÃO}

A Matemática, como ciência, sempre teve uma relação muito especial com as novas tecnologias, desde as calculadoras, os computadores, aos sistemas multimédia e a Internet. No entanto, os professores costumam demorar a perceber como tirar partido destas tecnologias como ferramentas de trabalho. \cite{da1997ensino}. 

A medida que a quantidade de recursos tecnológicos na sala de aula foram aumentando, tornou-se necessário a criação de novas metodologias de ensino, especificamente na Educação Matemática. Tal busca procura fazer da Matemática uma disciplina atraente, desvinculada do ensino tradicional que já se mostrou ineficiente \cite{silva2009ambiente}.

Tendo em vista essa necessidade, o presente trabalho, busca apresentar o projeto e desenvolvimento de um Ambiente Virtual de Aprendizagem (AVA) \cite{valentini2010aprendizagem}, para auxiliar estudantes no ensino e aprendizagem de conteúdos matemáticos. Este ambiente se propõe a servir como ferramenta para estudantes que buscam estudar fora do ambiente escolar e de uma forma auto-ritmada, ou seja, no seu próprio ritmo. Entretanto, o ambiente também dará suporte ao ensino e aprendizagem dentro da sala de aula, auxiliando professores com informações relevantes sobre o andamento do aprendizado de cada um de seus aluno, além das dificuldades que os mesmos apresentam.

Grande parte dos alunos tem dificuldades em aprender matemática, e muitas vezes essas dificuldades ocorrem não pela falta de atenção ou por não gostar do conteúdo, mas por fatores mentais ou psicológicos que envolvem uma série de trabalhos
e conceitos que precisam ser desenvolvidos \cite{sa2015software}. Mas como auxiliar alunos com dificuldades na aprendizagem da matemática?

Em busca dessa resposta, diferentes sistemas de \textit{softwares} foram desenvolvidos buscando servir a educação. Em 2006, Salman Khan funda a Khan Academy, uma organização educacional que tem por objetivo oferecer exercícios, vídeos de instrução e um painel de aprendizado personalizado que habilita os estudantes a aprender no seu próprio ritmo dentro e fora da sala de aula \cite{khan2012one}. A plataforma criado por khan utiliza de vídeo-aulas e resolução de problemas para ensinar seus alunos, permitindo assim, que cada um tenha uma aprendizagem de forma independente e auto-ritmada.

Um outro trabalho também  importante nessa área, podemos citar o de  \citeonline{melis2001activemath}, onde os autores desenvolvem um AVA que permite os alunos desfrutarem da experiência de estudar num curso gerado dinamicamente a partir do estado em que o mesmo se encontra dentro do sistema. Estes e outros trabalhos são apresentados com mais detalhes em trabalhos relacionados.

\subsection{Objetivos do Trabalho}

Este trabalho visa contribuir com o processo de ensino e aprendizagem de matemática, buscando projetar e desenvolver um ambiente virtual que auxilie no processo de ensino e aprendizagem de matemática por alunos, dentro e fora da sala de aula. 

Como objetivos específicos para este trabalho, temos: 
\begin{alineascomponto}
    \item Elaborar um método para diagnosticar, nos alunos, dificuldades existentes em certos conteúdos de matemática.
	\item Desenvolver o AVA, aplicando técnicas de gamificação.
    \item Aplicar o AVA numa turma de graduação onde os alunos estejam cursando disciplinas de matemática.
    \item Analisar a influência e impactos provocados pelo AVA, por meio de dados gerados pela ferramenta ao longo de sua utilização e questionários aplicados aos alunos que utilizarem o AVA.

\end{alineascomponto}


\subsection{Divisão do Trabalho}

Fundamentando-se na problemática mencionada, e tendo em vista o objeto de estudo, esta dissertação foi dividida em cinco capítulos. No capítulo inicial é feita uma apresentação do tema a ser discutido durante toda esta dissertação e o objetivo do trabalho.

No Capítulo 2, destacam-se os aspectos teóricos sobre ensino e aprendizagem, assim como as tradicionais metodologias de ensino e as apoiadas por computador, além de conceitos sobre gamificação e os trabalhos que serviram de referencial para abordar os conceitos e ideias utilizadas no trabalho aqui desenvolvido.

No Capítulo 3 compreende-se a concepção, construção e modelagem do sistema, apresentando o que o mesmo deve possuir e por que.

No Capítulo 4 são apresentados os resultados preliminares.
	\chapter{Título do segundo capítulo}
\label{cap:fundamentacao-teorica}

    Alguns autores preferem fazer uma "fundamentação teórica" ~no segundo capítulo, outros, preferem fazer uma "revisão da literatura". Entretanto, isto é particular de cada trabalho e o autor deve escolher o título mais adequado. Consultar o orientador é importante para determinar o título apropriado.

    Evite começar da seção secundária, ou seja, não passe direto do título do capítulo para o título da seção secundária. Escreva um texto para introduzir as seções subsequentes. Lembre-se de utilizar primeira letra maiúscula quando estiver se referindo a um objeto com numeração específica como capítulo, seção, subseção, figura, tabela, quadro, equação, normalmente, se escreve a primeira letra maiúscula da palavra do objeto seguido do \textit{label}. Por exemplo, a Seção \ref{sec:citacoes} explica como fazer citações bibliográficas. Observe no código fonte deste texto como foi feita a referência cruzada. Isso permite enumerar a seção do modo automático o que facilita caso novas seções sejam criadas.  

    \section{Citações bibliográficas}
    \label{sec:citacoes}

        Esta frase mostra como citar um livro sobre descargas atmosféricas \cite{rakov2003lightning}. Também podem ser citados sites como \citeonline{elat2015densidade}.

        Referenciando outro site \cite{secretaria1999}. Texto texto texto texto texto texto texto texto texto texto texto texto texto texto texto texto texto texto texto. Texto texto texto texto texto texto texto texto texto texto texto texto texto texto texto texto texto texto texto. Texto texto texto texto texto texto texto texto texto texto texto texto texto texto texto texto texto texto texto. Texto texto texto texto texto texto texto texto texto texto texto texto texto texto texto texto texto texto texto. Citando uma norma \cite{NBR10520:1988}.
        
        Citação de duas referências que concordam entre si \cite{lamport1986latex,Maia2011}. Texto texto texto texto texto texto texto texto texto texto texto texto texto texto texto texto texto texto texto. Texto texto texto texto texto texto texto texto texto texto texto texto texto texto texto texto texto texto texto. Texto texto texto texto texto texto texto texto texto texto texto texto texto texto texto texto texto texto texto. Texto texto texto texto texto texto texto texto texto texto texto texto texto texto texto texto texto texto texto texto texto texto texto texto texto texto. Citando um manual \cite{manuais1989}. 
        
        Outro tipo de citação é a citação literal ou direta com mais de três linhas. Este tipo de citação deve ser destacada com recuo de $4~cm$ da margem esquerda com letra menor (tamanho 10), sem aspas e com espaçamento simples.  Para exemplificar esse tipo de citação, considere a afirmação de \citeonline{feitosa2016}:
        
        \begin{citacao}
            A cultura é o processo através do qual o homem cria o algo onde antes imperava o
            nada. Esse algo é toda complexidade de criações simbólicas, de sentidos e significados que
            damos às coisas e ao mundo. Um “algo” que não se sustenta se não se entender os processos
            culturais como mecanismos de mediação entre nós e os fenômenos. Assim, mais do que
            apenas um elemento da comunicação, a mediação é, por excelência, cultural. As diversas
            modalidades de mediação são apenas sotaques diferenciados dessa mediação cultural. Assim
            é a mediação informacional.
        \end{citacao}
        
        A afirmação do parágrafo anterior também pode ser reproduzida com a citação na final como mostra o exemplo a seguir: 
        
        \begin{citacao}
            A cultura é o processo através do qual o homem cria o algo onde antes imperava o
            nada. Esse algo é toda complexidade de criações simbólicas, de sentidos e significados que
            damos às coisas e ao mundo. Um “algo” que não se sustenta se não se entender os processos
            culturais como mecanismos de mediação entre nós e os fenômenos. Assim, mais do que
            apenas um elemento da comunicação, a mediação é, por excelência, cultural. As diversas
            modalidades de mediação são apenas sotaques diferenciados dessa mediação cultural. Assim
            é a mediação informacional \cite{feitosa2016}.
        \end{citacao}
        
        %Mais exemplos e opções de citações podem ser encontradas em:
        
%        https://en.wikibooks.org/wiki/LaTeX/Bibliography_Management
%        https://github.com/cfgnunes/latex-cefetmg/blob/master/latex-cefetmg/03-elementos-pos-textuais/apendices.tex        
        
        

    \section{Inserindo figuras}
    \label{sec:figuras}
    
    A Figura \ref{fig:reitoria} apresenta a fotografia da reitoria da Universidade Federal do Ceará. Observe a estrutura do código para inserir a Figura \ref{fig:reitoria}. No título, apenas a primeira letra da frase é maiúscula  exceto nomes próprios e não há ponto final. Procure ajustar o tamanho da figura para preencher a largura delimitada pelas margens esquerda e direita. Não esqueça de indicar fonte da figura. 
    
   	\begin{figure}[h!]
		\Caption{\label{fig:reitoria} Fotografia da Reitoria da Universidade Federal do Ceará}
		%\centering
		\UFCfig{}{
			\fbox{\includegraphics[width=12cm]{figuras/exemplo-1}}
		}{
			\Fonte{\citeonline{UFC2012}.}
		}	
	\end{figure}
	
    Texto texto texto texto texto texto texto texto texto texto texto texto texto texto texto texto texto texto texto texto texto texto texto texto texto texto texto texto texto texto texto texto texto texto texto texto texto texto texto texto texto texto texto texto texto.

    Texto texto texto texto texto texto texto texto texto texto texto texto texto texto texto texto texto texto texto. Texto texto texto texto texto texto texto texto texto texto texto texto texto texto texto texto texto texto texto.

    A Figura \ref{fig:sondas} Texto texto texto texto texto texto texto texto texto texto texto texto texto texto texto texto texto texto texto. Texto texto texto texto texto texto texto texto texto texto texto texto texto texto texto texto texto texto texto.

	\begin{figure}[h!]
		\centering
		\Caption{\label{fig:sondas} Gráfico da Atmosfera Superior}	
		\UFCfig{}{
			\fbox{\includegraphics[width=12cm]{figuras/sondas}}
		}{
			\Fonte{adaptado de \citeonline{NASA2016}.}}	
	\end{figure}

    Texto texto texto texto texto texto texto texto texto texto texto texto texto texto texto texto texto texto texto texto texto texto texto texto texto texto texto texto texto texto texto texto texto texto texto texto texto texto texto texto texto texto texto texto texto.

    Texto texto texto texto texto texto texto texto texto texto texto texto texto texto texto texto texto texto texto texto texto texto texto texto texto texto texto texto texto texto texto texto texto texto texto texto texto texto texto texto texto texto texto texto texto.

    Texto texto texto texto texto texto texto texto texto texto texto texto texto texto texto texto texto texto texto texto texto texto texto texto texto texto texto texto texto texto texto texto texto texto texto texto texto texto texto texto texto texto texto texto texto.

    Texto texto texto texto texto texto texto texto texto texto texto texto texto texto texto texto texto texto texto texto texto texto texto texto texto texto texto texto texto texto texto texto texto texto texto texto texto texto texto texto texto texto texto texto texto.

    Texto texto texto texto texto texto texto texto texto texto texto texto texto texto texto texto texto texto texto texto texto texto texto texto texto texto texto texto texto texto texto texto texto texto texto texto texto texto texto texto texto texto texto texto texto.

    \section{Inserindo tabelas}
    \label{sec:tabelas}

	A Tabela \ref{tab:exemplo-1} Texto texto texto texto texto texto texto texto texto texto texto texto texto texto texto texto texto texto texto. Texto texto texto texto texto texto texto texto texto texto texto texto texto texto texto texto texto texto texto.
		
	\begin{table}[h!]	
		\centering
		\Caption{\label{tab:exemplo-1} Exemplo de tabela}	
		\UFCtab{}{
			\begin{tabular}{cll}
				\toprule
				Ranking & Exon Coverage & Splice Site Support \\
				\midrule \midrule
				E1 & Complete coverage by a single transcript & Both splice sites\\
				E2 & Complete coverage by more than a single transcript & Both splice sites\\
				E3 & Partial coverage & Both splice sites\\
				E4 & Partial coverage & One splice site\\
				E5 & Complete or partial coverage & No splice sites\\
				E6 & No coverage & No splice sites\\
				\bottomrule
			\end{tabular}
		}{
		\Fonte{elaborado pelo autor.}
	}
	\end{table}

\subsection{Exemplo de subseção} \label{sec:ex_sec}
	
Texto texto texto texto texto texto texto texto texto texto texto texto texto texto texto texto texto texto texto texto texto texto texto texto texto texto texto texto texto texto texto texto texto texto texto texto texto texto texto texto texto texto texto texto texto.

%acrlong{DATASUS},\acrlong{DNV},\acrlong{DO},\acrlong{ESF},\acrlong{IBGE},\acrlong{MFC},\acrlong{MI},\acrlong{MS},\acrlong{NV},\acrlong{ODM},\acrlong{OI},\acrlong{OMS},\acrlong{ONU},\acrlong{PNI},\acrlong{PSF},\acrlong{RIPSA},\acrlong{RN},\acrlong{SIM},\acrlong{SINASC},\acrlong{SUS},\acrlong{TMI},\acrlong{TMMFC}


\begin{alineascomponto}
	\item Integer non lacinia magna. Aenean tempor lorem tellus, non sodales nisl commodo ut
	\item Proin mattis placerat risus sit amet laoreet. Praesent sapien arcu, maximus ac fringilla efficitur, vulputate faucibus sem. Donec aliquet velit eros, sit amet elementum dolor pharetra eget
	\item Integer eget mattis libero. Praesent ex velit, pulvinar at massa vel, fermentum dictum mauris. Ut feugiat accumsan augue, et ultrices ipsum euismod vitae
	\begin{subalineascomponto}
		\item Integer non lacinia magna. Aenean tempor lorem tellus, non sodales nisl commodo ut
		\item Proin mattis placerat risus sit amet laoreet.
	\end{subalineascomponto}
\end{alineascomponto}

Teste de siglas \gls{TMMFC}, \gls{DA}, \gls{MCEG}
	\chapter{Metodologia}
\label{chap:metodologia}

Texto texto texto texto texto texto texto texto texto texto texto texto texto texto texto texto texto texto texto texto texto texto texto texto texto texto texto texto texto texto texto texto texto texto texto texto texto texto texto texto texto texto texto texto texto texto texto texto texto texto texto texto texto texto texto texto texto texto texto texto texto texto texto texto texto texto texto texto texto.

\begin{table}[h!]
	\Caption{\label{tabela-ibge} Um Exemplo de tabela alinhada que pode ser longa ou curta}%
	\IBGEtab{}{%
		\begin{tabular}{ccc}
			\toprule
			Nome & Nascimento & Documento \\
			\midrule \midrule
			Maria da Silva & 11/11/1111 & 111.111.111-11 \\
			Maria da Silva & 11/11/1111 & 111.111.111-11 \\
			Maria da Silva & 11/11/1111 & 111.111.111-11 \\
			\bottomrule
		\end{tabular}%
	}{%
	\Fonte{o autor.}%
	\Nota{esta é uma nota, que diz que os dados são baseados na
		regressão linear.}%
	\Nota[Anotações]{uma anotação adicional, seguida de várias outras.}%
}
\end{table}

Texto texto texto texto texto texto texto texto texto texto texto texto texto texto texto texto texto texto texto texto texto texto texto texto texto texto texto texto texto texto texto texto texto texto texto texto texto texto texto texto texto texto texto texto texto texto texto texto texto texto texto texto texto texto texto texto texto texto texto texto texto texto texto texto texto texto texto texto texto.


\section{Exemplo de Algoritmos e Figuras}
\label{sec:exemplo-de-algoritmos-e-figuras}

Texto texto texto texto texto texto texto texto texto texto texto texto texto texto texto texto texto texto texto texto texto texto texto texto texto texto texto texto texto texto texto texto texto texto texto texto texto texto texto texto texto texto texto texto texto texto texto texto texto texto texto texto texto texto texto texto texto texto texto texto texto texto texto texto texto texto texto texto texto.
%\begin{algorithm}[h!]
%	\SetSpacedAlgorithm
%	\caption{\label{exemplo-de-algoritmo}Como escrever algoritmos no \LaTeX2e}
%	\Entrada{o proprio texto}
%	\Saida{como escrever algoritmos com  Latex:}% \LaTeX2e }
%	\Inicio{
%		inicialização;
%		\Repita{fim do texto}{
%			leia o atual;
%			\Se{entendeu}{
%				vá para o proximo\;
%				próximo se torna o atual;}
%			\Senao{volte ao início da seção;}
%		}
%	}	
%\end{algorithm}

Texto texto texto texto texto texto texto texto texto texto texto.

%\begin{algorithm}[H]
%	\Entrada{o proprio texto}
%	\Saida{como escrever algoritmos com \LaTeX2e }
%	\Inicio{
%		inicialização\;
%		\Repita{fim do texto}{
%			leia o atual\;
%			\Se{entendeu}{
%				vá para o próximo\;
%				próximo se torna o atual\;}
%			\Senao{volte ao início da seção\;}
%		}
%	}
%	\caption{Exemplo de Algoritmo Versao 02}
%\end{algorithm}

%\begin{algorithm}
%	\begin{algorithmic}
%	\Entrada{o proprio texto}
%	\Saida{como escrever algoritmos com \LaTeX2e }	
%	\end{algorithmic}
%\end{algorithm}

Exemplo de alíneas com números:

\begin{alineascomnumero}
	\item Texto texto texto texto texto texto texto texto texto texto texto texto .
	\item Texto texto texto texto texto texto texto texto texto texto texto texto .
	\item Texto texto texto texto texto texto texto texto texto texto texto texto .
	\item Texto texto texto texto texto texto texto texto texto texto texto texto .
	\item Texto texto texto texto texto texto texto texto texto texto texto texto .
	\item Texto texto texto texto texto texto texto texto texto texto texto texto .
\end{alineascomnumero}

Texto texto texto texto texto texto texto texto texto texto texto texto texto texto texto texto texto texto texto texto texto texto texto texto texto texto texto texto texto texto texto texto texto texto texto texto texto texto texto texto texto texto texto texto texto texto texto texto texto texto texto texto texto texto texto texto texto texto texto texto texto texto texto texto texto texto texto texto texto.

 


Ou então figuras podem ser incorporadas de arquivos externos, como é o caso da \autoref{fig-grafico-1}. Se a figura que ser incluída se tratar de um diagrama, um gráfico ou uma ilustração que você mesmo produza, priorize o uso de imagens vetoriais no formato PDF. Com isso, o tamanho do arquivo final do trabalho será menor, e as imagens terão uma apresentação melhor, principalmente quando impressas, uma vez que imagens vetorias são perfeitamente escaláveis para qualquer dimensão. Nesse caso, se for utilizar o Microsoft Excel para produzir gráficos, ou o Microsoft Word para produzir ilustrações, exporte-os como PDF e os incorpore ao documento conforme o exemplo abaixo. No entanto, para manter a coerência no uso de software livre (já que você está usando LaTeX e abnTeX),  teste a ferramenta InkScape\index{InkScape}. ao CorelDraw\index{CorelDraw} ou ao Adobe Illustrator\index{Adobe! Illustrator}.  De todo modo, caso não seja possível  utilizar arquivos de imagens como PDF, utilize qualquer outro formato, como JPEG, GIF, BMP, etc.  Nesse caso, você pode tentar aprimorar as imagens incorporadas com o software livre \index{Gimp}Gimp. Ele é uma alternativa livre ao Adobe Photoshop\index{Adobe! Photoshop}.

\section{Usando Fórmulas Matemáticas}

A seguir, exemplos de inserção de fórmulas:
Texto texto texto texto texto texto texto texto texto texto texto texto texto texto texto texto texto texto texto texto texto texto texto texto texto texto texto texto texto texto texto texto texto texto texto texto texto texto texto texto texto texto texto texto texto texto texto texto texto texto texto texto texto texto texto texto texto texto texto texto texto texto texto texto texto texto texto texto texto.

	\begin{equation}
		\begin{aligned}
			x = a_0 + \cfrac{1}{a_1
				+ \cfrac{1}{a_2
					+ \cfrac{1}{a_3 + \cfrac{1}{a_4} } } }
		\end{aligned}
	\end{equation}
	
Texto texto texto texto texto texto texto texto texto texto texto texto texto texto texto texto texto texto texto texto texto texto texto texto texto texto texto texto texto texto texto texto texto texto texto texto texto texto texto texto texto texto texto texto texto texto texto texto texto texto texto texto texto texto texto texto texto texto texto texto texto texto texto texto texto texto texto texto texto.

	\begin{equation}
		\begin{aligned}
			k_{n+1} = n^2 + k_n^2 - k_{n-1}
		\end{aligned}
	\end{equation}
Texto texto texto texto texto texto texto texto texto texto texto texto texto texto texto texto texto texto texto texto texto texto texto texto texto texto texto texto texto texto texto texto texto texto texto texto texto texto texto texto texto texto texto texto texto texto texto texto texto texto texto texto texto texto texto texto texto texto texto texto texto texto texto texto texto texto texto texto texto.

	\begin{equation}
		\begin{aligned}
			\cos (2\theta) = \cos^2 \theta - \sin^2 \theta
		\end{aligned}
	\end{equation}
	
Texto texto texto texto texto texto texto texto texto texto texto texto texto texto texto texto texto texto texto texto texto texto texto texto texto texto texto texto texto texto texto texto texto texto texto texto texto texto texto texto texto texto texto texto texto texto texto texto texto texto texto texto texto texto texto texto texto texto texto texto texto texto texto texto texto texto texto texto texto.

	\begin{equation}
		\begin{aligned}
			A_{m,n} =
			\begin{pmatrix}
			a_{1,1} & a_{1,2} & \cdots & a_{1,n} \\
			a_{2,1} & a_{2,2} & \cdots & a_{2,n} \\
			\vdots  & \vdots  & \ddots & \vdots  \\
			a_{m,1} & a_{m,2} & \cdots & a_{m,n}
			\end{pmatrix}
		\end{aligned}
	\end{equation}

Texto texto texto texto texto texto texto texto texto texto texto texto texto texto texto texto texto texto texto texto texto texto texto texto texto texto texto texto texto texto texto texto texto texto texto texto texto texto texto texto texto texto texto texto texto texto texto texto texto texto texto texto texto texto texto texto texto texto texto texto texto texto texto texto texto texto texto texto texto.
	\begin{equation}
		\begin{aligned}
			f(n) = \left\{ 
			\begin{array}{l l}
			n/2 & \quad \text{if $n$ is even}\\
			-(n+1)/2 & \quad \text{if $n$ is odd}
			\end{array} \right.
		\end{aligned}
	\end{equation}
Texto texto texto texto texto texto texto texto texto texto texto texto texto texto texto texto texto texto texto texto texto texto texto texto texto texto texto texto texto texto texto texto texto texto texto texto texto texto texto texto texto texto texto texto texto texto texto texto texto texto texto texto texto texto texto texto texto texto texto texto texto texto texto texto texto texto texto texto texto.

\section{Usando Algoritmos}

 Texto texto texto texto texto texto texto texto texto texto texto texto texto texto texto texto texto texto texto texto texto texto texto texto texto texto texto texto texto texto texto texto texto texto texto texto texto texto texto texto texto texto texto texto texto texto texto texto texto texto texto texto texto texto texto texto texto texto texto texto texto texto texto texto texto texto texto texto texto.

%\begin{algorithm}[h!]
%	\SetSpacedAlgorithm
%	\caption{\label{alg:algoritmo_de_colonica_de_formigas}Algoritmo de Otimização por Colônia de Formiga}
%	\Entrada{Entrada do Algoritmo}
%	\Saida{Saida do Algoritmo}
%	\Inicio{
%		Atribua os valores dos parâmetros\;
%		Inicialize as trilhas de feromônios\;
%		\Enqto{não atingir o critério de parada}{
%			\Para{cada formiga}{
%				Construa as Soluções\;
%			}
%			Aplique Busca Local (Opcional)\;
%			Atualize o Feromônio\;
%		}	
%	}		
%\end{algorithm}

 

\section{Usando Código-fonte}

 Texto texto texto texto texto texto texto texto texto texto texto texto texto texto texto texto texto texto texto texto texto texto texto texto texto texto texto texto texto texto texto texto texto texto texto texto texto texto texto texto.

\lstinputlisting[language=C++,caption={Hello World em C++}]{figuras/main.cpp}

 Texto texto texto texto texto texto texto texto texto texto texto texto texto texto texto texto texto texto texto texto texto texto texto texto texto texto texto texto texto texto.

\begin{lstlisting}[language=Java,caption={Hello World em Java}]
public class HelloWorld {
	public static void main(String[] args) {
		System.out.println("Hello World!");
	}
}
\end{lstlisting}

 Texto texto texto texto texto texto texto texto texto texto texto texto texto texto texto texto texto texto.

\section{Usando Teoremas, Proposições, etc}

 Texto texto texto texto texto texto texto texto texto texto texto texto texto texto texto texto texto texto texto texto texto texto texto texto texto.

\begin{teo}[Pitágoras]
	Em todo triângulo retângulo o quadrado do comprimento da
	hipotenusa é igual a soma dos quadrados dos comprimentos dos catetos.
\end{teo}


Texto texto texto texto texto texto texto texto texto texto texto texto texto texto texto.

\begin{teo}[Fermat]
	Não existem inteiros $n > 2$, e $x, y, z$ tais que $x^n + y^n = z$
\end{teo}

Texto texto texto texto texto texto texto texto texto texto texto texto texto texto texto.

\begin{prop}
	Para demonstrar o Teorema de Pitágoras...
\end{prop}

Texto texto texto texto texto texto texto texto texto texto texto texto texto texto texto.

\begin{exem}
	Este é um exemplo do uso do ambiente exem definido acima.
\end{exem}

Texto texto texto texto texto texto texto texto texto texto texto texto texto texto texto.


\begin{xdefinicao}
	Definimos o produto de ...
\end{xdefinicao}

Texto texto texto texto texto texto texto texto texto texto texto texto texto texto texto.

\section{Usando Questões}


Texto texto texto texto texto texto texto texto texto texto texto texto texto texto texto.


Exemplo de elaboração de questões:
\begin{questao}
	\item Esta é a primeira questão com alguns itens:
		\begin{enumerate}
			\item Este é o primeiro item
			\item Segundo item
		\end{enumerate}
	\item Esta é a segunda questão:
		\begin{enumerate}
			\item Este é o primeiro item
			\item Segundo item
		\end{enumerate}
	\item Lorem ipsum dolor sit amet, consectetur adipiscing elit. Nunc dictum sed tortor nec viverra. consectetur adipiscing elit. Nunc dictum sed tortor nec viverra.
		\begin{enumerate}
			\item consectetur
			\item adipiscing
			\item Nunc
			\item dictum
		\end{enumerate}
\end{questao}

	\chapter{RESULTADOS}
\label{chap:resultados}

Nesta seção, apresentaremos os resultados obtidos com este trabalho.

Definimos o processo que foi utilizado durante o desenvolvimento do sistema, coletamos, analisamos e validamos os requisitos do sistema e realizamos o projeto do sistema. 

Na implementação, desenvolvemos os módulos: 

\begin{alineascomponto}
    \item Gerenciador de Usuários: módulo responsável por gerenciar os usuários do sistema como professores, assistentes e alunos.
    \item Gerenciador de Turmas: módulo responsável por gerenciar as turmas de alunos do sistema.
    \item Gerenciador de Disciplinas: módulo responsável por gerenciar as disciplinas que serão cadastradas no sistema.
    \item Gerenciador de Lições: módulo responsável por gerenciar as lições que os professores irão cadastrar no sistema.
    \item Gerenciador de Questões: módulo responsável por gerenciar os problemas que os assistentes e professores poderão cadastrar para cada lição.
    \item Gerenciador de Pontuação: módulo responsável por gerenciar a pontuação ganha pelos alunos, assim como seu nível de experiência ao longo da utilização do sistema. 
    \item Fórum: módulo responsável por permitir que alunos postem dúvidas dos mais variados  assuntos relacionadas ao sistema, seja dúvidas em relação ao conteúdo apresentado em sala de aula, assim como informações sobre o sistema e sugestões.
    \item Gerenciador de Progresso: módulo responsável por acompanhar o andamento de cada estudante durante seu aprendizado e identificar os obstáculo epistemológicos enfrentados, indicando ao aluno a exist\^encia desses obst\'aculos e o(s) conte\'udo(s) que ele possui defici\^encia (causador(es) do obstáculo) para ele assim poder pausar o conte\'udo que est\'a estudando e voltar a estudar o(s) conte\'udo(s) que o sistema indicar.
	\item Gerador de Estatísticas: módulo responsável por gerar as estatísticas que o professor utilizará para acompanhar o andamento de suas turmas e alunos, assim como para o uso pelo estudante, 
que utilizará para acompanhar seu próprio progresso durante sua aprendizagem no sistema. 
\end{alineascomponto}

A seguir, apresentaremos algumas telas do sistema:

\begin{figure}[H]
  \centering
  \begin{minipage}[b]{0.49\textwidth}
	\Caption{Tela inicial}
	\UFCfig{}{    
    	\fbox{\includegraphics[width=\textwidth]{figuras/askmath/1}}
    }{
    \Fonte{\url{www.askmath.quixada.ufc.br}}
    }
  \end{minipage}
  \hfill
  \begin{minipage}[b]{0.49\textwidth}
	\Caption{Tela com lições de Proposições}
	\UFCfig{}{    
    	\fbox{\includegraphics[width=\textwidth]{figuras/askmath/2}}
    }{
    \Fonte{\url{www.askmath.quixada.ufc.br}}
    }	
  \end{minipage}

\bigskip
 
  \begin{minipage}[b]{0.49\textwidth}
	\Caption{Tela de problemas do estudante}
	\UFCfig{}{    
    	\fbox{\includegraphics[width=\textwidth]{figuras/askmath/4}}
    }{
    \Fonte{\url{www.askmath.quixada.ufc.br}}
    }
  \end{minipage}
  \hfill
  \begin{minipage}[b]{0.49\textwidth}
	\Caption{Tela de uma postagem no fórum}
	\UFCfig{}{    
    	\fbox{\includegraphics[width=\textwidth]{figuras/askmath/5}}
    }{
    \Fonte{\url{www.askmath.quixada.ufc.br}}
    }
  \end{minipage} 

\bigskip

   \begin{minipage}[b]{0.49\textwidth}
	\Caption{Tela de administra\c{c}\~ao}
	\UFCfig{}{    
    	\fbox{\includegraphics[width=\textwidth]{figuras/askmath/3}}
    }{
    \Fonte{\url{www.askmath.quixada.ufc.br}}
    }
  \end{minipage}
  \hfill
  \begin{minipage}[b]{0.49\textwidth}
	\Caption{Tela de problemas do administrador}
	\UFCfig{}{    
    	\fbox{\includegraphics[width=\textwidth]{figuras/askmath/6}}
    }{
    \Fonte{\url{www.askmath.quixada.ufc.br}}
    }
  \end{minipage}  
\end{figure}
	\chapter{CONCLUSÕES E TRABALHOS FUTUROS}
\label{chap:conclusoes-e-trabalhos-futuros}

Esse capítulo apresenta as considerações finais a cerca desse trabalho. Além disso, é proposta a inclusão de novas funcionalidades que podem ser desenvolvidas em trabalhos futuros.

\section{Conclusões}

Após a implementação da solução proposta por este trabalho, constatou-se que tanto o objetivo geral quanto os específicos foram atendidos. O sistema foi desenvolvido conforme as melhores práticas e padrões em desenvolvimento de \textit{software} disponíveis no momento, tornando o sistema flexível, modularizado e reusável. Além disso, o sistema proporciona de fato um ambiente em que alunos podem práticar os conhecimentos adquiridos e até assimilar novos conhecimento. 

Finalmente, o sistema pode ainda ser utilizado como complemento ao ensino presencial, auxiliando professores e tutores no acompanhamento do aprendizado de seus alunos e aprendizes.

Podem-se evidenciar limitações na solução desenvolvida na questão da comunicação entre aluno-aluno e aluno-profesor, já que isso só ocorre através do fórum de discurções, vale ressaltar que é improvável que, durante a aplicação do sistema, tais limitações sejam atingidas a ponto de comprometer a finalidade do sistema.

\section{Sugestões para Trabalhos Futuros}

No que tange trabalhos futuros que podem ser realizados baseando-se neste projeto, podem ser citados: desenvolver um módulo de bate-papo entre os alunos e professores, adição de conteúdos apresentados através de recursos visuais como vídeo, melhorias na \textit{interface} gráfica para aumentar a usabilidade do sistema. 

Por fim, pode-se avaliar o aprendizado que o sistema pode proporcionar à alunos durante seu uso como complemento ao ensino.




	
	%Elementos pós-textuais	
	\bibliography{elementos-pos-textuais/referencias}
%	\imprimirglossario	
	\imprimirapendices
		% Adicione aqui os apendices do seu trabalho
		\apendice{ TÍTULO}
\label{ap:TITULO}

Texto texto texto texto texto texto texto texto texto texto texto texto texto texto texto texto texto texto texto texto texto texto texto texto texto texto texto texto texto texto texto texto texto texto texto texto texto texto texto texto.
		\apendice{Modelo de Capa}
\label{ap:apendice_b}

Texto texto texto texto texto texto texto texto texto texto texto texto texto texto texto texto texto texto texto texto texto texto texto texto texto texto texto texto texto texto texto texto texto texto texto texto texto texto texto texto.

	\imprimiranexos
		% Adicione aqui os anexos do seu trabalho
		\anexo{Exemplo de Anexo}
\label{an:exemplo-de-anexo}

Texto texto texto texto texto texto texto  texto texto texto  texto texto texto  texto texto texto  texto texto texto  texto texto texto  texto texto texto  texto texto texto  texto texto texto  texto texto texto  texto texto texto. 		
		\anexo{Titulo anexo}
\label{an:anexo_b}

Texto texto texto texto texto texto texto texto texto texto texto.

	\imprimirindice

\end{document}