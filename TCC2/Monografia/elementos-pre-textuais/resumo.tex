Com o aumento dos recursos tecnológicos em sala de aula, surgiu a nescessidade da criação de novas metodologias de ensino. Nessa perspectiva, o objetivo do presente trabalho foi desenvolver um sistema de informação para auxiliar no ensino e aprendizado de matemática por alunos de níveis médio e superior que buscam complemento ao atual método de ensino. Para isso, desenvolvemos um estudo de natureza tecnológica de caráter exploratório. O sistema de informação implementado baseou-se em padrões de projeto e tecnologias voltadas para a \textit{Internet}. Como resultado desse trabalho, obtivemos um Ambiente Virtual de Aprendizagem que utilizando-se da metodologia de ensino resolução de problemas, possibilita alunos estudarem matemática dentro e fora da sala de aula de maneira auto-ritmada, permitindo ainda que professores possam acompanhar o progresso no aprendizado de seus alunos e intervir quanto estes enfrentarem dificuldades no aprendizado.   


% Separe as palavras-chave por ponto
\palavraschave{Ambiente Virtual de Aprendizagem. Educação. Metodologia de Ensino.}