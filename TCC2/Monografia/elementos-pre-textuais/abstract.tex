With the increased technological resources available to the classroom, came the demand for conceiving new teaching methodologies. In this perspective, the present work aimed to develop an information system to assist high school and univeristy level students in their teaching and learning of mathematics. To that end, we developed a technological study of exploratory character. The information system was developed using design patterns and Internet-oriented technologies. As a result of this work, we obtained a Virtual Environment of Learning that, resorting to the teaching methodology of problem resolution, enables students to study math in and out of classroom in their own rhythm, while still allowing teachers to track the learning progress of their students and intervene as they face learning difficulties.

\keywords{Virtual learning environment. Education. Teaching methodology}