With the increase of technological resources in the classroom, came the nescessidade of creation of new methods of teaching. In this line, the objective of this study was to develop a teaching platform to assist students from levels middle and higher in their teaching and learning of mathematics. For this reason, the platform developed uses of teaching methodology based on problem resolution to enable students to study mathematics within and outside of the classroom and at their own pace, allowing further that teachers can follow the progress in learning of their students and intervene when they face difficulties during their learning.

\keywords{Software - development. Mathematics teaching. Virtual learning environment. Teaching-Learning Process.}