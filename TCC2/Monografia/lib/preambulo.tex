%%%%%%%%%%%%%%%%%%%%%%%%%%%%%%%%%%%%%%%%%%%%%%%%%%%%%%%
%%      Para começar a usar este template, primeiro, %%
%% você dever criar uma conta no ShareLates. Depois, %%
%% vá nasopções no canto esquerdo superior da tela e %%
%% clique em "Copiar Projeto". Dê um novo nome para  %%
%% o projeto. This work has the LPPL maintenance     %%
%% status `maintained' The Currentt Maintainer of    %%
%% this work are:                                    %%
%%                                                   %%
%%        Ednardo Moreira Rodrigues (UFC/DEE)        %%
%%                      and                          %%
%%        Alan Batista de Oliveira (UFC/DEE)         %%
%%                                                   %%
%% Review:                                           %%
%%                                                   %%
%% - Eliene Maria Vieira de Moura;                   %%
%% - Francisco Edvander Pires Santos;                %%  
%% - Izabel Lima dos Santos;                         %%
%% - Juliana Soares Lima;                            %%
%% - Kalline Yasmin Soares Feitosa.                  %%
%%                                                   %%
%% This work may be distributed and/or modified under%%
%% theconditions of the LaTeX Project Public License,%%
%% either version 1.3 of this license or (at your    %%
%% option) any                                       %%
%% later version. The latest version of this license %%
%% is in http://www.latex-project.org/lppl.txt and   %%
%% version 1.3 or later is part of all distributions %%
%% of LaTeX version 2005/12/01 or later.             %%
%% The First Maintainer of this work was:            %%
%% Thiago Nascimento  (UECE)                         %%
%% Project available on:                             %%
%% https://github.com/thiagodnf/uecetex2             %%
%% Further information about abnTeX2                 %%
%% are available on http://abntex2.googlecode.com/   %%
%%%%%%%%%%%%%%%%%%%%%%%%%%%%%%%%%%%%%%%%%%%%%%%%%%%%%%%

% \documentclass[        
%     a4paper,          % Tamanho da folha A4
%     12pt,             % Tamanho da fonte 12pt
%     chapter=TITLE,    % Todos os capitulos devem ter caixa alta
%     section=TITLE,    % Todas as secoes devem ter caixa alta
%     oneside,          % Usada para impressao em apenas uma face do papel
%     english,          % Hifenizacoes em ingles
%     spanish,          % Hifenizacoes em espanhol
%     brazil            % Ultimo idioma eh o idioma padrao do documento
% ]{abntex2}

% Importações de pacotes
% possibilita ctrl+c com acentos em português
\usepackage[utf8]{inputenc}                         % Acentuação direta
\usepackage[T1]{fontenc}                            % Codificação da fonte em 8 bits
\usepackage{graphicx}                               % Inserir figuras
\usepackage{amsfonts, amssymb, amsmath}             % Fonte e símbolos matemáticos
\usepackage{booktabs}                               % Comandos para tabelas
\usepackage{verbatim}                               % Texto é interpretado como escrito no documento
\usepackage{multirow, array}                        % Múltiplas linhas e colunas em tabelas
\usepackage{indentfirst}                            % Endenta o primeiro parágrafo de cada seção.
\usepackage{listings}                               % Utilizar codigo fonte no documento
\usepackage{xcolor}
\usepackage{microtype}                              % Para melhorias de justificação?
\usepackage[portuguese,ruled,lined]{algorithm2e}    % Escrever algoritmos
\usepackage{algorithmic}                            % Criar Algoritmos  
%\usepackage{float}                                 % Utilizado para criação de floats
\usepackage{amsgen}
\usepackage{lipsum}                                 % Usar a simulação de texto Lorem Ipsum
%\usepackage{titlesec}                              % Permite alterar os títulos do documento
\usepackage{tocloft}                                % Permite alterar a formatação do Sumário
\usepackage{etoolbox}                               % Usado para alterar a fonte da Section no Sumário
\usepackage[acronym,nogroupskip,nonumberlist]{glossaries}   % Permite fazer o glossario
\usepackage[font=singlespacing]{caption}                                % Altera o comportamento da tag caption
\usepackage[alf, abnt-emphasize=bf, recuo=0cm, abnt-etal-cite=2, abnt-etal-list=0, abnt-etal-text=it]{abntex2cite}  % Citações padrão ABNT
%\usepackage[bottom]{footmisc}                      % Mantém as notas de rodapé sempre na mesma posição
%\usepackage{times}                                 % Usa a fonte Times
\usepackage{mathptmx}                               % Usa a fonte Times New Roman										
%\usepackage{lmodern}                               % Usa a fonte Latin Modern
%\usepackage{subfig}                                % Posicionamento de figuras
%\usepackage{scalefnt}                              % Permite redimensionar tamanho da fonte
%\usepackage{color, colortbl}                       % Comandos de cores
%\usepackage{lscape}                                % Permite páginas em modo "paisagem"
\usepackage{ae, aecompl}                            % Fontes de alta qualidade
%\usepackage{picinpar}                              % Dispor imagens em parágrafos
\usepackage{latexsym}                               % Símbolos matemáticos
%\usepackage{upgreek}                               % Fonte letras gregas
\usepackage{appendix}                               % Gerar o apendice no final do documento
\usepackage{paracol}                                % Criar paragrafos sem identacao
\usepackage{lib/ufctex}		                        % Biblioteca com as normas da UFC para trabalhos academicos
\usepackage{pdfpages}                               % Incluir pdf no documento
\usepackage{amsmath}                                % Usar equacoes matematicas


% Meus Pacotes - Não pertencem ao template original
% adicionar os simbolos dos question\'arios
\usepackage{threeparttable}
\usepackage{wasysym}


% Organiza e gera a lista de abreviaturas, simbolos e glossario
\makeglossaries

% Gera o Indice do documento
\makeindex
