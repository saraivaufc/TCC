% arara: pdflatex
% arara: nomencl
% arara: pdflatex

%%%%%%%%%%%%%%%%%%%%%%%%%%%%%%%%%%%%%%%%%%%%%%%%%%%%%%%
%%      Para começar a usar este template, primeiro, %%
%% você dever criar uma conta no ShareLates. Depois, %%
%% vá nasopções no canto esquerdo superior da tela e %%
%% clique em "Copiar Projeto". Dê um novo nome para  %%
%% o projeto. This work has the LPPL maintenance     %%
%% status `maintained' The Currentt Maintainer of    %%
%% this work are:                                    %%
%%                                                   %%
%%        Ednardo Moreira Rodrigues (UFC/DEE)        %%
%%                      and                          %%
%%        Alan Batista de Oliveira (UFC/DEE)         %%
%%                                                   %%
%% Review:                                           %%
%%                                                   %%
%% - Eliene Maria Vieira de Moura;                   %%
%% - Francisco Edvander Pires Santos;                %%  
%% - Izabel Lima dos Santos;                         %%
%% - Juliana Soares Lima;                            %%
%% - Kalline Yasmin Soares Feitosa.                  %%
%%                                                   %%
%% This work may be distributed and/or modified under%%
%% theconditions of the LaTeX Project Public License,%%
%% either version 1.3 of this license or (at your    %%
%% option) any                                       %%
%% later version. The latest version of this license %%
%% is in http://www.latex-project.org/lppl.txt and   %%
%% version 1.3 or later is part of all distributions %%
%% of LaTeX version 2005/12/01 or later.             %%
%% The First Maintainer of this work was:            %%
%% Thiago Nascimento  (UECE)                         %%
%% Project available on:                             %%
%% https://github.com/thiagodnf/uecetex2             %%
%% Further information about abnTeX2                 %%
%% are available on http://abntex2.googlecode.com/   %%
%%%%%%%%%%%%%%%%%%%%%%%%%%%%%%%%%%%%%%%%%%%%%%%%%%%%%%%

\documentclass[        
    a4paper,          % Tamanho da folha A4
    12pt,             % Tamanho da fonte 12pt
    chapter=TITLE,    % Todos os capitulos devem ter caixa alta
    section=Title,    % Todas as secoes devem ter caixa alta somente na primeira letra
    subsection=Title, % Todas as subsecoes devem ter caixa alta somente na primeira letra
    oneside,          % Usada para impressao em apenas uma face do papel
    english,          % Hifenizacoes em ingles
    spanish,          % Hifenizacoes em espanhol
    brazil,           % Ultimo idioma eh o idioma padrao do documento
    fleqn             % Coloca as equações alinhadas a esquerda
]{abntex2}

\input{lib/preambulo}

%%%%%%%%%%%%%%%%%%%%%%%%%%%%%%%%%%%%%%%%%%%%%%%%%%%%%
%%          Configuracoes do ufctex                %%
%%%%%%%%%%%%%%%%%%%%%%%%%%%%%%%%%%%%%%%%%%%%%%%%%%%%%

% Opcoes disponiveis

\trabalhoacademico{tccgraduacao}
%\trabalhoacademico{tccespecializacao}
%\trabalhoacademico{dissertacao}
%\trabalhoacademico{tese}

% Define se o trabalho eh uma qualificacao
% Coloque 'nao' para versao final do trabalho

\ehqualificacao{nao}

% Remove as bordas vermelhas e verdes do PDF gerado
% Coloque 'sim' pare remover

\removerbordasdohyperlink{sim} 

% Adiciona a cor Azul a todos os hyperlinks

\cordohyperlink{nao}

%%%%%%%%%%%%%%%%%%%%%%%%%%%%%%%%%%%%%%%%%%%%%%%%%%%%%
%%          Informação sobre a IES                 %%
%%%%%%%%%%%%%%%%%%%%%%%%%%%%%%%%%%%%%%%%%%%%%%%%%%%%%

\ies{Universidade Federal do Ceará}
\iessigla{UFC}
\centro{Campus Quixadá}

%%%%%%%%%%%%%%%%%%%%%%%%%%%%%%%%%%%%%%%%%%%%%%%%%%%%%
%%        Informação para TCC de Graduacao %%
%%%%%%%%%%%%%%%%%%%%%%%%%%%%%%%%%%%%%%%%%%%%%%%%%%%%%

\graduacaoem{Sistemas de Informação}
\habilitacao{bacharel} % Pode colocar tambem 'licenciada'

%%%%%%%%%%%%%%%%%%%%%%%%%%%%%%%%%%%%%%%%%%%%%%%%%%%%%
%%     Informação para TCC de Especializacao       %%
%%%%%%%%%%%%%%%%%%%%%%%%%%%%%%%%%%%%%%%%%%%%%%%%%%%%%

% \especializacaoem{Descargas Atmosféricas}

%%%%%%%%%%%%%%%%%%%%%%%%%%%%%%%%%%%%%%%%%%%%%%%%%%%%%
%%         Informação para Dissertacao             %%
%%%%%%%%%%%%%%%%%%%%%%%%%%%%%%%%%%%%%%%%%%%%%%%%%%%%%

% \programamestrado{Programa de Pós-Graduação em Xxxxxxx}
% \nomedomestrado{Mestrado Acadêmico em Xxxxxxx}
% \mestreem{Engenharia Xxxxxx}
% \areadeconcentracaomestrado{Engenharia Xxxxxx}

%%%%%%%%%%%%%%%%%%%%%%%%%%%%%%%%%%%%%%%%%%%%%%%%%%%%%
%%               Informação para Tese              %%
%%%%%%%%%%%%%%%%%%%%%%%%%%%%%%%%%%%%%%%%%%%%%%%%%%%%%

% \programadoutorado{Programa de Pós-Graduação em Xxxxxx}
% \nomedodoutorado{Doutorado em Xxxxxxx}
% \doutorem{Engenharia Xxxxxx}
% \areadeconcentracaodoutorado{Engenharia Xxxxxxx}

%%%%%%%%%%%%%%%%%%%%%%%%%%%%%%%%%%%%%%%%%%%%%%
%%  Informacoes relacionadas ao trabalho     %%
%%%%%%%%%%%%%%%%%%%%%%%%%%%%%%%%%%%%%%%%%%%%%%

\autor{Marciano Machado Saraiva}
\titulo{Um Ambiente Virtual de Aprendizagem para Auxiliar no Processo de Ensino e Aprendizagem de Matemática}
\data{2016}
\local{Quixadá}

% Exemplo: \dataaprovacao{01 de Janeiro de 2012}
\dataaprovacao{}

%%%%%%%%%%%%%%%%%%%%%%%%%%%%%%%%%%%%%%%%%%%%%
%%     Informação sobre o Orientador       %%
%%%%%%%%%%%%%%%%%%%%%%%%%%%%%%%%%%%%%%%%%%%%%

\orientador{Prof. Me. Samy Soares Passos de Sá}
\orientadories{Universidade Federal do Ceará (UFC)}
\orientadorcentro{Campus Quixadá}
\orientadorfeminino{nao} % Coloque 'sim' se for do sexo feminino

%%%%%%%%%%%%%%%%%%%%%%%%%%%%%%%%%%%%%%%%%%%%%
%%      Informação sobre o Co-orientador   %%
%%%%%%%%%%%%%%%%%%%%%%%%%%%%%%%%%%%%%%%%%%%%%

% Deixe o nome do coorientador em branco para remover do documento

\coorientador{}
\coorientadories{Universidade Co-orientador (SIGLA)}
\coorientadorcentro{Centro do Co-orientador (SIGLA)}
\coorientadorfeminino{nao} % Coloque 'sim' se for do sexo feminino

%%%%%%%%%%%%%%%%%%%%%%%%%%%%%%%%%%%%%%%%%%%%%
%%      Informação sobre a banca           %%
%%%%%%%%%%%%%%%%%%%%%%%%%%%%%%%%%%%%%%%%%%%%%

% Atenção! Deixe o nome do membro da banca para remover da folha de aprovacao

% Exemplo de uso:
% \membrodabancadois{Prof. Dr. Fulano de Tal}
% \membrodabancadoisies{Universidade Federal do Ceará - UFC}

\membrodabancadois{Prof. Dr. Wladmir Araujo Tavares}
\membrodabancadoiscentro{Campus Quixadá}
\membrodabancadoisies{Universidade Federal do Ceará - UFC}

\membrodabancatres{Prof. Me. Carlos Roberto Rodrigues Filho}
\membrodabancatrescentro{Campus Quixadá}
\membrodabancatresies{Universidade Federal do Ceará - UFC}

% \membrodabancaquatro{}
% \membrodabancaquatrocentro{Centro de Ciências e Tecnologia (CCT)}
% \membrodabancaquatroies{Universidade do Membro da Banca Quatro (SIGLA)}
% \membrodabancacinco{}
% \membrodabancacincocentro{Teste}
% \membrodabancacincoies{Universidade do Membro da Banca Cinco (SIGLA)}
% \membrodabancaseis{}
% \membrodabancaseiscentro{}
% \membrodabancaseisies{Universidade do Membro da Banca Seis (SIGLA)}

\begin{document}	

	% Elementos pré-textuais
	\imprimircapa
	\imprimirfolhaderosto{}
	\imprimirfichacatalografica{elementos-pre-textuais/ficha-catalografica}
	%\imprimirerrata{elementos-pre-textuais/errata}
	\imprimirfolhadeaprovacao
	\imprimirdedicatoria{elementos-pre-textuais/dedicatoria}
	\imprimiragradecimentos{elementos-pre-textuais/agradecimentos}
	\imprimirepigrafe{elementos-pre-textuais/epigrafe}
	\imprimirresumo{elementos-pre-textuais/resumo}
	\imprimirabstract{elementos-pre-textuais/abstract}
	\imprimirlistadeilustracoes
	\imprimirlistadetabelas
	%\imprimirlistadequadros
	%\imprimirlistadealgoritmos
	%\imprimirlistadecodigosfonte
 	\imprimirlistadeabreviaturasesiglas
	\imprimirlistadesimbolos{elementos-pre-textuais/lista-de-simbolos}   
	\imprimirsumario
	
	%Elementos textuais
	\textual
	\section{INTRODUÇÃO}

A Matemática, como ciência, sempre teve uma relação muito especial com as novas tecnologias, desde as calculadoras, os computadores, aos sistemas multimédia e a Internet. No entanto, os professores costumam demorar a perceber como tirar partido destas tecnologias como ferramentas de trabalho. \cite{da1997ensino}. 

A medida que a quantidade de recursos tecnológicos na sala de aula foram aumentando, tornou-se necessário a criação de novas metodologias de ensino, especificamente na Educação Matemática. Tal busca procura fazer da Matemática uma disciplina atraente, desvinculada do ensino tradicional que já se mostrou ineficiente \cite{silva2009ambiente}.

Tendo em vista essa necessidade, o presente trabalho, busca apresentar o projeto e desenvolvimento de um Ambiente Virtual de Aprendizagem (AVA) \cite{valentini2010aprendizagem}, para auxiliar estudantes no ensino e aprendizagem de conteúdos matemáticos. Este ambiente se propõe a servir como ferramenta para estudantes que buscam estudar fora do ambiente escolar e de uma forma auto-ritmada, ou seja, no seu próprio ritmo. Entretanto, o ambiente também dará suporte ao ensino e aprendizagem dentro da sala de aula, auxiliando professores com informações relevantes sobre o andamento do aprendizado de cada um de seus aluno, além das dificuldades que os mesmos apresentam.

Grande parte dos alunos tem dificuldades em aprender matemática, e muitas vezes essas dificuldades ocorrem não pela falta de atenção ou por não gostar do conteúdo, mas por fatores mentais ou psicológicos que envolvem uma série de trabalhos
e conceitos que precisam ser desenvolvidos \cite{sa2015software}. Mas como auxiliar alunos com dificuldades na aprendizagem da matemática?

Em busca dessa resposta, diferentes sistemas de \textit{softwares} foram desenvolvidos buscando servir a educação. Em 2006, Salman Khan funda a Khan Academy, uma organização educacional que tem por objetivo oferecer exercícios, vídeos de instrução e um painel de aprendizado personalizado que habilita os estudantes a aprender no seu próprio ritmo dentro e fora da sala de aula \cite{khan2012one}. A plataforma criado por khan utiliza de vídeo-aulas e resolução de problemas para ensinar seus alunos, permitindo assim, que cada um tenha uma aprendizagem de forma independente e auto-ritmada.

Um outro trabalho também  importante nessa área, podemos citar o de  \citeonline{melis2001activemath}, onde os autores desenvolvem um AVA que permite os alunos desfrutarem da experiência de estudar num curso gerado dinamicamente a partir do estado em que o mesmo se encontra dentro do sistema. Estes e outros trabalhos são apresentados com mais detalhes em trabalhos relacionados.

\subsection{Objetivos do Trabalho}

Este trabalho visa contribuir com o processo de ensino e aprendizagem de matemática, buscando projetar e desenvolver um ambiente virtual que auxilie no processo de ensino e aprendizagem de matemática por alunos, dentro e fora da sala de aula. 

Como objetivos específicos para este trabalho, temos: 
\begin{alineascomponto}
    \item Elaborar um método para diagnosticar, nos alunos, dificuldades existentes em certos conteúdos de matemática.
	\item Desenvolver o AVA, aplicando técnicas de gamificação.
    \item Aplicar o AVA numa turma de graduação onde os alunos estejam cursando disciplinas de matemática.
    \item Analisar a influência e impactos provocados pelo AVA, por meio de dados gerados pela ferramenta ao longo de sua utilização e questionários aplicados aos alunos que utilizarem o AVA.

\end{alineascomponto}


\subsection{Divisão do Trabalho}

Fundamentando-se na problemática mencionada, e tendo em vista o objeto de estudo, esta dissertação foi dividida em cinco capítulos. No capítulo inicial é feita uma apresentação do tema a ser discutido durante toda esta dissertação e o objetivo do trabalho.

No Capítulo 2, destacam-se os aspectos teóricos sobre ensino e aprendizagem, assim como as tradicionais metodologias de ensino e as apoiadas por computador, além de conceitos sobre gamificação e os trabalhos que serviram de referencial para abordar os conceitos e ideias utilizadas no trabalho aqui desenvolvido.

No Capítulo 3 compreende-se a concepção, construção e modelagem do sistema, apresentando o que o mesmo deve possuir e por que.

No Capítulo 4 são apresentados os resultados preliminares.
	\chapter{Revisão Bibliográfica}
\label{cap:fundamentacao-teorica}

Este capítulo aborda os seguintes temas: metodologias no ensino da matemática, conceito de ambientes virtuais de aprendizagem, o uso da gamificação aplicada em AVAs, e, por fim, o que está sendo 
desenvolvido por outros pesquisadores da área.

\section{Metodologias no Ensino da Matemática}

Ao longo da história, várias metodologias e abordagens matemáticas foram utilizadas visando a melhoria do ensino. Segundo \citeonline{hammes2003tendencias}, algumas delas foram aula expositiva, 
resolução de problemas, modelagem matemática, e o uso de computadores. Nas seções a seguir, descreveremos cada uma delas.

\subsection{Aula expositiva}

Em uma aula expositiva, o professor comumente faz uma revisão da aula anterior, apresenta o novo conteúdo e passa aos alunos uma série de exercícios de fixação. Esse novo conteúdo é apresentado de 
forma oral ou escrita, sem levar em consideração o conhecimento prévio dos estudantes nem tempo para perguntas. Essa é, sem dúvida, uma das mais utilizadas e antigas metodologias existentes. Durante 
o século passado, aulas expositivas foram o único processo empregado em sala de aula pelos professores. Dessa forma, a aula expositiva pode ser considerada cansativa e desinteressante, já que o aluno não participa do processo de ensino  \cite{hammes2003tendencias}. Para \citeonline{de1996gerencia}, essa abordagem possui diversos problemas. \citeonline{de1996gerencia} sugere que a escola tradicional não somente está desatualizada para atender às necessidades 
crescentes da sociedade contemporânea, como também apresenta algumas características que inibem o desenvolvimento do potencial de criação dos alunos:

\begin{alineascomponto}
\item Destaca-se a incompetência, a ignorância e a incapacidade do aluno, deixando de assinalar os talentos e habilidades de cada um; 
\item O ensino voltado para o passado, em que se enfatiza a reprodução e a memorização do conhecimento;
\item Desconsidera-se a imaginação e a fantasia como dimensões importantes da mente;
\item Exercício de resposta única, em que se cultua o medo do erro e do fracasso;
\item A obediência, dependência, passividade e conformismo são os traços mais cultivados;
\item Descaso em cultivar uma visão otimista do futuro;
\item As habilidades cognitivas são desenvolvidas de forma limitada;
\end{alineascomponto}

Autores como \citeonline{lopes1995aula}, defendem que a aula expositiva ``poderá ser transformada em uma atividade dinâmica, participativa e estimuladora do pensamento critico do aluno''. Uma alternativa descrita por \citeonline{lopes1995aula} é transformar a aula expositiva em uma aula expositiva dialógica. Essa forma de aula expositiva utiliza o diálogo entre professor e aluno para estabelecer uma relação de intercambio de conhecimentos e experiências \cite{lopes1995aula}. De acordo com \citeonline{freire1982educaccao}, o ensino dialógico se contrapõe ao ensino 
autoritário,  transformando  a  sala  de  aula  em  ambiente  propicio  à  reelaboração  e produção de conhecimentos.


\subsection{Resolução de problemas}\label{resolucao_problemas}

A resolução de problemas deve ser entendida como uma oportunidade para o aluno obter novos conhecimentos e não apenas conhecimentos prontos que fazem parte da nossa história. Ela ajuda o aluno a desenvolver sua autonomia buscando as respostas para seus próprios questionamentos. \citeonline{fossa1998tendencias} ressaltam  que esta metodologia visa o desenvolvimento de habilidades metacognitivas, favorecendo a reflexão e o questionamento do aluno que aprende a pensar por si mesmo, levantando hipóteses, testando-as, tirando conclusões e até discutindo-as com os colegas.

Na metodologia de resolução de problemas, é importante que os problemas apresentem uma incógnita que necessite ser descoberta. Para resolvê-los, o aluno terá que inventar estratégias e gerar novas ideias. Segundo \citeonline{dante1991didatica} é importante que o problema possa gerar muitos processos de pensamento, levantar muitas hipóteses e propiciar várias estratégias de solução. O pensar e o fazer criativo devem ser componentes fundamentais no processo de resolução de problemas.

\subsection{Modelagem Matemática}

De acordo com \citeonline[p.~16]{bassanezi2002ensino}, “[...] a modelagem consiste na arte de transformar situações da realidade em problemas matemáticos cujas soluções devem ser interpretadas na linguagem do mundo real”. Nas metodologias anteriores, o processo de ensino é deflagrado pelo professor. Na Modelagem Matemática, o processo é compartilhado com o grupo de alunos, pois sua motivação advém
do interesse pelo assunto. 

A modelagem matemática é uma metodologia que busca proporcionar ao aluno uma visão prática do conhecimento teórico aprendido na sala de aula, através de problemas de ordem prática ou de natureza empírica \cite{fossa1998tendencias}. 

\citeonline{burak2004modelagem} destaca alguns aspectos importantes a cerca dos benefícios da utilização da modelagem matemática:

\begin{alineascomponto}
	\item Maior interesse do grupo, pois o fato do grupo escolher aquilo que gostaria de estudar, ter a oportunidade de se manifestar, de discutir e propor, desenvolve o interesse do grupo.
    \item Interação maior no processo de ensino e de aprendizagem, pois o grupo de alunos trabalha com aquilo que gostam e que apresenta significado, tornando-os co-responsáveis pela aprendizagem. 
\end{alineascomponto}


\subsection{O Uso de Computadores}

A aprendizagem mediada por computadores surgiu em 1960 na Universidade de Illinois com o projeto PLATO (Programmed Logic for Automatic Teaching Operations)\cite{bitzer1961plato}, que deu origem ao primeiro sistema de ensino assistido por computador, o qual permitia a criação e apresentação de materiais sobre gram\'atica (passar um verbo para o passado, reescrever um substantivo no plural, etc.) com revisão automática. De acordo com \citeonline{woolley1994plato}, PLATO apoiou inicialmente apenas uma única sala de aula com 20 alunos, até que em 1972, o sistema migrou para uma nova geração de \textit{mainframes} que acabaria por apoiar milhares de terminais gráficos distribuídos em todo o mundo. O principal fator motivador para a introdução do computador na educação, segundo \citeonline{silva2009ambiente}, foi o surgimento no final do século XX de um conhecimento baseado em simulação, característico da cultura informática, o que fez com que o computador fosse visto como  um recurso didático  indispensável.

Os benefícios da utilização do computador como um instrumento de ensino e aprendizagem, de acordo com \citeonline[p.12]{almeida2000proinfo}, referem-se a sua utilização como ``uma máquina que 
possibilita testar ideias ou hipóteses, que levam à criação de um mundo abstrato e simbólico, ao mesmo tempo em que permite introduzir diferentes formas de atuação e interação entre as pessoas''. Já a 
principal associação de professores de matemática dos Estados Unidos (NCTM), diz que ``a tecnologia é essencial no ensino e 
na aprendizagem da Matemática'' e ``influencia a Matemática que é ensinada e melhora a aprendizagem dos alunos'' permitindo que estes se concentrem ``nas decisões a tomar, na reflexão, no raciocínio e 
na resolução de problemas''. \cite[p.26]{melo2007principios}.

Quando se fala no uso das Tecnologias da Informação e Comunicação (TIC) como um mediador para o processo de ensino e aprendizagem, muitas vezes, acaba-se esquecendo do papel do professor nesse 
processo. Num mundo em que há uma grande variedade de formas de utilização das TIC para apoio a educação, cabe ao professor decidir como e quando utilizar essas tecnologias e quais s\~ao formas 
mais eficazes para sua utilização levando em consideração os conteúdos que serão ofertados. Para isso, o professor necessita mudar sua metodologia de ensino, e isso pode resultar em problemas, já 
que, assim como afirmam \citeonline[p.~96]{bitner2002integrating}:

\begin{citacao}
``Adultos não mudam facilmente. Mudança de qualquer tipo traz medo, ansiedade e preocupação. A utilização da tecnologia como uma ferramenta de ensino e aprendizagem na sala de aula faz isso em um grau ainda maior, uma vez que envolve tanto mudanças nos procedimentos de sala de aula e o uso de tecnologias muitas vezes desconhecidas. Os responsáveis por pedir aos professores para usar a tecnologia no currículo devem estar cientes de que existem medos e preocupações.'' \cite[p.~96, Tradução Livre]{bitner2002integrating}.
\end{citacao}

Os problemas com a inserção do computador na educação podem ser ainda maiores, tendo em vista que muito se cogita sobre seu uso no ensino ser a solução para muitos dos problemas da educação, sendo que muitos desses problemas não podem encontrar uma solução nas tecnologias digitais \cite{silva2009ambiente}. Um dos fatores que pode influenciar negativamente no processo de aprendizagem mediada por computador é o domínio do computador pelo aluno, tendo em vista que sua rapidez de evolução assim como sua própria complexidade, torna essa tarefa muito difícil de ser alcançada.

\section{Ambientes Virtuais de Aprendizagem}

O surgimento de Ambientes Virtuais de Aprendizagem deu-se logo após o surgimento da internet nos anos 90. Nessa mesma época, novas ferramentas e produtos foram desenvolvidas para explorar os benefícios que a rede mundial de computadores trouxe \cite{oleary2002virtual}.

Para \citeonline{valentini2010aprendizagem}, um Ambiente Virtual de Aprendizagem é um espaço social, constituído de interações cognitivo-sociais sobre (ou em torno de) um objeto de conhecimento no qual as pessoas interagem, mediadas pela linguagem da hipermídia, visando o processo de ensino-aprendizagem.

Geralmente, os AVAs possuem algumas características que os distinguem de outros tipos de sistemas de softwares. Segundo \citeonline{oleary2002virtual}, algumas dessas características são:
\begin{alineascomponto}
    \item a comunicação entre tutores e alunos - por exemplo, email, fórum de discussão e bata-papo virtual;
    \item a auto-avaliação e avaliação sumativa - por exemplo, avaliação de múltipla escolha com automatizada marcação e feedback imediato; 
    \item entrega de recursos de aprendizagem e materiais - por exemplo, através do fornecimento de notas de aula e materiais, imagens e clips de vídeo;
    \item áreas do grupo de trabalho compartilhados - possibilita os usuários compartilharem arquivos e se comunicarem;
    \item suporte para estudantes - possibilita a comunicação entre os tutores e seus estudantes, fornecimentos de materiais didáticos e alguma forma de tirar as dúvidas dos alunos;
    \item gestão e acompanhamento dos estudantes - sistema de autenticação para permitir que apenas estudantes tenham acesso aos cursos;
    \item ferramentas para o estudante - por exemplo agendas e calendários eletrônicos;
    \item aparência consistente e personalizável - uma interface padrão de fácil utilização, permitindo personalização, mas com um modo de utilização básico.
\end{alineascomponto}

A utilização de AVAs traz diversas vantagens como cita \citeonline[p.153]{tajra2001ferramentas}: acessibilidade a fontes inesgotáveis de assuntos para pesquisas, páginas educacionais específicas para a pesquisa escolar, comunicação e interação com outras escolas, estímulo para pesquisas a partir de temas previamente definidos ou a partir da curiosidade dos próprios alunos, estímulo ao raciocínio lógico, troca de experiências entre professores/professores, aluno/aluno e professor/aluno, dentre outras.

\citeonline{carvalho2013ambiente} afirma que AVAs integram funcionalidades de comunicação e partilha de informações e isso permite aceder à aprendizagem de uma forma flexível; em qualquer espaço(\textit{anywhere}) e em qualquer hora (\textit{anytime}). O autor complementa que:

\begin{citacao}
``Um AVA deve, por um lado, enfatizar a aprendizagem através da integração de ferramentas interativas e comunicativas, da partilha de conteúdos multimédia, do alojamento de trabalhos e projetos, da integração de ferramentas de aprendizagem colaborativa, e por outro, deve proporcionar estratégias que potenciem a participação ativa e significativa dos alunos, abranger possibilidades didáticas de aprendizagem individual e em grupo, criar novos acessos a websites como forma de enriquecer o conhecimento, possuir ferramentas de controlo de acesso e registro de utilizadores e de gestão de grupos de trabalho'' \cite[p.~41]{carvalho2013ambiente}. 
\end{citacao}

Os AVAs  geralmente utilizam diversas ferramentas para apoio ao ensino, como já foi citado anteriormente, tais como fóruns, chats, wikis, glossários, portfólios, enquetes, questionários, entre outros. Essas ferramentas, de acordo com \citeonline{masetto2012competencia}, são recursos em linguagem digital e podem colaborar significativamente para tornar a educação mais eficiente e eficaz.

\section{Gamificação}

Ao longo da história, buscamos metodologias inovadoras para auxiliar a educação. Uma das metodologias que mais diferem do tradicional método de ensino é a utilização de jogos para o ensino e aprendizagem. 

Surgiu em 2002, por meio de Nick Pelling, programador de computadores e inventor britânico, o termo Gamificação. \citeonline{fardo2013gamificaccao} define a Gamificação como o emprego de conceitos e 
técnicas criadas e utilizadas em jogos para auxiliar na educação. Alguns exemplos s\~ao:

\begin{alineascomponto}
	\item Sistema de Pontos: são abertos, diretos e motivacionais, permitindo a utilização de vários tipos diferentes de pontuação, de acordo com o objetivo proposto.
	\item Medalhas e Conquistas: tratam-se de uma representação visual de alguma realização/conquista do usuário no sistema. Os usuários querem receber medalhas dentro de um ambiente por diversos 
 motivos, para muitos, o objetivo é a experiência agradável de receber a medalha ou por “colecionar” medalhas.
	\item Desafios e Missões: são os elementos que orientam os usuários sobre as atividades que devem ser realizadas dentro de um sistema. É importante que existam desafios 
para os usuários completarem, pois isso fará com que exista algo interessante para ele realizar enquanto interage com o sistema.
	\item N\'iveis: indicam o progresso do usuário dentro do sistema. 
	\item Rankings: seu propósito principal é a comparação entre os jogadores/usuários envolvidos.
	\item Personalização: permite o usuário transformar e personalizar itens que compõem o sistema de acordo com seu gosto, promovendo motivação, engajamento, sentimento de posse e controle sobre 
o sistema. 
\end{alineascomponto}


Para \citeonline{halliwell2013gamification}, o objetivo da Gamificação não é transformar tudo em um jogo, mas sim, encontrar a diversão, encontrar os aspectos `jogáveis' de um problema, quaisquer que 
sejam, e usá-los para criar um ambiente que mova as pessoas um pouco mais em direção a um objetivo que tenham criado. Um exemplo bem interessante do uso de gamificação é o Duolingo\footnote{O 
Duolingo é uma plataforma para ensino de idiomas gratuito, dispon\'ivel em \url{www.duolingo.com} } \cite{von2013duolingo}, uma plataforma online de aprendizagem em línguas que trabalha com o 
conceito de conhecimento coletivo e voluntário. No Duolingo, usuários podem ganhar pontos com as respostas corretas e lições completadas, assim como perder pontos a cada resposta incorreta. O mesmo 
ainda atribui status de reconhecimento de acordo o conhecimento obtido durante o jogo. Veja mais na \autoref{trabalho_relacionados_duolingo}.

\section{Trabalhos Relacionados}
\label{cap:trabalhos-relacionados}

Os Ambientes Virtuais de Aprendizagem vêm sendo utilizados na educação, principalmente como uma ferramenta mais dinâmica, se comparados \`as metodologias de ensino tradicionais, e são de 
grande potencial na área da educação. Existem diversas pesquisas voltadas a aplicação de AVAs para apoiar o processo de ensino-aprendizagem e, em especial, a Matemática se destaca entre elas.

\subsubsection{\textit{The one world schoolhouse: Education reimagined}}
Em um trabalho iniciado em 2006, \citeonline{khan2012one} fundou a chamada Khan Academy\footnote{Plataforma de aprendizagem disponível em: \url{www.khanacademy.org}}, organização educacional que tem 
por objetivo oferecer exercícios, vídeos de instrução e um painel de aprendizado personalizado que habilita os estudantes a aprender no seu próprio ritmo dentro e fora da sala de aula. Em sua 
plataforma, são abordados assuntos como matemática, ciência, programação de computadores, história, história da arte e economia, entre outros \cite{khan2012one}.

No ambiente, o desempenho do estudante é representado por medalhas. De acordo com o site da organização, medalhas e insignias estimulam o aprendizado de maneira lúdica. As estatísticas mostram o 
quanto de trabalho o estudante está fazendo a cada dia, o quanto o estudante está focado em áreas de habilidades e tópicos e as habilidades que o estudante concluiu. Com os relatórios gerados pela 
plataforma, o tutor pode acompanhar todos os passos do estudante.

Durante uma conferência TED\footnote{É uma série de conferências realizadas na Europa, Ásia e Américas pela fundação Sapling com o objetivo de disseminar ideias que podem mudar o mundo.} em 2011, 
denominada \textit{Salman Khan: Let's Use Video to Reinvent Education} \cite{tedtalk2011reinvend}, Khan fala sobre o funcionamento da plataforma e também do provável motivo do seu sucesso. Segundo 
\citeonline{tedtalk2011reinvend}, o diferencial da plataforma está em dois pontos importantes, a aprendizagem auto-ritmada e nos dados fornecidos aos tutores sobre o aprendizado de seus estudantes. 

Em relação a aprendizagem auto-ritmada, Khan afirma que:
\begin{citacao}
``Então quando fala-se em aprendizagem auto-ritmada, isso faz sentido para todo mundo – na chamada aprendizagem diferenciada – mas é meio maluco quando você vê isso na sala de aula. Porque toda vez 
que fizemos isso, em cada sala de aula que fizemos, repetidas vezes, depois de cinco dias nisso, há um grupo de garotos que está adiantado e há outro grupo de garotos um pouco atrás. E num modelo 
tradicional, se você fizesse uma avaliação pontual, diria: ``Esses são garotos inteligentes, aqueles são garotos lerdos. Talvez eles devessem ser acompanhados de forma diferente. Talvez devêssemos 
coloca-los em salas diferentes.'' Mas quando você deixa cada aluno trabalhar em seu próprio ritmo – e vemos isso repetidas vezes – você vê alunos que tomam um tempo extra em um conceito ou outro, mas 
uma vez que adquirem esse conceito, eles apenas vão adiante. E assim os mesmos garotos que você pensava que eram lerdos semanas atrás, agora você pensa que são inteligentes. E vemos isso repetidas 
vezes. E isso faz você se perguntar quantos estereótipos que talvez vários de nós recebemos eram apenas devido a uma coincidência de tempo'' \cite[13:29, Tradu\c{c}\~ao 
Livre]{tedtalk2011reinvend}.
\end{citacao}.


\citeonline{tedtalk2011reinvend} também fala sobre a importância dos dados fornecidos pelos tutores:
\begin{citacao}
``[...]. Nosso paradigma é armar os professores com a maior quantidade de dados possível – dados que, em quase qualquer outro campo, são esperados, se você trabalha com finanças ou propaganda ou 
fabricação. E assim o professores podem realmente diagnosticar o que está errado com os alunos de maneira que podem fazer suas interações mais produtivas possível. Agora os professores sabem 
exatamente o que os alunos têm feito, quanto tempo eles gastam todo dia, a quais vídeos assistiram, quando pausaram os vídeos, o que fez que parassem de ver, quais exercícios estavam fazendo, no que 
eles estavam se focando? [...]'' \cite[12:26, Tradu\c{c}\~ao Livre]{tedtalk2011reinvend}.
\end{citacao}

O trabalho de \citeonline{khan2012one}, assim como o apresentado nesta monografia, apresenta o uso de um AVA para auxiliar na educação. Dessa forma, esse trabalho servirá como referência para abordar 
o uso da aprendizagem auto-ritmada na educação matemática, assim como para definir as informações que deverão ser apresentadas aos tutores sobre a evolução da aprendizagem de seus estudantes na 
plataforma fruto deste trabalho. Contudo, os vídeos que fizeram da Khan Academy tão popular não farão parte desse trabalho.

O aspecto mais importante a se considerar aqui está na forma como as duas plataformas lidam com Obstáculos Epistemológicos. Segundo \citeonline{bachelard1996formaccao}, durante o ato do conhecimento, 
ocorrem ``lentidões e conflitos'' que levam o estudante a parar diante do problema. A esta ``inércia'' é que foi relacionado o conceito. Na metodologia criada por \citeonline{khan2012one}, quando a 
plataforma é aplicada dentro da sala de aula, o professor pode identificar através da ferramenta, os estudantes que estão com esses obstáculos e o mesmo pode intervir para ajudar esses estudantes a 
superar essa barreira. No trabalho apresentado aqui, essa barreira será superada quando o sistema, ao identificar o obstáculo enfrentado, indicar ao estudante a exist\^encia desse obst\'aculo e o(s) 
conte\'udo(s) que ele possui defici\^encia (causador(es) da barreira) para ele assim poder pausar o conte\'udo que est\'a estudando e voltar a estudar o(s) conte\'udo(s) mais b\'asicos que o sistema 
indicar. Por esse método o  estudante, ao enfrentar um obst\'aculo na aprendizagem, saber\'a quais os conte\'udos estudados anteriormente foram os causadores desse obst\'aculo e poder\'a, dessa 
forma, focar seus estudos nesses conte\'udos para superar essa barreira.

\subsubsection{\textit{ActiveMath: A generic and adaptive web-based learning environment}}

O projeto ActiveMath visa apoiar a aprendizagem verdadeiramente interativa, exploratória, e assume que o estudante deve ser responsável por seu aprendizado. Portanto, uma relativa liberdade para 
navegar através de um curso e para as escolhas de aprendizagem lhes é dada e, por padrão, o modelo de usuário é inspecionável e modificável \cite{melis2001activemath}. 
\citeonline{melis2001activemath} afirmam que a maioria dos sistemas tutores inteligentes não contam com uma escolha de adaptação de conteúdos e isso, segundo \citeonline{melis2001activemath}, pode 
influenciar quando o público-alvo for alunos de faculdades e universidades, já que, diferentemente das escolas de ensino fundamental,  um mesmo assunto é ensinado de forma diferente para diferentes 
grupos de utilizadores e em contextos diferentes \cite{melis2001activemath}.

Para conseguir um ambiente dinâmico de aprendizagem, ActiveMath utiliza regras pedagógicas que definem em quais momentos determinadas funcionalidades do sistema estarão disponíveis, em que ordem os 
conteúdos serão apresentados para os alunos e como os mesmos deverão ser apresentados. O trabalho referido,  assim como o desenvolvido por \citeonline{khan2012one} descrito anteriormente, descreve a 
criação de um AVA, o mesmo utiliza tamb\'em técnicas que permitem a geração dinâmica de cursos para os alunos.

O trabalho desenvolvido por \citeonline{melis2001activemath} apresenta um subsistema de exerc\'icios que suporta diagn\'osticos de erros e equ\'ivocos dos estudantes, que 
gera estrat\'egias tutoriais configur\'aveis para o \textit{feedback}. Em nosso trabalho, quando o ambiente diagnosticar erros e equ\'ivocos frequentes cometidos pelos estudantes relacionados a 
um conte\'udo estudado anteriormente, o sistema alertar\'a ao estudante sobre uma poss\'ivel defici\^encia que ele possua e o orientar\'a a voltar a estudar o conte\'udo indicado.


\subsubsection{\textit{Duolingo: Learn a Language for Free while Helping to Translate the Web}}\label{trabalho_relacionados_duolingo}

Nesse trabalho desenvolvido por \citeonline{von2013duolingo}, \'e apresentado o Duolingo, uma plataforma de ensino de idiomas e tradu\c{c}\~ao autom\'atica de documentos. O ambiente funciona de 
maneira que os usuários progridam nas lições ao mesmo tempo que traduzem conteúdo real da internet. 

O método utilizado pela plataforma se caracteriza pela li\c{c}\~oes fragmentadas, pelas quais os 
alunos, atrav\'es do m\'etodo de repeti\c{c}\~ao, fixam o conte\'udo da língua estudada. \`A medida que o usu\'ario avan\c{c}a, ele progride em uma \'arvore de habilidades que o leva gradativamente 
ao fim do curso, enquanto oferece a op\c{c}\~ao de voltar atr\'as para refazer li\c{c}\~oes antigas que j\'a poderiam ter sidas esquecidas. Um estudo realizado na Universidade da Cidade de Nova York 
\cite{vesselinov2012duolingo} disse que 34 horas gastas no Duolingo igualou-se a um semestre de um curso de l\'inguas.

Uma das características mais marcantes dessa plataforma \'e quantidade de técnicas de Gamifica\c{c}\~ao empregadas. Possui sistema de pontua\c{c}\~ao, n\'iveis, rankings, miss\~oes, medalhas, 
personaliza\c{c}\~ao, entre outras. Assim como a plataforma desenvolvida por \citeonline{von2013duolingo}, o ambiente que desenvolveremos tamb\'em contar\'a com t\'ecnicas de Gamifica\c{c}\~ao que 
ser\~ao utilizadas para melhorar a experi\^encia do usu\'ario, assim como para motivar esses usu\'arios durante sua utiliza\c{c}\~ao do sistema.

\subsubsection{\textit{Resolução de problemas em ambientes virtuais de aprendizagem num curso de licenciatura em matemática na modalidade a distância}}

O trabalho desenvolvido por \citeonline{dutra2011resoluccao}, trata da utilização da metodologia de Resolução de Problemas (apresentada na \autoref{resolucao_problemas}) em um AVA com o objetivo de 
investigar as contribuições que essa jun\c{c}\~ao pode trazer para um curso de Licenciatura em Matem\'atica da Universidade Federal de Ouro Preto (UFOP). Para isso, 
o ambiente desenvolvido por \citeonline{dutra2011resoluccao} utiliza de fóruns semanais para discussão e resolução dos problemas, além de chats, utilizados ao final de algumas atividades, e um 
questionário final a ser respondido pelos alunos na última semana de aula.

Essa metodologia, segundo \citeonline{dutra2011resoluccao}, funciona atrav\'es dos seguintes passos:
\begin{alineascomnumero}
	\item A atividade é postada na Plataforma Moodle\footnote{``A  palavra  Moodle  referia-se  originalmente  ao  acrônimo:  `Modular Object-Oriented  Dynamic  Learning  Environment'(...).  Em  inglês  a  palavra Moodle é também um verbo que descreve a ação que, com frequência, conduz a resultados criativos, de deambular com preguiça, enquanto se faz com gosto o  que  for  aparecendo  para  fazer''. O  Moodle  deu  o  nome  a  uma  plataforma  de  e-learning,  de  utilização livre  e  código  fonte  aberto,  pela  mão  de  Martin  Dougiamas \cite{oro29585}.} no início da semana, pela manhã (segunda-feira).  Assim,  as  atividades  são  distribuídas  aos  alunos  para  que possam ler, interpretar e entender o problema. 
	\item Os  alunos  passam a  semana  postando  suas  resoluções  dos  problemas e discutindo-as no fórum com os colegas, por meio da Plataforma Moodle. 
	\item A  pesquisadora  observa,  incentiva  e  participa  do  processo  de discussão, ajuda nos problemas secundários, dando \textit{feedback} das resoluções postadas, respondendo e fazendo 
perguntas, tirando  dúvidas, acompanhando de perto as discussões entre os alunos no fórum.
	\item As impressões dos alunos sobre os problemas, no início da semana seguinte, e a formalização dos resultados são apresentadas nos chats semanais. Trata-se  de  uma  plenária  virtual  para  discutir  os  problemas,  finalizando-a  com  uma solução aceita por todos. 
	\item Uma resolução é postada na Plataforma Moodle, para todos os pesquisados, observando os conteúdos apresentados nos problemas. 
\end{alineascomnumero}

O trabalho apresentado aqui, assim como o desenvolvido por \citeonline{dutra2011resoluccao}, utiliza a Metodologia de Resolu\c{c}\~ao de Problemas para auxiliar no ensino de matem\'atica. Uma 
outra característica que as duas metodologias t\^em em comum \'e o uso de f\'oruns de discuss\~ao como ferramenta que propicia uma 
constru\c{c}\~ao coletiva de conhecimento, atrav\'es do compartilhamento do conhecimento.


\section{Considerações finais do cap\'itulo}

O sistema que desenvolveremos faz uso da metodologia de resolução de problemas, possibilitando o aluno adquirir novos conhecimentos além daqueles estudados no momento. Por se tratar de um AVA, 
o mesmo possibilitará ao aluno uma maior interação com o professor e outros alunos, além da possibilidade de uma aprendizagem auto-ritmada. A aplicação da Gamificação no sistema, servir\'a para 
desenvolver  engajamento, participação, e comprometimento entre os usuários do sistema. 
	\chapter{PROCEDIMENTOS METODOLÓGICOS}
\label{chap:procedimentos-metodologicos}

Os procedimentos metodológicos definem as principais etapas realizadas para o desenvolvimento deste trabalho, incluindo as pesquisas, o desenvolvimento e avaliação do \textit{software}. Nas seções a 
seguir, descrevemos essas etapas.

\section{Definição do Processo}

Processos de \textit{software} são utilizados pelos engenheiros de \textit{software} para controlar e coordenar projetos de desenvolvimento de \textit{softwares} reais \cite{talma2006desenvolvimento}. 
\citeonline{padua2003engenharia} descreve um processo como um conjunto de passos parcialmente ordenados, constituídos por atividades, 
métodos, práticas e transformações, usados para atingir uma meta. 
Desta forma, um modelo de processo de \textit{software} é uma descrição simplificada de um processo, sendo também uma representação abstrata do mesmo para explicar as diferentes abordagens de 
desenvolvimento \cite{sommerville2003engenharia}. Tais processos de \textit{software} são complexos e dependem do julgamento humano como 
em qualquer processo intelectual. Sendo assim, não existe um processo de \textit{software} ideal, todos são desenvolvidos de maneiras 
diferentes por cada organização \cite{sommerville2003engenharia}.

O processo de \textit{software} utilizado para desenvolver o sistema foi baseado no modelo cascata, também chamado de ciclo de vida clássico, proposto por Royce em 1970. Neste modelo, as fases são 
sistematicamente seguidas de maneira sequencial \cite{pressman2006engenharia}. O modelo inicia com a fase de especificação de requisitos, passando pelo planejamento, modelagem, construção e 
implantação, finalizando na manutenção progressiva do software, como apresentamos na \autoref{fig:ciclo-cascata}.

As vantagens desse modelo se devem ao fato de que só se avança para a tarefa seguinte quando o cliente valida e aceita os produtos finais da tarefa atual, facilitando assim a compreensão adquirida ao 
longo do projeto, além de facilitar o processo de criação da documentação para o sistema \cite{pressman2006engenharia}. Já as principais desvantagens, segundo \citeonline{pressman2006engenharia}, se 
devem ao fato de que os projetos reais raramente seguem o fluxo sequencial ao qual o modelo propõe. Este modelo exige ainda que todos os requisitos sejam estabelecidos na fase 
inicial, fato que geralmente é difícil tanto para o cliente quanto para o desenvolvedor, pois os requisitos geralmente est\~ao em constante mudan\c{c}a. Outro grande problema com esse modelo é que o 
cliente só recebe uma versão executável do sistema no final de todo o processo de desenvolvimento, o que não agrada a muitos clientes.

Levando em consideração as vantagens e desvantagens antes citadas, esse modelo foi escolhido como base para o processo por facilitar o desenvolvimento de uma documentação mais detalhada e 
principalmente pela equipe de desenvolvimento ser formada por uma única pessoa, o autor deste trabalho, impossibilitando a divisão de tarefas característica de metodologias 
ágeis \footnote{Metodologias de desenvolvimento de software que tem enfoque nas pessoas e não em processos ou algoritmos, além de uma preocupação menor em documentação e maior em implementação 
\cite{michel2004metodologias}.}.

\begin{figure}[H]
    \centering
    \Caption{\label{fig:ciclo-cascata} Ciclo Cascata}	
    \UFCfig{}{
	\fbox{\includegraphics[width=7cm]{figuras/figura_ciclo_cascata}}
    }{
      \Fonte{\citeonline{pressman2006engenharia}}
    }	
\end{figure}


Dividimos o processo em nove atividades. A seguir, descreveremos brevemente cada uma delas:
\begin{alineascomnumero}
	\item \textbf{In\'icio do Projeto}: nessa atividade identificamos os objetivos do projeto e suas restri\c{c}\~oes.
	\item \textbf{Requisitos}: nessa atividade realizamos o levantamento e an\'alise dos requisitos, sua documenta\c{c}\~ao, verifica\c{c}\~ao e valida\c{c}\~ao, al\'em de um estudo para saber 
atrav\'es dos requisitos se o projeto \'e vi\'avel. 
	\item \textbf{Projeto}: nessa atividade desenvolvemos o projeto da arquitetura do sistema, o planejamento dos m\'odulos, o projeto da interface gr\'afica com o usu\'ario  e como ser\~ao 
persistidos as entidades do sistema.
	\item \textbf{Implementação}: nessa atividade escolhemos primeiramente um m\'odulo do sistema para implementar, em seguida planejamos como ocorrer\'a a implementa\c{c}\~ao desse 
m\'odulo, implementamos o m\'odulo, e por \'ultimo realizamos a integra\c{c}\~ao de m\'odulo ao resto do sistema. Esse ciclo \'e repetido at\'e que todos os m\'odulos estejam desenvolvidos.
	\item \textbf{Testes}: durante essa atividade realizaremos o planejamento dos testes para o sistema, em seguida executaremos os testes 
para cada m\'odulo do sistema e para o sistema como um todo, e por \'ultimo iremos gerar um relat\'orio com os resultados dos testes.
	\item \textbf{Implantação}: ap\'os o sistema desenvolvido e devidamente testado, ele segue para a implanta\c{c}\~ao, em que prepararemos 
a plataforma onde o sistema ser\'a executado e realizaremos a implanta\c{c}\~ao. Ap\'os a implanta\c{c}\~ao, realizaremos testes de 
aceita\c{c}\~ao e a cria\c{c}\~ao de um manual para o sistema.
	\item \textbf{Gerenciamento do Projeto}: essa atividade ocorre durante todo o desenvolvimento do sistema e tem por objetivo permitir que o projeto consiga ser conclu\'ido dentro do prazo e da 
qualidade desejada.
	\item \textbf{Avalia\c{c}\~ao do Processo}: essa atividade \'e executada durante todo o desenvolvimento do projeto e permite que o processo utilizado seja constantemente avaliado e adaptado 
conforme segue o desenvolvimento do projeto.
	\item \textbf{Encerramento do Projeto}: nessa \'ultima atividade ser\'a realizado apenas uma reuni\~ao em que ser\'a oficializado o final do projeto.
\end{alineascomnumero}

O processo desenvolvido est\'a dispon\'ivel em \url{www.askmath.quixada.ufc.br/static/process/}. 

\section{Levantamento e Análise de Requisitos}


O Levantamento e An\'alise de Requisitos é a fase do desenvolvimento de um \textit{software} em que o analista verifica junto ao usuário 
quais as necessidades, condições e princípios que o \textit{software} deverá atender \cite{matuda2013mapas}. Essa fase possibilitou 
conhecer e estudar as necessidades do cliente, assim como as restrições que o software estará sujeito.

Para realizar a coleta dos requisitos, optamos por utilizar entrevistas com o cliente, que no caso foi um dos professores de matemática da \gls{ufc}. Nessas entrevistas, que se caracterizaram como semi-estruturadas \footnote{Pois foram guiadas por um 
roteiro previamente elaborado e composto por questões abertas \cite{belei2008uso}}, foi possível obter os requisitos do sistema, assim como 
o público-alvo a quem o sistema atenderá. Essa técnica foi utilizada porque permitia uma organização flexível e ampliação dos 
questionamentos à medida que as informações foram sendo fornecidas pelo cliente \cite{fujisawa2000utilizaccao}.

A plataforma atender\'a quatro tipos de usu\'arios que formaram seu p\'ublico-alvo, ser\~ao administradores, professores, assistentes, e estudantes. Os administradores ser\~ao respons\'aveis por 
cadastrar novos professores, assim como realizar a manuten\c{c}\~ao do sistema. Os professores ser\~ao responsáveis por definir 
as disciplinas e li\c{c}\~oes que far\~ao parte do sistema. 
Ap\'os as disciplinas e li\c{c}\~oes serem definidas pelos professores, os assistentes ser\~ao respons\'aveis por manter\footnote{Adicionar, editar e excluir os problemas.} os problemas de cada 
li\c{c}\~ao. Os estudantes resolver\~ao os problemas desenvolvidos pelos assistentes e poder\~ao postar d\'uvidas sobre qualquer conte\'udo 
no f\'orum de discuss\~oes. Quando a plataforma for aplicado em sala de aula, os professores ser\~ao ainda respons\'aveis por gerenciar suas 
turmas e acompanhar o andamento do progresso de cada um de seus estudantes. J\'a os assistentes, ser\~ao responsáveis por tirar d\'uvidas 
dos estudantes no f\'orum de discuss\~oes.

Para uma melhor compreensão do público-alvo, foram criadas Personas \cite{pruitt2003personas}, personagens fictícios usados para 
caracterizar os papéis dos diferentes usuários do sistema \cite{guerra2010colaboraccao}. Cada Persona criada possui um nome, hábitos, 
histórias pessoais, motivações, objetivos, entre outras (ver \autoref{ap:personas}). A escolha dessa técnica deu-se pelo fato de que ela 
permite ao desenvolvedor saber mais precisamente para 
quem ele deveria construir o sistema, além de permitir uma distinção maior do público-alvo e, dessa forma, aprofundar-se nos interesses individuais de cada um.

Apresentaremos a seguir, os principais requisitos levantados durante essa etapa:

\begin{alineascomponto}
	\item Hierarquia dos Conteúdos: no sistema, dever\~ao existir disciplinas. Cada disciplina dever\'a possuir li\c{c}\~oes e cada li\c{c}\~ao ser\'a formada por problemas. Os professores poder\~ao 
cadastrar v\'arias disciplinas e li\c{c}\~oes, j\'a os assistentes ficar\~ao responsáveis por adicionar os problemas nas li\c{c}\~oes. Quando o estudante entrar no sistema, ele poder\'a ver somente 
as li\c{c}\~oes de uma disciplina por vez, podendo alternar entre disciplinas. Para um estudante come\c{c}ar a responder os problemas, ele dever\'a escolher inicialmente a disciplina e em seguida a 
li\c{c}\~ao. 

	\item Estrutura do Problema: cada problema possuir\'a uma descri\c{c}\~ao e v\'arios itens. Os problemas ser\~ao somente de m\'ultipla escolha e poder\~ao possuir no m\'inimo dois e no m\'aximo 
cinco itens. A quantidade de itens em cada problema ficar\'a a crit\'erio do assistente que adicionar\'a o problema ao sistema.

	\item Saltar Problemas: o sistema deverá permitir ao aluno saltar problemas e rever os saltos 
realizados com algumas restrições na quantidade de saltos..
	
	\item Pedir Ajuda: para todo problema, o estudante poderá consultar conte\'udo de ajuda. No momento em que o assistente estiver adicionando o problema, ele poderá ou não adicionar um texto de 
ajuda para o estudante, isso fica a critério dele. Quando o estudante estiver resolvendo os problemas, ele terá um botão que, quando acionado, mostrar\'a o texto de ajuda. O sistema dever\'a salvar a 
quantidade 
de vezes que o estudante pedir ajuda e em quais problemas.
	\item F\'orum: o sistema deverá possuir um fórum onde os estudantes possam postar suas dúvidas para que professores, assistentes e outro participantes possam lhes ajudar com o problema. Esse fórum, deve possuir tópicos, comentários para os tópicos, e os comentários devem oferecer a opção de se comentar com imagens.     

\end{alineascomponto}

Após o levantamento dos requisitos, foi realizada a análise dos mesmos. Nessa análise, os requisitos foram agrupados em categorias. As categorias utilizadas são descritas por 
\citeonline{sommerville2003engenharia} como:
\begin{alineascomponto}
    \item Requisitos Funcionais -- especificam ações que um sistema deve ser
capaz de executar, sem levar em consideração restrições físicas. Os requisitos
funcionais especificam, portanto, o comportamento de entrada e saída de um
sistema.
    \item Requisitos Não Funcionais -- descrevem apenas atributos do sistema ou
atributos do ambiente do sistema, como segurança, desempenho, usabilidade e
confiabilidade.
    \item Requisitos de Domínio -- são os requisitos do domínio da aplicação do sistema e que refletem características desse domínio.
\end{alineascomponto}

Após esse agrupamento, os requisitos funcionais foram representados em Casos de Uso \cite{jacobson92engenharia}. Um caso de uso identifica os agentes envolvidos em uma interação e especifica o tipo  de interação utilizando anotações sugeridas pela 
\gls{uml} \citeonline{sommerville2003engenharia}. Em seguida, foi realizada a documentação dos requisitos (ver \autoref{ap:requisitos}).

No final dessa etapa, ocorreu a Validação dos Requisitos junto ao cliente.  A Validação dos Requisitos é definida como o processo que certifica que o modelo de requisitos gerado  esteja  consistente  
com  as  necessidades  e  intenções  de  clientes  e usuários \cite{rilston2003metodologia}. Esta etapa permitiu que os requisitos coletados e documentados estivesse de acordo com o que o 
cliente solicitou.

\section{Projeto do Sistema}

O Projeto de Software é à atividade de engenharia cujo foco é definir ``como'' os requisitos estabelecidos no projeto devem ser implementados no software \cite{pressman2006engenharia}. O objetivo da  
atividade de projetar é gerar um modelo ou representação que apresente solidez, comodidade e deleite \cite{pressman2006engenharia}. Nesta fase, definimos como será aplicado o conhecimento obtido na 
pesquisa bibliográfica para se desenvolver o sistema. Para isto, definimos a arquitetura de software e as ferramentas que serão utilizadas durante o desenvolvimento do sistema. Nas seções a seguir, 
descrevemos um pouco sob cada um.

\subsection{Arquitetura}
Por arquitetura de software, entende-se a estrutura ou a organização de componentes de módulos, a maneira através da qual esses componentes interagem e a estrutura de dados que será usada pelos componentes \cite{pressman2006engenharia}.

A arquitetura utilizada baseia-se na arquitetura Cliente-Servidor\cite{david2013everything}, em que o processamento é dividido em processos distintos. Um processo é responsável pela manutenção da 
informação (servidor) e os outros são responsáveis pela captação de dados (clientes). Nessa arquitetura, os clientes enviam pedidos para o servidor, e este por sua vez processa estes dados e envia as 
respostas dos pedidos aos clientes.

Este modelo de arquitetura facilitará na manutenção do sistema, tendo em vista que toda atualização só necessitará ser realizada no servidor e automaticamente a mesma se propagará para todos os 
clientes. Com todos os recursos centralizados no servidor, podemos também ter um maior ganho com segurança, já que podemos centralizar os nossos esforços para manter a segurança das informações em 
apenas um único ponto, além de possibilitar que apenas cliente credenciados possam acessar e/ou alterar essas informações. Uma das outras grandes vantagens que temos  ao utilizar esse modelo, é que, 
\`a medida que a quantidade de clientes aumente, será possível suprir esses clientes sem necessitar realizar nenhuma modificação essencial.

Para uma visualização mais detalhada da arquitetura de \textit{software} definida, ver o  \autoref{ap:arquitetura}.

\subsection{Ferramentas}

A análise do sistema foi feita com o auxílio da ferramenta de criação de diagramas Astah \cite{astah2016}. A implementação com a linguagem 
Python \cite{vanrossum2010python}, o sistema de gerenciamento de banco de dados PostgresSQl \cite{momjian2001postgresql} e a camada de 
aplicação utilizar\'a o framework Django \cite{django2016}. O módulo Rosseta \cite{rosetta2016} ser\'a empregado 
para permitir a internacionalização do sistema.
 
A seguir, apresentaremos a lista das ferramentas e das tecnologias utilizadas para o desenvolvimento do projeto:

\begin{alineas}
	\item \textbf{Astah}: para a modelagem baseada em \acrshort{uml} do sistema.
	\item \textbf{Python}: linguagem de programação para implementação do sistema.
    \item \textbf{Django}: framework web responsável pela camada de aplicação.
    \item \textbf{Rosetta}: aplicação desenvolvida em Django que facilitará a tradução do projeto para diversas línguas.  
    \item \textbf{PostgreSQL}: como banco de dados para armazenamento e consulta de informações.
    \item \textbf{Metro UI CSS}: framework que faz uso de HTML, Cascading Stype Sheet (CSS) e Javascript para criação do front-end do sistema.
    \item \textbf{MathJax}: \'e uma engine\footnote{Uma biblioteca ou pacote de funcionalidades que são utilizadas para facilitar o desenvolvimento de alguma tecnologia.} de código aberto 
desenvolvido em 
javascript na forma de um plugin para incluir equações matemáticas em todos os navegadores, esse plugin aceita expressões em  MathML e Latex.

\end{alineas}

Algumas dessas ferramentas foram selecionadas por se tratarem de ferramentas \textit{Open Source}, ou seja, que seu código-fonte pode ser 
alterado para diferentes fins, possibilitando assim que qualquer um consulte, examine ou as modifique, e outras por serem ferramentas que 
possibilitam um rápido desenvolvimento.

No final dessa etapa, foi gerado o Plano de Projeto, documento que guiou o desenvolvedor durante todo o processo de desenvolvimento.

\section{Implementação do Sistema}

A implementação envolve as atividades de codificação, compilação e integração. A codificação visa traduzir o design num programa, utilizando linguagens e  ferramentas adequadas. A codificação 
deve refletir a estrutura e o comportamento descrito no projeto. Os componentes arquiteturais devem ser codificados de forma independente e depois integrados \cite{aguiar2012requisitos}.

O ambiente de desenvolvimento que est\'a sendo utilizado para desenvolver o sistema consiste em:
\begin{alineas}
	\item \textbf{Sistema Operacional}: Ubuntu 14.04 LTS;
	\item \textbf{Interpretador}: Python 2.7.6;
	\item \textbf{Banco de Dados}: PostgreSQL 9.3.13;
	\item \textbf{\textit{Framework} de \textit{back-end}}: Django 1.7.11;
	\item \textbf{\textit{Framework} de \textit{front-end}}: Metro UI CSS 3.0.15;
\end{alineas}

Para permitir uma maior independ\^encia desse ambiente com a maquina em que estamos desenvolvendo o sistema, decidimos criar um ambiente de desenvolvimento com o auxilio da 
ferramenta Virtualenv 15.0.2. O Virtualenv permite isolar o ambiente de desenvolvimento de um determinado projeto para garantir que um pacote instalado não interfira nos demais projetos que 
rodam na maquina, al\'em de possibilitar saber facilmente a vers\~ao de todos os pacotes e componentes que est\~ao sendo utilizados para desenvolver o sistema. Essa informa\c{c}\~ao \'e 
importante quando formos replicar o ambiente de desenvolvimento no servidor de produ\c{c}\~ao\footnote{O ambiente de produção é onde os usuários finais acessarão o software}. 

Todos os c\'odigos-fonte gerados nessa etapa, passaram por um processo de versionamento\footnote{O versionamento de uma aplicação tem como 
foco principal documentar as inclusões, alterações ou até mesmo exclusões de funcionalidades.} atrav\'es da ferramenta Git 1.9.1, 
essa ferramenta permite:

\begin{alineascomponto}
	\item \textbf{Controle do histórico}: facilidade em desfazer e analisar o histórico do desenvolvimento, como também facilidade no resgate de versões mais antigas e estáveis.
	\item \textbf{Marca\c{c}\~ao e Resgate de Vers\~oes Est\'aveis}: permite marca onde um documento estava com uma vers\~ao est\'avel, para ser facilmente resgatado no futuro.
	\item \textbf{Ramifica\c{c}\~ao do Projeto}: permite dividir o projeto em v\'arias linhas de desenvolvimento sem que uma interfira na outra. 
\end{alineascomponto}

No final dessa etapa, tivemos os códigos-fonte de todos os módulos do sistema, assim como o próprio sistema em funcionamento. 

\section{Verificação e Validação}
Essa etapa destinou-se a mostrar que o sistema está de acordo com a especificação e que ele atende às expectativas de clientes e usuários, al\'em de assegurar que o  programa está fazendo aquilo que foi definido na sua especificação e que não possui  erros  de execução \cite{aguiar2012requisitos}. 

Durante essa fase, realizamos Testes de Unidade e Integra\c{c}\~ao em cada modulo do sistema, assim como Testes de Sistema no sistema j\'a integrado.

\citeonline{aniche2014teste} define essas categorias de Testes de Software como:

\begin{alineascomponto}
	\item \textbf{Teste de Unidade} -- é aquele que testa uma única unidade do sistema. Ele a testa de maneira isolada, geralmente simulando as 
prováveis dependências que aquela unidade tem. Em sistemas orientados a objetos, é comum que a unidade seja uma classe. Ou seja, quando 
queremos escrever testes de unidade para a classe Pedido, essa bateria de testes testará o funcionamento da classe Pedido, 
isolada, sem interações com outras classes.
	\item \textbf{Teste de Integração} -- é aquele que testa a integração entre duas partes do seu sistema. Os testes que você escreve para a sua 
classe PedidoDao, por exemplo, em que seu teste vai até o banco de dados, é um teste de integração. Afinal, você está testando a integração 
do seu sistema com o sistema externo, que é o banco de dados. Testes que garantem que suas classes comunicam-se bem 
com serviços web, escrevem arquivos texto, ou mesmo mandam mensagens via \textit{socket} são considerados testes de integração.
	\item \textbf{Teste de Sistema} -- \'e aquele que garante que o sistema funciona como um todo. Este 
nível de teste está interessado se o sistema funciona como um todo, com todas as 
unidades trabalhando juntas. Ele é comumente chamado de teste de caixa preta, já 
que durante o teste n\~ao se importa com a forma que os dados s\~ao processados, mas sim se as sa\'idas est\~ao de acordo com o que era esperado para entradas. 
\end{alineascomponto}

\section{Definição dos Conteúdo para o Sistema}

Após o sistema estar verificado e validado, tivemos que adicionar conteúdos para serem utilizados por usu\'arios durante os testes da \textit{interface}. Decidimos optar por deixar os monitores\footnote{É o aluno de graduação concursado para exercer, juntamente com o professor, atividades técnico-didáticas condizentes com o seu grau de conhecimento junto à determinada disciplina, já por ele cursada.} das turmas de matemática desenvolverem os conteúdos que foram utilizados no sistema durante a fase de avaliação. 

Os monitores passaram por um treinamento, no qual aprenderam a utilizar o sistema para adicionarem os conteúdos.
	\chapter{AVALIA\c{C}\~AO DO AMBIENTE VIRTUAL DE APRENDIZAGEM}
\label{chap:avaliacao-ambiente-virtual-aprendizagem}
\setcounter{table}{0}

A interface permite o contato dos usu\'arios com o sistema, a qualidade dessa interface est\'a intimamente ligada com o grau de 
satisfa\c{c}\~ao que os usu\'ario venham a ter sobre o sistema. Projetar sistemas adequados ao uso e agrad\'aveis de se utilizar
melhorar\'a a efici\^encia de uso do usu\'ario e agregará mais qualidade ao produto final \cite{antonino2015avaliacao}. 

\citeonline{antonino2015avaliacao} acredita que a utiliza\c{c}\~ao de interfaces que dificultam a execu\c{c}\~ao de tarefas 
pode ser frustrante para o usu\'ario e pode fazer com que ele busque outras op\c{c}\~oes de softwares ou at\'e mesmo crie uma 
rejei\c{c}\~ao ao sistema. 

A qualidade da interface \'e verificada atrav\'es da avalia\c{c}\~ao da usabilidade. Essa técnica utiliza m\'etodos, t\'ecnicas e 
ferramentas, para checar a conformidade de um sistema com os crit\'erios de usabilidade para encontrar problemas de intera\c{c}\~ao e 
corrigi-los \cite{antonino2015avaliacao}. 

\citeonline{nielsen1993usability} divide os critérios básicos de usabilidade em cinco:

\begin{alineascomponto}
	\item \textbf{Facilidade de aprendizado}: O sistema precisa ser fácil de aprender para que o usuário possa explorar de forma agradável as 
funcionalidades do sistema;
	\item \textbf{Eficiência}: O sistema deve ser o mais produtivo possível, possibilitando a execução de tarefas de forma ágil, e sem muito esforço;
	\item \textbf{Memorização}: Mesmo depois de um longo período sem utilizar o sistema, ao retornar, o usuário deve sentir facilidade de uso, sem 
precisar reaprender a usá-lo novamente;
	\item \textbf{Erros}: O sistema precisa ter a menor taxa de erros possível, e caso ocorram é preciso apresentar maneiras simples e rápidas de 
recuperá-los;
	\item \textbf{Satisfação}: A satisfação do usuário determinará o sucesso ou não do sistema. Por isso, quanto mais agradável e prazerosa for a experiência dele com o sistema, melhor será o êxito do produto de software.
\end{alineascomponto}

Problemas relacionados a esses crit\'erios de usabilidade, podem dificultar a intera\c{c}\~ao do usu\'ario com o sistema e diminuir a 
qualidade em seu uso. Pensando nisso, realizamos uma avalia\c{c}\~ao de usabilidade no sistema. \citeonline{lima2006ambientecolaborativo} acredita que a avalia\c{c}\~ao de usabilidade tem como objetivos gerais: Validar a eficácia da interação humano-computador face a efetiva realização das tarefas por parte dos usuários; verificar a eficiência desta interação, face os recursos empregados (tempo, quantidade de incidentes, passos desnecessários, 
busca de ajuda, etc.); e obter indícios da satisfação ou insatisfação (efeito subjetivo) que ela possa trazer ao usuário.

A avaliação da plataforma foi realizada seguindo as seguintes etapas:

\begin{alineascomnumero}
	\item \textbf{Definição do Escopo da Avaliação}: Foram definidos os módulos e interfaces do sistema que seriam objeto da avaliação.
	\item \textbf{Planejamento}: Foram definidas as tarefas que seriam realizadas pelos usuários, assim como as métricas da avaliação, o roteiro da entrevista e a realização de um teste piloto\footnote{Teste realizado preliminarmente em uma escala menor de abrangência, ou seja, menor número de entrevistas, que servirá como orientação para realização da pesquisa propriamente dita, uma vez que fornecerá as devidas correções de questionário, amostra etc.}. 
	\item \textbf{Execução}: Consistiu na utilização de um um método de observação aliado a um método de investigação. Como método de observação, tivemos Testes de Usabilidade, já para o método de investigação contamos com estrevistas.
	\item \textbf{An\'alise dos Resultados}: Foi realizada com o apoio de Gravações de Áudio e Tela obtidas durante a execução dos testes com os usuários.
\end{alineascomnumero}

\section{Definição do Escopo da Avalia\c{c}\~ao}
Como qualquer outro AVA, a plataforma desenvolvida é formada por um conjunto de módulos. Os módulos avaliados são os responsáveis por apresentar as funcionalidades que são acionadas pelos alunos, como visualizar disciplinas, visualizar lições, responder questões, saltar questões, entre várias outras. A seguir, apresentamos o mapa da plataforma com os módulos que fizeram parte do escopo da avaliação:

\begin{figure}[H]
    \centering
    \Caption{\label{fig:mapa_ambiente} Mapa da Plataforma}	
    \UFCfig{}{
	\fbox{\includegraphics[width=10cm]{figuras/askmath/road_map.png}}
    }{
      \Fonte{Do Próprio Autor}
    }	
\end{figure}

\section{Planejamento}

Durante o planejamento, definimos todos os elementos do plano do teste de usabilidade, tais como os objetivos, métricas, tarefas, entre outros. A seguir, apresentaremos as etapas do planejamento da avaliação.

\subsection{Métricas da Avaliação}

O principal foco dessa avaliação é a usabilidade, por isso, avaliamos alguns fatores
como a facilidade de aprendizado, eficiência no uso, satisfação do usuário, segurança no uso e facilidade de memorização. Na \autoref{tab:metricas_avaliacao}, apresentamos as métricas que foram consideradas durante a avaliação:

\begin{table}[H]
	\Caption{\label{tab:metricas_avaliacao} Métricas para a avaliação}%
	\IBGEtab{}{%

\resizebox{\textwidth}{!}{

\begin{tabular}{|l|l|l|l|l|l|}
\hline
\multicolumn{1}{|c|}{\textbf{Id}} & \multicolumn{1}{c|}{\textbf{Fator}} & \multicolumn{1}{c|}{\textbf{Método de Medição}} & \multicolumn{1}{c|}{\textbf{Nível Ruim}} & \multicolumn{1}{c|}{\textbf{Nível Almejado}} & \multicolumn{1}{c|}{\textbf{Nível Ótimo}} \\ \hline
M1 & Eficiência no uso & \begin{tabular}[c]{@{}l@{}}Tempo gasto para realizar \\ uma tarefa\end{tabular} & \multicolumn{1}{c|}{--} & \multicolumn{1}{c|}{--} & \multicolumn{1}{c|}{--} \\ \hline
M2 & Segurança no uso & \begin{tabular}[c]{@{}l@{}}Número de erros cometidos\\ para cada tarefa\end{tabular} & Por volta de 2 erros & Nenhum erro & Nenhum erro \\ \hline
M3 & Eficiência no uso & \begin{tabular}[c]{@{}l@{}}Número de cliques para \\ realizar uma tarefa\end{tabular} & Por volta de 6 cliques & No máximo 3 cliques & \multicolumn{1}{c|}{--} \\ \hline
M4 & Facilidade de aprendizado & \begin{tabular}[c]{@{}l@{}}Se entende corretamente as \\ informações da interface\end{tabular} & Por volta de 3 perguntas & Sem perguntas & Sem perguntas \\ \hline
M5 & Satisfação do usuário & \begin{tabular}[c]{@{}l@{}}Avaliação do usuário através\\ da entrevista(subjetivo)\end{tabular} & neutro & positivo & muito positivo \\ \hline
\end{tabular}

}

	}{%
	\Fonte{Do Próprio Autor}%
	%\Nota{esta é uma nota, que diz que os dados são baseados na
	%	regressão linear.}%
	%\Nota[Anotações]{uma anotação adicional, seguida de várias outras.}%
}
\end{table}


\subsection{Tarefas}

Após termos o escopo da avaliação e as métricas que seriam utilizadas para avaliar, definimos as tarefas que deveriam ser realizadas pelo usuário durante a avaliação. Confira na tabela a seguir.

\begin{table}[H]
	\Caption{\label{tab:metricas_avaliacao} Tarefas para a avaliação}%
	\IBGEtab{}{%

\resizebox{\textwidth}{!}{

\begin{tabular}{|l|l|}
\hline
\multicolumn{1}{|c|}{\textbf{Id}} & \multicolumn{1}{c|}{\textbf{Tarefa}} \\ \hline
T1 & \begin{tabular}[c]{@{}l@{}}Imagine que você está à procura de uma ferramenta para praticar um pouco dos seus conhecimentos em matemática e \\ acaba de conhecer o AskMath. Para utilizar o sistema pela primeira vez, você deve realizar o seu cadastro.\end{tabular} \\ \hline
T2 & Com o acesso ao sistema, você deve buscar a lição denominada "Introdução a Proposições". \\ \hline
T3 & Após encontrar a lição "Introdução a Proposições", você deve começar a responder as questões apresentadas. \\ \hline
T4 & \begin{tabular}[c]{@{}l@{}}Após resolver algumas questões da lição "Introdução a Proposições", você deve procurar pela disciplina chamada \\ "Diferença".\end{tabular} \\ \hline
T5 & Realize sua inscrição em alguma Turma de Matemática. \\ \hline
T6 & \begin{tabular}[c]{@{}l@{}}Suponha que você ficou com dúvida em algum conteúdo e deseja pedir ajuda. Sendo assim, você deve postar essa \\ dúvida no Fórum de discussões.\end{tabular} \\ \hline
\end{tabular}

}

	}{%
	\Fonte{Do Próprio Autor}%
	%\Nota{esta é uma nota, que diz que os dados são baseados na
	%	regressão linear.}%
	%\Nota[Anotações]{uma anotação adicional, seguida de várias outras.}%
}
\end{table}

Essas tarefas foram escolhidas, pois permitiam o usuário explorar todas as principais funcionalidades do sistemas, assim como suas interfaces. 

\subsection{Roteiro da Entrevista}
Entrevista é uma das técnicas mais utilizadas de coleta de dados e levantamento de requisitos. Trata-se de uma conversa guiada por roteiro de perguntas ou tópicos, na qual o entrevistador busca obter informação de um entrevistado \cite{seidman1998interview}. Existem três tipos de entrevistas:
\begin{itemize}
	\item \textbf{Estruturadas}: o entrevistador se mantém fiel a um roteiro, fazendo perguntas previamente definidas na ordem especificada. Ele não tem muita liberdade para explorar tópicos novos que surjam durante a entrevista.
	\item \textbf{Não estruturada}:  o entrevistador realiza pergunta de modo flexível, usando perguntas abertas e aprofundando mais em alguns tópicos. O comprometimento do entrevistador é com o tópico abordado.
	\item \textbf{Entrevista semiestruturada}: o roteiro é composto de tópicos ou perguntas (geralmente abertas) que devem ser endereçados na entrevista, em uma ordem lógica. O entrevistador tem liberdade para explorar as respostas e até mesmo modificar a ordem dos tópicos, mas deve manter o foco nos objetivos da entrevista. 
\end{itemize}

Para a avaliação, optamos por utilizar uma estrevista semiestruturada,  por possibilitar uma maior flexibilidade e também uma maior exploração do assunto estudado. Na \autoref{tab:roteiro_entrevista}, apresentamos o roteiro que nos guiou durante as estrevistas.


\begin{table}[H]
	\Caption{\label{tab:roteiro_entrevista} Roteiro para a entrevista}%
	\IBGEtab{}{%

\resizebox{\textwidth}{!}{

\begin{tabular}{|l|l|}
\hline
\multicolumn{1}{|c|}{\textbf{Finalidade}} & \multicolumn{1}{c|}{\textbf{Pergunta}} \\ \hline
\multirow{2}{*}{Caracterização} & Você utiliza regularmente computador? \\ \cline{2-2} 
 & Você já usou algum outro sistema parecido com este? \\ \hline
\multirow{3}{*}{Funcionalidade} & Você acha que o sistema faz o que ele se propõe a fazer? \\ \cline{2-2} 
 & O que você acha das funções do sistema? \\ \cline{2-2} 
 & Tem alguma função que você adicionaria no sistema? \\ \hline
\multirow{2}{*}{Confiabilidade} & Em algum momento o sistema apresentou alguma falha? \\ \cline{2-2} 
 & \begin{tabular}[c]{@{}l@{}}(Caso tenha ocorrido falha). \\ Você acha que o sistema agiu adequadamente na ocorrência da falha?\end{tabular} \\ \hline
\multirow{4}{*}{Usabilidade} & O que você acha em relação a aprendizagem do uso do sistema? \\ \cline{2-2} 
 & O que você achou do formato dos formulários do sistema? \\ \cline{2-2} 
 & O que você acha das respostas que o sistema dá quando você executa uma ação? \\ \hline
\multirow{2}{*}{Eficiência} & O que você acha do tempo de resposta do sistema? \\ \cline{2-2} 
 & É fácil encontrar o que se procura no sistema? \\ \hline
\multirow{3}{*}{Melhorias} & O que você não gostou no sistema? \\ \cline{2-2} 
 & O que você gostou no sistema? \\ \cline{2-2} 
 & Qual sugestão de melhoria você daria para o sistema? \\ \hline
\end{tabular}

}

	}{%
	\Fonte{Do Próprio Autor}%
	%\Nota{esta é uma nota, que diz que os dados são baseados na
	%	regressão linear.}%
	%\Nota[Anotações]{uma anotação adicional, seguida de várias outras.}%
}
\end{table}

\subsection{Teste Piloto}

O objetivo do Teste Piloto foi identificar problemas ocasionados por ambiguidades e erros nos materiais de avaliação que poderiam prejudicar a avaliação. Após esse teste, foram realizados correções no roteiro da entrevista e nas tarefas para o teste de usabilidade.

\section{Execução}

Como a abordagem é qualitativa, foram feitos poucos testes, pois o objetivo não é chegar a resultados estatisticamente válidos, mas sim obter evidências ou indicações de problemas e sugestões de como melhorar a qualidade de uso da interface. Tipicamente em testes com usuários se envolve de 5 a 12 usuários \cite{dumas1999practical}. Em nossa avaliação, participaram 6 usuários, que foram escolhidos conforme a definição do público-alvo da plataforma.

Para facilitar a coleta e análise dos dados, gravamos as entrevistas com o auxílio de um gravador e a própria interação do usuário com o sistema através de \textit{softwares} especializados. Ressaltamos que isso só foi possível pois os participantes nos concederam, através de um termo de consentimento livre e esclarecido (\autoref{ap:tcle}), permissão para as gravações. Desde que fosse mantida total confidencialidade dos dados gravados, servindo apenas para facilitar sua análise.

A execução do teste consistiu nas seguintes etapas:
\begin{enumerate}
	\item Primeiramente, cada participante foi cumprimentado pelo avaliador e orientado a sentar-se e ficar à vontade. Foi realizado algumas rápidas perguntas curtas para verificar o perfil do participante.
	\item Em seguinda, o avaliador apresentou o propósito e objetivos da avaliação.   
	\item Após as devidas explicações, foi iniciada a realização das tarefas, as quais foram descritas oralmente pelo avaliador. Nesta etapa, o avaliador requisitou que o participante verbalizasse suas dúvidas para ajudá-lo a coletar essas informações.
	\item Depois de concluídas todas as tarefas, o participante foi submetido a uma entrevista, onde o avaliador veio a tratar, além de alguns questionamentos pré-definidos, fatos ocorridos durante a realização das tarefas.
	\item Após a entrevista, o entrevistador agradeceu ao participante e lhe entregou um brinde como forma de agradecimento.  
\end{enumerate}

\section{An\'alise dos Resultados}

\subsection{Tempo de execução das tarefas}
Na \autoref{tab:execucao_tarefas} apresentamos as medidas de tempo de execução das seis tarefas realizadas comparando-as com valores previamente estabelecidos: nível ruim, almeijado e ótimo.


\begin{table}[H]
	\Caption{\label{tab:execucao_tarefas}  Tempo de Execução das Tarefas}%
	\IBGEtab{}{%

\resizebox{\textwidth-8cm}{!}{

\begin{tabular}{|l|c|c|c|c|c|c|}
\hline
\multicolumn{1}{|c|}{\textbf{\begin{tabular}[c]{@{}c@{}}Tarefas/\\ Participantes\end{tabular}}} & \textbf{T1} & \textbf{T2} & \textbf{T3} & \textbf{T4} & \textbf{T5} & \textbf{T6} \\ \hline
Participante 1 & 103s & 20s & - & 5s & 5s & 47s \\ \hline
Participante 2 & 126s & 9s & - & 175s & 7s & 34s \\ \hline
Participante 3 & 78s & 10s & - & 35s & 5s & 55s \\ \hline
Participante 4 & 48s & 5s & - & 30s & 5s & 40s \\ \hline
Participante 5 & 90s & 5s & - & 11s & 4s & 35s \\ \hline
Participante 6 & 35s & 9s & - & 20s & 8s & 25s \\ \hline
\textbf{Média} & \textbf{80s} & \textbf{9s} & \textbf{-} & \textbf{46s} & \textbf{5s} & \textbf{39s} \\ \hline
Nível ruim & 120 & 30s & - & 100s & 15s & 60s \\ \hline
Nível almeijado & 45s & 5s & - & 10s & 5s & 30s \\ \hline
Nível ótimo & 20s & 2s & - & 4s & 3s & 15s \\ \hline
\end{tabular}

}
}{%
	\Fonte{Do Próprio Autor}%
	\Nota{Os dados estão expressos em segundos}
	\Nota{Não é correto calcular o tempo de realização da atividade T3, já que isso depende do nível de conhecimento dos participantes com o conteúdo apresentado}
	%\Nota[Anotações]{uma anotação adicional, seguida de várias outras.}%
}
\end{table}

Em geral, o tempo de execução das tarefas foi maior que o nível almeijado, isso pode indicar que a eficiência no uso do sistema pode está comprometida. Em especial, nas tarefas T1 e T4, o tempo para execução dessas tarefas são bem maiores do que o nível almeijado, em alguns casos, até o dobro do tempo. Com isso, podemos concluir que a eficiência de uso para realizar um cadastro e buscar uma lição específica está abaixo do nível desejado.  

\subsection{Número de erros cometidos na execução das tarefas}
Na \autoref{tab:erros_cometidos_tarefas} apresentamos as medidas de número de erros cometidos durante a execução das seis tarefas, comparando-as com valores previamente estabelecidos: nível ruim, almeijado e ótimo. 

\begin{table}[H]
	\Caption{\label{tab:erros_cometidos_tarefas}  Número de Erros Cometidos}%
	\IBGEtab{}{%

\resizebox{\textwidth-8cm}{!}{

\begin{tabular}{|l|c|c|c|c|c|c|}
\hline
\multicolumn{1}{|c|}{\textbf{\begin{tabular}[c]{@{}c@{}}Tarefas/\\ Participantes\end{tabular}}} & \textbf{T1} & \textbf{T2} & \textbf{T3} & \textbf{T4} & \textbf{T5} & \textbf{T6} \\ \hline
Participante 1 & 1 & 1 & 0 & 0 & 0 & 0 \\ \hline
Participante 2 & 1 & 1 & 0 & 4 & 0 & 0 \\ \hline
Participante 3 & 0 & 0 & 0 & 1 & 0 & 0 \\ \hline
Participante 4 & 0 & 0 & 0 & 1 & 0 & 0 \\ \hline
Participante 5 & 0 & 0 & 0 & 0 & 0 & 0 \\ \hline
Participante 6 & 0 & 0 & 0 & 1 & 0 & 0 \\ \hline
\textbf{Média} & \textbf{0.16} & \textbf{0.33} & \textbf{0} & \textbf{1.16} & \textbf{0} & \textbf{0} \\ \hline
Nível ruim & 2 & 2 & 2 & 2 & 2 & 2 \\ \hline
Nível almeijado & 0 & 0 & 0 & 0 & 0 & 0 \\ \hline
Nível ótimo & 0 & 0 & 0 & 0 & 0 & 0 \\ \hline
\end{tabular}

}
	}{%
	\Fonte{Do Próprio Autor}%
	%\Nota{esta é uma nota, que diz que os dados são baseados na
	%	regressão linear.}%
	%\Nota[Anotações]{uma anotação adicional, seguida de várias outras.}%
}
\end{table}

Em relação ao número de erros cometidos durante a realização das tarefas, foram poucos. Somente o Participante 2 durante a realização da tarefa T4 que acabou cometendo muito mais erros do que o nível almeijado. Mas como ele acabou encontrando uma forma alternativa de realizar a tarefa (usando a barra de busca), isso pode ser relevado. Dessa forma, podemos concluir que o sistema possui uma boa segurança no uso nesse quesito. 

\subsection{Número de cliques realizados na execução das tarefas}
Na \autoref{tab:cliques_realizados_tarefas} apresentamos as medidas de número de cliques durante a execução das seis tarefas realizadas, comparando-as com valores previamente estabelecidos: nível ruim, almeijado e ótimo. 

\begin{table}[H]
	\Caption{\label{tab:cliques_realizados_tarefas}  Número de Cliques Realizados}%
	\IBGEtab{}{%

\resizebox{\textwidth-8cm}{!}{

\begin{tabular}{|l|c|c|c|c|c|c|}
\hline
\multicolumn{1}{|c|}{\textbf{\begin{tabular}[c]{@{}c@{}}Tarefas/\\ Participantes\end{tabular}}} & \textbf{T1} & \textbf{T2} & \textbf{T3} & \textbf{T4} & \textbf{T5} & \textbf{T6} \\ \hline
Participante 1 & 3 & 4 & 2 & 4 & 3 & 4 \\ \hline
Participante 2 & 5 & 3 & 2 & 7 & 3 & 4 \\ \hline
Participante 3 & 3 & 2 & 2 & 4 & 3 & 4 \\ \hline
Participante 4 & 3 & 2 & 2 & 4 & 3 & 4 \\ \hline
Participante 5 & 3 & 2 & 2 & 3 & 3 & 4 \\ \hline
Participante 6 & 3 & 2 & 2 & 4 & 3 & 4 \\ \hline
\textbf{Média} & \textbf{3.33} & \textbf{2.50} & \textbf{2.00} & \textbf{4.33} & \textbf{3.00} & \textbf{4.00} \\ \hline
Nível ruim & 6 & 6 & 6 & 6 & 6 & 6 \\ \hline
Nível almeijado & 3 & 3 & 3 & 3 & 3 & 3 \\ \hline
Nível ótimo & 3 & 2 & 2 & 3 & 3 & 3 \\ \hline
\end{tabular}

}
	}{%
	\Fonte{Do Próprio Autor}%
	%\Nota{esta é uma nota, que diz que os dados são baseados na
	%	regressão linear.}%
	%\Nota[Anotações]{uma anotação adicional, seguida de várias outras.}%
}
\end{table}

Apesar do tempo de realização das tarefas serem maior do que o esperado, a quantidade de cliques realizados para completar as tarefas foi próximo do nível almeijado, o que indica que os participantes levaram mais tempo buscando entender a interface do que tomando caminhos alternativos para alcançar seus objetivos.

\subsection{Número de perguntas realizadas durante a execução das tarefas}
Durante a entrevista, não foram feitas muitas perguntas por parte dos participantes. Apenas o Participante 2 na Tarefa 4, que após tentar realizar a atividade por um tempo, questionou o avaliador se poderia usar a barra de busca para achar a lição ao qual a tarefa propunha.

\subsection{Resultados da Entrevista}

\begin{itemize}
	\item Você utiliza regularmente computador? \\
	Para esta pergunta, cinco participantes responderam que sim, apenas o Participante 3 respondeu que não usa tão constantemente.
	\item Você já usou algum outro sistema parecido com este? \\	
	Dois seis participantes, apenas um respondeu que já usou um sistema parecido, mas que já fazia muito tempo e não recordava o nome.
	\item Você acha que o sistema faz o que ele se propõe a fazer? \\
	Todos responderam que sim. Quando questionado, o Participante 1 respondeu: ''Sim, gostei muito da ideia e gostaria de usar depois''.
	\item O que você acha das funções do sistema? \\
	Todos dissetam que, das que viram, gostaram. O Participante 5 disse ainda: ``Acho que ainda tem poucas funcionalidades, mas talvez isso possa deixar mais fácil pra gente usar''.
	\item Tem alguma função que você adicionaria no sistema? \\
	As funções citadas pelos participantes foram: Bate-papo entre os alunos, um sistema de Ranking e videoaulas dos conteúdos. 
	\item Em algum momento o sistema apresentou alguma falha? \\
	Nenhum dos participante relatou falhas do sistema durante seu uso.
	\item Você acha que o sistema agiu adequadamente na ocorrência da falha? \\
	Não foi realizada, já que de acordo com os participantes, o sistema não apresentou falhas.
	\item O que você achou da execução das funções do sistema? \\
	Todos gostaram das funções, apenas o Participante 4 que relatou uma dificuldade em buscar uma lição especifica que acabou ficando fácil com o uso da barra de buscar.
	\item O que você acha em relação a aprendizagem do uso do sistema? \\
	Em geral, os participantes respoderam que por ter poucas funcionalidades, foi fácil aprender a utilizar o sistema. 
	\item O que você achou do formato dos formulários do sistema? \\
	Nessa pergunta, todos respoderam que acharam normais. O participante 6 relatou: ``É bem padrão, acho que não deve ser mudado''.
	\item O que você acha do tempo de resposta do sistema? \\
	Todos relataram que o tempo de resposta do sistema foi rápido.
	\item É fácil encontrar o que se procura no sistema? \\
	Alguns dos participantes relataram uma certa dificuldade em encontrar um lição específica.
	\item O que você não gostou no sistema? \\
	Os participantes informaram que não teve algo que eles não gostaram no sistema.
	\item O que você gostou no sistema? \\
	Dois participantes disseram que gostaram da opção de poder saltar uma pergunta, um deles falou sobre a descrição das perguntas que segundo ele "estão bem explicadas", e dois gostaram das lições possuirem níveis e do sistema de pontuação.
	\item Qual sugestão de melhoria você daria para o sistema? \\
	Nessa pergunta, os participante voltaram a falar na adição de mais funcionalidades ao sistema, como bate-papo, ranking e a adição de conteúdos em vídeo.
\end{itemize}
	\chapter{RESULTADOS}
\label{chap:resultados}

Nesta seção, apresentaremos os resultados obtidos com este trabalho.

Definimos o processo que foi utilizado durante o desenvolvimento do sistema, coletamos, analisamos e validamos os requisitos do sistema e realizamos o projeto do sistema. 

Na implementação, desenvolvemos os módulos: 

\begin{alineascomponto}
    \item Gerenciador de Usuários: módulo responsável por gerenciar os usuários do sistema como professores, assistentes e alunos.
    \item Gerenciador de Turmas: módulo responsável por gerenciar as turmas de alunos do sistema.
    \item Gerenciador de Disciplinas: módulo responsável por gerenciar as disciplinas que serão cadastradas no sistema.
    \item Gerenciador de Lições: módulo responsável por gerenciar as lições que os professores irão cadastrar no sistema.
    \item Gerenciador de Questões: módulo responsável por gerenciar os problemas que os assistentes e professores poderão cadastrar para cada lição.
    \item Gerenciador de Pontuação: módulo responsável por gerenciar a pontuação ganha pelos alunos, assim como seu nível de experiência ao longo da utilização do sistema. 
    \item Fórum: módulo responsável por permitir que alunos postem dúvidas dos mais variados  assuntos relacionadas ao sistema, seja dúvidas em relação ao conteúdo apresentado em sala de aula, assim como informações sobre o sistema e sugestões.
    \item Gerenciador de Progresso: módulo responsável por acompanhar o andamento de cada estudante durante seu aprendizado e identificar os obstáculo epistemológicos enfrentados, indicando ao aluno a exist\^encia desses obst\'aculos e o(s) conte\'udo(s) que ele possui defici\^encia (causador(es) do obstáculo) para ele assim poder pausar o conte\'udo que est\'a estudando e voltar a estudar o(s) conte\'udo(s) que o sistema indicar.
	\item Gerador de Estatísticas: módulo responsável por gerar as estatísticas que o professor utilizará para acompanhar o andamento de suas turmas e alunos, assim como para o uso pelo estudante, 
que utilizará para acompanhar seu próprio progresso durante sua aprendizagem no sistema. 
\end{alineascomponto}

A seguir, apresentaremos algumas telas do sistema:

\begin{figure}[H]
  \centering
  \begin{minipage}[b]{0.49\textwidth}
	\Caption{Tela inicial}
	\UFCfig{}{    
    	\fbox{\includegraphics[width=\textwidth]{figuras/askmath/1}}
    }{
    \Fonte{\url{www.askmath.quixada.ufc.br}}
    }
  \end{minipage}
  \hfill
  \begin{minipage}[b]{0.49\textwidth}
	\Caption{Tela com lições de Proposições}
	\UFCfig{}{    
    	\fbox{\includegraphics[width=\textwidth]{figuras/askmath/2}}
    }{
    \Fonte{\url{www.askmath.quixada.ufc.br}}
    }	
  \end{minipage}

\bigskip
 
  \begin{minipage}[b]{0.49\textwidth}
	\Caption{Tela de problemas do estudante}
	\UFCfig{}{    
    	\fbox{\includegraphics[width=\textwidth]{figuras/askmath/4}}
    }{
    \Fonte{\url{www.askmath.quixada.ufc.br}}
    }
  \end{minipage}
  \hfill
  \begin{minipage}[b]{0.49\textwidth}
	\Caption{Tela de uma postagem no fórum}
	\UFCfig{}{    
    	\fbox{\includegraphics[width=\textwidth]{figuras/askmath/5}}
    }{
    \Fonte{\url{www.askmath.quixada.ufc.br}}
    }
  \end{minipage} 

\bigskip

   \begin{minipage}[b]{0.49\textwidth}
	\Caption{Tela de administra\c{c}\~ao}
	\UFCfig{}{    
    	\fbox{\includegraphics[width=\textwidth]{figuras/askmath/3}}
    }{
    \Fonte{\url{www.askmath.quixada.ufc.br}}
    }
  \end{minipage}
  \hfill
  \begin{minipage}[b]{0.49\textwidth}
	\Caption{Tela de problemas do administrador}
	\UFCfig{}{    
    	\fbox{\includegraphics[width=\textwidth]{figuras/askmath/6}}
    }{
    \Fonte{\url{www.askmath.quixada.ufc.br}}
    }
  \end{minipage}  
\end{figure}
	\chapter{CONCLUSÕES E TRABALHOS FUTUROS}
\label{chap:conclusoes-e-trabalhos-futuros}

Esse capítulo apresenta as considerações finais a cerca desse trabalho. Além disso, é proposta a inclusão de novas funcionalidades que podem ser desenvolvidas em trabalhos futuros.

\section{Conclusões}

Após a implementação da solução proposta por este trabalho, constatou-se que tanto o objetivo geral quanto os específicos foram atendidos. O sistema foi desenvolvido conforme as melhores práticas e padrões em desenvolvimento de \textit{software} disponíveis no momento, tornando o sistema flexível, modularizado e reusável. Além disso, o sistema proporciona de fato um ambiente em que alunos podem práticar os conhecimentos adquiridos e até assimilar novos conhecimento. 

Finalmente, o sistema pode ainda ser utilizado como complemento ao ensino presencial, auxiliando professores e tutores no acompanhamento do aprendizado de seus alunos e aprendizes.

Podem-se evidenciar limitações na solução desenvolvida na questão da comunicação entre aluno-aluno e aluno-profesor, já que isso só ocorre através do fórum de discurções, vale ressaltar que é improvável que, durante a aplicação do sistema, tais limitações sejam atingidas a ponto de comprometer a finalidade do sistema.

\section{Sugestões para Trabalhos Futuros}

No que tange trabalhos futuros que podem ser realizados baseando-se neste projeto, podem ser citados: desenvolver um módulo de bate-papo entre os alunos e professores, adição de conteúdos apresentados através de recursos visuais como vídeo, melhorias na \textit{interface} gráfica para aumentar a usabilidade do sistema. 

Por fim, pode-se avaliar o aprendizado que o sistema pode proporcionar à alunos durante seu uso como complemento ao ensino.




	
	%Elementos pós-textuais	
	\bibliography{elementos-pos-textuais/referencias}
	\imprimirglossario
	\imprimirapendices
		% Adicione aqui os apendices do seu trabalho
 		\apendice{\textbf{DOCUMENTO DE PERSONAS}}\label{ap:personas}
\addtocontents{toc}{\protect\setcounter{tocdepth}{0}}

\section{Charlie - Técnico em Informática}

\begin{figure}[h!]
  \centering
  \Caption{\label{fig:persona_1} Imagem de Charlie}	
  \UFCfig{}{
    \fbox{\includegraphics[width=6cm]{figuras/personas/figura_persona_1}}
  }{
    \Fonte{\url{www.geradordepersonas.com.br}}
  }	
\end{figure}

Empresa: Charlie trabalha na TechSolutions, uma empresa de tecnologia pequena, mas que está ganhando
clientes e ampliando seus negócios.

Idade: 24 anos.

Genêro: Masculino.

Educação: Ensino técnico.

Mídias: Lê a revista Exame e usa ativamente email para trocar informações com 
outros funcionários da empresa.

Objetivos: Como a empresa que Charlie trabalha está ampliando seus negócios, ela
necessita que alguns de seus funcionários possuam um melhor conhecimento na 
área, por isso, Charlie decidiu investir num ensino superior. O objetivo de Charlie é 
concluir seu curso de Engenharia de Software e voltar a trabalhar normalmente 
para sua empresa.

Desafios: Charlie está com muita dificuldade na primeira disciplina de Matemática de seu curso. Como ele já concluiu o ensino médio há bastante tempo, ele não se lembra de como resolver 
questões simples de matemática como funções de 1º grau, 2º grau, racionais, entre outras. Ele começou a estudar, mas 
não consegue encontrar uma forma intuitiva para verificar se está indo bem nos 
seus estudos, ficando preso aos exercícios que ele encontra nos livros.

Como podemos ajudá-lo: O AskMath possibilitará que Charlie estude 
todos os conteúdos que ele j\'a havia estudado no ensino médio e que já os tinha esquecido. São lições agrupadas por disciplina, cada lição possui um conjunto de problemas criados especialmente para 
pessoas com dificuldade em aprender. Nosso sistema possibilitará que Charlie veja instantaneamente se sua resposta é correta ou não, ele também poderá pedir ajuda 
e saltar quest\~oes caso o mesmo não se sinta a vontade para respondê-las naquele 
momento. Com isso, é possível que ele acompanhe através de estatísticas como 
está seu desempenho durante os estudos.

\section{Ruby - Bolsista de Graduação}

\begin{figure}[h!]
  \centering
  \Caption{\label{fig:persona_2} Imagem de Ruby}	
  \UFCfig{}{
   \fbox{\includegraphics[width=6cm]{figuras/personas/figura_persona_2}}
  }{
    \Fonte{ \url{www.geradordepersonas.com.br} }
  }	
\end{figure}


Empresa: Ruby trabalha como bolsista de monitoria das disciplinas envolvendo 
Matemática na Universidade Federal do Ceará.

Idade: 21 anos.

Genêro: Feminino.

Educação: Ensino superior.

Mídias: Usa ativamente o facebook, twitter e wattsapp.

Objetivos: O principal objetivo de Ruby é terminar sua graduação e tentar uma 
bolsa de mestrado numa universidade renomada.

Desafios: Ruby como monitora das disciplinas de matemática, não consegue se
aproximar dos alunos para ajuda-los com suas dúvidas. Segundo ela, "eles tem medo de 
dizer para nós monitores que est\~ao com dificuldades", então Ruby está em busca de 
novas formas de ajudar os estudantes que estão com dificuldades.

Como podemos ajudá-la: Com o AskMath, Ruby poderá ajudar esses alunos adicionando problemas para eles exercitarem seus conhecimentos e tirar suas d\'uvidas no f\'orum de discuss\~ao que 
o sistema oferece.

\section{Samuel - Professor}

\begin{figure}[h!]
  \centering
  \Caption{\label{fig:persona_3} Imagem de Samuel}	
  \UFCfig{}{
   \fbox{\includegraphics[width=6cm]{figuras/personas/figura_persona_3}}
  }{
    \Fonte{ \url{www.geradordepersonas.com.br} }
  }	
\end{figure}

Empresa: Samuel trabalho como professor da disciplina de C\'alculo I na
Universidade Federal do Ceará.

Idade: 59 anos.

Genêro: Masculino.

Educação: Doutorado.

Mídias: Lê o jornal The New York Times e usa email para tirar dúvidas dos alunos.

Objetivos: Samuel é um professor exemplar e se preocupa muito em ensinar seus 
alunos. Seu principal objetivo é ensinar da melhor forma possível seus alunos, 
de forma que todos possam seguir excelentes carreiras quando se formar e se 
tornem ótimos profissionais.

Desafios: Samuel gosta de acompanhar o andamento dos estudos de seus alunos. Quando Samuel os questionam sobre as dificuldade que eles est\~ao enfrentando nos seus estudos, os mesmos dizem que ``está 
tudo bem'', o que não reflete em suas notas. Sabendo disso, Samuel fica triste, já que não consegue saber como anda os estudos de seus alunos e gostaria de ajudá-los ainda mais.

Como podemos ajudá-lo: Com o AskMath, Samuel poder\'a acompanhar o andamento de sua turma e saber em quais conteúdos os alunos est\~ao com mais dificuldade, para dar uma aten\c{c}\~ao especial a 
esses conte\'udos. Samuel poderá ainda adicionar os conteúdos que ele achar mais interessantes, para que seus alunos possam praticar os conhecimentos adquiridos em sala de aula no conforto de suas 
casas. Samuel tamb\'em ver\'a as dúvidas que os alunos postam no fórum de discuss\~oes e quais desses alunos est\~ao enfrentando obstáculos durante sua aprendizagem.

\section{Stefane - Estudante do Ensino Médio}

\begin{figure}[h!]
  \centering
  \Caption{\label{fig:persona_4} Imagem Stefane}	
  \UFCfig{}{
   \fbox{\includegraphics[width=6cm]{figuras/personas/figura_persona_4}}
  }{
    \Fonte{ \url{www.geradordepersonas.com.br} }
  }	
\end{figure}



Empresa: Ela trabalha no salão de beleza de sua mãe.

Idade: 18 anos.

Genêro: Feminino.

Educação: Ensino médio.

Mídias: Usa ativamente o facebook e WatssApp.

Objetivos: Stefane busca concluir o ensino médio e tirar uma boa nota no ENEM 
para tentar conseguir uma vaga para cursar Sistemas de Informação na Universidade Federal do Cear\'a, que é 
o curso de seus sonhos.

Desafios: Stefane, como estudante, não gosta de estudar matemática por livros e acha incomodo ter que levar seus livros enormes de matemática em sua bolsa para o salão de beleza de sua mãe para 
estudar, mas tamb\'em quer ficar estudando enquanto não aparece clientes no salão. Um outro problema de Stefane, é que às vezes, ela fica com dúvidas na resolução de algum problema e precisa ligar 
para seus amigos em busca de explicações que nem sempre encontra.

Como podemos ajudá-la: Com o AskMath, Stefane não precisará mais levar seus livros de Matemática em sua bolsa, para aprender matemática ela precisar\'a apenas de uma celular e isso possibilita que 
ela possa continuar estudando no salão enquanto este estiver vazio. Para solucionar as dúvidas de Stefane, 
temos pessoas esperando para ajudá-la no f\'orum de discuss\~ao, em que ela pode compartilhar suas d\'uvidas para que outros estudantes e professores possam ajuda-la e ao mesmo tempo ajudar outros 
que estejam com a mesma d\'uvida de Stefane. 

\addtocontents{toc}{\protect\setcounter{tocdepth}{1}}
		\input{elementos-pos-textuais/apendices/requisitos}
		\input{elementos-pos-textuais/apendices/arquitetura}
		\apendice{\textbf{TERMO DE CONSENTIMENTO LIVRE E ESCLARECIDO}}\label{ap:tcle}
\addtocontents{toc}{\protect\setcounter{tocdepth}{0}}

Declaro, por meio deste termo, que concordei em participar de um estudo referente ao projeto intitulado AskMath, desenvolvido por Marciano Saraiva. Fui informado(a) de que a pesquisa é orientada por Samy Soares, a quem poderei contatar a qualquer momento que julgar necessário através do e-mail samysoares@gmail.com. Afirmo que aceitei participar por minha própria vontade, sem receber qualquer incentivo financeiro ou ter qualquer ônus, e com a finalidade exclusiva de colaborar para o sucesso da pesquisa. Fui informado(a) dos objetivos estritamente acadêmicos do estudo, que, em linhas gerais é avaliar o sistema de aprendizagem AskMath, na perspectiva da usabilidade. Fui também esclarecido(a) de que os usos das informações por mim oferecidas estão submetidos às normas éticas destinadas à pesquisa envolvendo seres humanos, da Comissão Nacional de Ética em Pesquisa (CONEP) do Conselho Nacional de Saúde, do Ministério da Saúde. Minha colaboração se fará de forma anônima, por meio de teste de usabilidade, entrevista semi-estruturada e observação a ser gravada a partir da assinatura desta autorização. O acesso e a análise dos dados coletados se farão apenas pelo pesquisador e/ou seu orientador. Fui ainda informado(a) de que posso me retirar dessa pesquisa a qualquer momento, sem prejuízo para meu acompanhamento ou sofrer quaisquer sanções ou constrangimentos. 

\begin{center}
	Atenciosamente.
\end{center}

\begin{center}
\noindent\rule{10cm}{0.4pt}\\
Nome e assinatura do pesquisador
\end{center}

\textbf{Consinto em participar deste estudo e declaro ter recebido uma cópia deste termo de consentimento, conforme recomendações da Comissão Nacional de Ética em Pesquisa (CONEP).}

\begin{center}
\noindent\rule{10cm}{0.4pt}\\
 Nome e assinatura do(a) participante
\end{center}

	\imprimiranexos
		% Adicione aqui os anexos do seu trabalho
		\anexo{Exemplo de Anexo}
\label{an:exemplo-de-anexo}

Texto texto texto texto texto texto texto  texto texto texto  texto texto texto  texto texto texto  texto texto texto  texto texto texto  texto texto texto  texto texto texto  texto texto texto  texto texto texto  texto texto texto. 
	\imprimirindice

\end{document}