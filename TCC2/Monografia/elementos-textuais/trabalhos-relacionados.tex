\section{Trabalhos Relacionados}
\label{cap:trabalhos-relacionados}

Os Ambientes Virtuais de Aprendizagem vêm sendo utilizados na educação, principalmente como uma ferramenta mais dinâmica, se comparados \`as metodologias de ensino tradicionais, e são de 
grande potencial na área da educação. Existem diversas pesquisas voltadas a aplicação de AVAs para apoiar o processo de ensino-aprendizagem e, em especial, a Matemática se destaca entre elas.

\subsubsection{\textit{The one world schoolhouse: Education reimagined}}
Em um trabalho iniciado em 2006, \citeonline{khan2012one} fundou a chamada Khan Academy\footnote{Plataforma de aprendizagem disponível em: \url{www.khanacademy.org}}, organização educacional que tem 
por objetivo oferecer exercícios, vídeos de instrução e um painel de aprendizado personalizado que habilita os estudantes a aprender no seu próprio ritmo dentro e fora da sala de aula. Em sua 
plataforma, são abordados assuntos como matemática, ciência, programação de computadores, história, história da arte e economia, entre outros \cite{khan2012one}.

No ambiente, o desempenho do estudante é representado por medalhas. De acordo com o site da organização, medalhas e insignias estimulam o aprendizado de maneira lúdica. As estatísticas mostram o 
quanto de trabalho o estudante está fazendo a cada dia, o quanto o estudante está focado em áreas de habilidades e tópicos e as habilidades que o estudante concluiu. Com os relatórios gerados pela 
plataforma, o tutor pode acompanhar todos os passos do estudante.

Durante uma conferência TED\footnote{É uma série de conferências realizadas na Europa, Ásia e Américas pela fundação Sapling com o objetivo de disseminar ideias que podem mudar o mundo.} em 2011, 
denominada \textit{Salman Khan: Let's Use Video to Reinvent Education} \cite{tedtalk2011reinvend}, Khan fala sobre o funcionamento da plataforma e também do provável motivo do seu sucesso. Segundo 
\citeonline{tedtalk2011reinvend}, o diferencial da plataforma está em dois pontos importantes, a aprendizagem auto-ritmada e nos dados fornecidos aos tutores sobre o aprendizado de seus estudantes. 

Em relação a aprendizagem auto-ritmada, Khan afirma que:
\begin{citacao}
``Então quando fala-se em aprendizagem auto-ritmada, isso faz sentido para todo mundo – na chamada aprendizagem diferenciada – mas é meio maluco quando você vê isso na sala de aula. Porque toda vez 
que fizemos isso, em cada sala de aula que fizemos, repetidas vezes, depois de cinco dias nisso, há um grupo de garotos que está adiantado e há outro grupo de garotos um pouco atrás. E num modelo 
tradicional, se você fizesse uma avaliação pontual, diria: ``Esses são garotos inteligentes, aqueles são garotos lerdos. Talvez eles devessem ser acompanhados de forma diferente. Talvez devêssemos 
coloca-los em salas diferentes.'' Mas quando você deixa cada aluno trabalhar em seu próprio ritmo – e vemos isso repetidas vezes – você vê alunos que tomam um tempo extra em um conceito ou outro, mas 
uma vez que adquirem esse conceito, eles apenas vão adiante. E assim os mesmos garotos que você pensava que eram lerdos semanas atrás, agora você pensa que são inteligentes. E vemos isso repetidas 
vezes. E isso faz você se perguntar quantos estereótipos que talvez vários de nós recebemos eram apenas devido a uma coincidência de tempo'' \cite[13:29, Tradu\c{c}\~ao 
Livre]{tedtalk2011reinvend}.
\end{citacao}.


\citeonline{tedtalk2011reinvend} também fala sobre a importância dos dados fornecidos pelos tutores:
\begin{citacao}
``[...]. Nosso paradigma é armar os professores com a maior quantidade de dados possível – dados que, em quase qualquer outro campo, são esperados, se você trabalha com finanças ou propaganda ou 
fabricação. E assim o professores podem realmente diagnosticar o que está errado com os alunos de maneira que podem fazer suas interações mais produtivas possível. Agora os professores sabem 
exatamente o que os alunos têm feito, quanto tempo eles gastam todo dia, a quais vídeos assistiram, quando pausaram os vídeos, o que fez que parassem de ver, quais exercícios estavam fazendo, no que 
eles estavam se focando? [...]'' \cite[12:26, Tradu\c{c}\~ao Livre]{tedtalk2011reinvend}.
\end{citacao}

O trabalho de \citeonline{khan2012one}, assim como o apresentado nesta monografia, apresenta o uso de um AVA para auxiliar na educação. Dessa forma, esse trabalho servirá como referência para abordar 
o uso da aprendizagem auto-ritmada na educação matemática, assim como para definir as informações que deverão ser apresentadas aos tutores sobre a evolução da aprendizagem de seus estudantes na 
plataforma fruto deste trabalho. Contudo, os vídeos que fizeram da Khan Academy tão popular não farão parte desse trabalho.

O aspecto mais importante a se considerar aqui está na forma como as duas plataformas lidam com Obstáculos Epistemológicos. Segundo \citeonline{bachelard1996formaccao}, durante o ato do conhecimento, 
ocorrem ``lentidões e conflitos'' que levam o estudante a parar diante do problema. A esta ``inércia'' é que foi relacionado o conceito. Na metodologia criada por \citeonline{khan2012one}, quando a 
plataforma é aplicada dentro da sala de aula, o professor pode identificar através da ferramenta, os estudantes que estão com esses obstáculos e o mesmo pode intervir para ajudar esses estudantes a 
superar essa barreira. No trabalho apresentado aqui, essa barreira será superada quando o sistema, ao identificar o obstáculo enfrentado, indicar ao estudante a exist\^encia desse obst\'aculo e o(s) 
conte\'udo(s) que ele possui defici\^encia (causador(es) da barreira) para ele assim poder pausar o conte\'udo que est\'a estudando e voltar a estudar o(s) conte\'udo(s) mais b\'asicos que o sistema 
indicar. Por esse método o  estudante, ao enfrentar um obst\'aculo na aprendizagem, saber\'a quais os conte\'udos estudados anteriormente foram os causadores desse obst\'aculo e poder\'a, dessa 
forma, focar seus estudos nesses conte\'udos para superar essa barreira.

\subsubsection{\textit{ActiveMath: A generic and adaptive web-based learning environment}}

O projeto ActiveMath visa apoiar a aprendizagem verdadeiramente interativa, exploratória, e assume que o estudante deve ser responsável por seu aprendizado. Portanto, uma relativa liberdade para 
navegar através de um curso e para as escolhas de aprendizagem lhes é dada e, por padrão, o modelo de usuário é inspecionável e modificável \cite{melis2001activemath}. 
\citeonline{melis2001activemath} afirmam que a maioria dos sistemas tutores inteligentes não contam com uma escolha de adaptação de conteúdos e isso, segundo \citeonline{melis2001activemath}, pode 
influenciar quando o público-alvo for alunos de faculdades e universidades, já que, diferentemente das escolas de ensino fundamental,  um mesmo assunto é ensinado de forma diferente para diferentes 
grupos de utilizadores e em contextos diferentes \cite{melis2001activemath}.

Para conseguir um ambiente dinâmico de aprendizagem, ActiveMath utiliza regras pedagógicas que definem em quais momentos determinadas funcionalidades do sistema estarão disponíveis, em que ordem os 
conteúdos serão apresentados para os alunos e como os mesmos deverão ser apresentados. O trabalho referido,  assim como o desenvolvido por \citeonline{khan2012one} descrito anteriormente, descreve a 
criação de um AVA, o mesmo utiliza tamb\'em técnicas que permitem a geração dinâmica de cursos para os alunos.

O trabalho desenvolvido por \citeonline{melis2001activemath} apresenta um subsistema de exerc\'icios que suporta diagn\'osticos de erros e equ\'ivocos dos estudantes, que 
gera estrat\'egias tutoriais configur\'aveis para o \textit{feedback}. Em nosso trabalho, quando o ambiente diagnosticar erros e equ\'ivocos frequentes cometidos pelos estudantes relacionados a 
um conte\'udo estudado anteriormente, o sistema alertar\'a ao estudante sobre uma poss\'ivel defici\^encia que ele possua e o orientar\'a a voltar a estudar o conte\'udo indicado.


\subsubsection{\textit{Duolingo: Learn a Language for Free while Helping to Translate the Web}}\label{trabalho_relacionados_duolingo}

Nesse trabalho desenvolvido por \citeonline{von2013duolingo}, \'e apresentado o Duolingo, uma plataforma de ensino de idiomas e tradu\c{c}\~ao autom\'atica de documentos. O ambiente funciona de 
maneira que os usuários progridam nas lições ao mesmo tempo que traduzem conteúdo real da internet. 

O método utilizado pela plataforma se caracteriza pela li\c{c}\~oes fragmentadas, pelas quais os 
alunos, atrav\'es do m\'etodo de repeti\c{c}\~ao, fixam o conte\'udo da língua estudada. \`A medida que o usu\'ario avan\c{c}a, ele progride em uma \'arvore de habilidades que o leva gradativamente 
ao fim do curso, enquanto oferece a op\c{c}\~ao de voltar atr\'as para refazer li\c{c}\~oes antigas que j\'a poderiam ter sidas esquecidas. Um estudo realizado na Universidade da Cidade de Nova York 
\cite{vesselinov2012duolingo} disse que 34 horas gastas no Duolingo igualou-se a um semestre de um curso de l\'inguas.

Uma das características mais marcantes dessa plataforma \'e quantidade de técnicas de Gamifica\c{c}\~ao empregadas. Possui sistema de pontua\c{c}\~ao, n\'iveis, rankings, miss\~oes, medalhas, 
personaliza\c{c}\~ao, entre outras. Assim como a plataforma desenvolvida por \citeonline{von2013duolingo}, o ambiente que desenvolveremos tamb\'em contar\'a com t\'ecnicas de Gamifica\c{c}\~ao que 
ser\~ao utilizadas para melhorar a experi\^encia do usu\'ario, assim como para motivar esses usu\'arios durante sua utiliza\c{c}\~ao do sistema.

\subsubsection{\textit{Resolução de problemas em ambientes virtuais de aprendizagem num curso de licenciatura em matemática na modalidade a distância}}

O trabalho desenvolvido por \citeonline{dutra2011resoluccao}, trata da utilização da metodologia de Resolução de Problemas (apresentada na \autoref{resolucao_problemas}) em um AVA com o objetivo de 
investigar as contribuições que essa jun\c{c}\~ao pode trazer para um curso de Licenciatura em Matem\'atica da Universidade Federal de Ouro Preto (UFOP). Para isso, 
o ambiente desenvolvido por \citeonline{dutra2011resoluccao} utiliza de fóruns semanais para discussão e resolução dos problemas, além de chats, utilizados ao final de algumas atividades, e um 
questionário final a ser respondido pelos alunos na última semana de aula.

Essa metodologia, segundo \citeonline{dutra2011resoluccao}, funciona atrav\'es dos seguintes passos:
\begin{alineascomnumero}
	\item A atividade é postada na Plataforma Moodle\footnote{``A  palavra  Moodle  referia-se  originalmente  ao  acrônimo:  `Modular Object-Oriented  Dynamic  Learning  Environment'(...).  Em  inglês  a  palavra Moodle é também um verbo que descreve a ação que, com frequência, conduz a resultados criativos, de deambular com preguiça, enquanto se faz com gosto o  que  for  aparecendo  para  fazer''. O  Moodle  deu  o  nome  a  uma  plataforma  de  e-learning,  de  utilização livre  e  código  fonte  aberto,  pela  mão  de  Martin  Dougiamas \cite{oro29585}.} no início da semana, pela manhã (segunda-feira).  Assim,  as  atividades  são  distribuídas  aos  alunos  para  que possam ler, interpretar e entender o problema. 
	\item Os  alunos  passam a  semana  postando  suas  resoluções  dos  problemas e discutindo-as no fórum com os colegas, por meio da Plataforma Moodle. 
	\item A  pesquisadora  observa,  incentiva  e  participa  do  processo  de discussão, ajuda nos problemas secundários, dando \textit{feedback} das resoluções postadas, respondendo e fazendo 
perguntas, tirando  dúvidas, acompanhando de perto as discussões entre os alunos no fórum.
	\item As impressões dos alunos sobre os problemas, no início da semana seguinte, e a formalização dos resultados são apresentadas nos chats semanais. Trata-se  de  uma  plenária  virtual  para  discutir  os  problemas,  finalizando-a  com  uma solução aceita por todos. 
	\item Uma resolução é postada na Plataforma Moodle, para todos os pesquisados, observando os conteúdos apresentados nos problemas. 
\end{alineascomnumero}

O trabalho apresentado aqui, assim como o desenvolvido por \citeonline{dutra2011resoluccao}, utiliza a Metodologia de Resolu\c{c}\~ao de Problemas para auxiliar no ensino de matem\'atica. Uma 
outra característica que as duas metodologias t\^em em comum \'e o uso de f\'oruns de discuss\~ao como ferramenta que propicia uma 
constru\c{c}\~ao coletiva de conhecimento, atrav\'es do compartilhamento do conhecimento.
