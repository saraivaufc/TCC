\chapter{CONCLUSÕES E TRABALHOS FUTUROS}
\label{chap:conclusoes-e-trabalhos-futuros}

Esse capítulo apresenta as considerações finais a cerca desse trabalho. Além disso, é proposta a inclusão de novas funcionalidades que podem ser desenvolvidas em trabalhos futuros.

\section{Conclusões}

Após a implementação da solução proposta por este trabalho, constatou-se que tanto o objetivo geral quanto os específicos foram atendidos. O sistema foi desenvolvido conforme as melhores práticas e padrões em desenvolvimento de \textit{software} disponíveis no momento, tornando o sistema flexível, modularizado e reusável. Além disso, o sistema proporciona de fato um ambiente em que alunos podem práticar os conhecimentos adquiridos e até assimilar novos conhecimento. 

Finalmente, o sistema pode ainda ser utilizado como complemento ao ensino presencial, auxiliando professores e tutores no acompanhamento do aprendizado de seus alunos e apredizes.

Podem-se evidenciar limitações na solução desenvolvida na questão da comunicação entre aluno-aluno e aluno-profesor, já que isso só ocorre através do fórum de discurções, vale ressaltar que é improvável que, durante a aplicação do sistema, tais limitações sejam atingidas a ponto de comprometer a finalidade para qualo sistema foi criado.

\section{Sugestões para Trabalhos Futuros}

No que tange trabalhos futuros que podem ser realizados baseando-se neste projeto, podem ser citados: desenvolver um módulo de bate-papo entre os alunos e professores, adição de conteúdos apresentados através de recursos visuais como vídeo, melhorias na \textit{interface} gráfica para aumentar a usabilidade do sistema. 

Por fim, pode-se avaliar o aprendizado que o sistema pode proporcionar à alunos durante seu uso como complemento ao ensino.



